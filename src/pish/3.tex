\newpage
\phantomsection
{\bfseries ҒТАМР 65.63.33}
\hfill {\bfseries \href{https://doi.org/10.58805/kazutb.v.2.23-394}{https://doi.org/10.58805/kazutb.v.2.23-394}}

\sectionwithauthors{Н.E. Альжаксина, A.Б. Далабаев, A.Ж. Хастаева}{ГРЕК ЖАҢҒАҒЫ НЕГІЗІНДЕГІ ӨСІМДІК СУСЫНЫНЫҢ ТАҒАМДЫҚ ЖӘНЕ
БИОЛОГИЯЛЫҚ ҚҰНДЫЛЫҒЫН АНЫҚТАУ}

\begin{center}
{\bfseries \textsuperscript{1}Н.E. Альжаксина\envelope, \textsuperscript{1}A.Б. Далабаев, \textsuperscript{2}A.Ж. Хастаева}

\textsuperscript{1}Астана филиалы ЖШС «Қазақ қайта өңдеу және тағам
өнеркәсіптері ғылыми-зерттеу институты, Астана, Қазақстан,

\textsuperscript{2}Қ. Құлажанов атындағы Қазақ технология және бизнес
университеті, Астана, Қазақстан,

\envelope Корреспондент-автор: alzhaxina@inbox.ru
\end{center}

Лактозаға, сүт казеиніне төзбеушілігі бар тұтынушылар санының артуымен
және өсімдік ақуызына физиологиялық тұрғыдан артықшылық беру арқылы
өсімдік негізіндегі сусындар нарығының қарқынды даму тенденциясы
байқалады. Қазіргі таңда сүттің өсімдік тектес аналогтары танымал бола
бастады, олар әртүрлі дәнді және майлы дақылдардан, жаңғақтардан
өндіріледі, өйткені олардың дәмі жағымды, тағамдық және биологиялық
құндылығы жоғары. Өсімдік негізіндегі сусындардың қолданыстағы
технологияларын жетілдіру кезінде келесі негізгі факторлар ескеріледі:
органолептикалық көрсеткіштерді анықтау, қоректік заттарды нормалары,
эмульсия жүйесінің сақтау қабілеті және ақуыздардың фракциялық құрамы.
Осы зерттеу аясында грек жаңғағы негізде өсімдік сусынының
органолептикалық көрсеткіштері мен тағамдық құндылығы анықталды.
Биологиялық құндылығын анықтау кезінде барлық алмастырылмайтын
аминқышқылдарының мөлшері жоғары екені анықталды. Ақуыздың
аминқышқылдарының жылдамдығы 86,3 \% құрады. Ақуыздың рационалдылық
коэффициенті-67,4 \%. Майдың жалпы салмағындағы полиқанықпаған май
қышқылдарының қосындысы 75,8 \% құрады. 100 г сусынның құрамында ω-3 май
қышқылдары - 0,86 г, ω-6 май қышқылдары - 2,63 г деңгейінде белгіленеді.
Функционалды тамақтану үшін ω-3: ω-6 май қышқылдарының оңтайлы қатынасы
-1: 3,0 екені белгіленді.

{\bfseries Түйін сөздер:} өсімдік сусыны, жаңғақ, тағамдық құндылығы,
биологиялық құндылығы, полиқанықпаған май қышқылдары, маңызды амин
қышқылдары, қоректік компоненттер.

\begin{center}
{\large\bfseries ОПРЕДЕЛЕНИЕ ПИЩЕВОЙ И БИОЛОГИЧЕСКОЙ ЦЕННОСТИ РАСТИТЕЛЬНОГО
НАПИТКА НА ОСНОВЕ ГРЕЦКОГО ОРЕХА}

{\bfseries \textsuperscript{1}Н.E. Альжаксина\envelope, \textsuperscript{1}A.Б. Далабаев, \textsuperscript{2}A.Ж.Хастаева}

\textsuperscript{1}Астанинский филиал ТОО «Казахский
научно-исследовательский институт перерабатывающей и пищевой
промышленности», Астана, Казахстан,

\textsuperscript{2}Казахский университет технологии и бизнеса им. К.
Кулажанова, Астана, Казахстан,

е-mail: alzhaxina@inbox.ru
\end{center}

С увеличением количества потребителей с непереносимостью лактозы,
молочного казеина, физиологической предпочтительности растительного
белка наблюдается тенденция интенсивного развития рынка растительных
напитков. На сегодняшний день растительные аналоги молока становятся
популярными, произведенные из различных злаковых и масличных культур,
плодов орехов, поскольку они обладают приятным вкусом, высокой пищевой и
биологической ценностью. При совершенствовании существующих технологий
напитков на растительной основе учитываются основные факторы:
определение органолептических показателей, нормы содержания питательных
нутриентов, хранительная способность эмульсионной системы и фракционный
состав белков. В рамках данного исследования определены
органолептические показатели и пищевая ценность растительного напитка на
основе грецкого ореха. При определении биологической ценности содержание
всех незаменимых аминокислот высокое. Аминокислотный скор белка
составляет 86,3 \%. Коэффициент рациональности белка - 67,4 \%. Сумма
полиненасыщенных жирных кислот в общей массе жира составляет 75,8 \%.
При этом содержание в 100 г напитка ω-3 жирных кислот установлено на
уровне - 0,86 г, ω-6 жирных кислот - 2,63 г. Оптимальное соотношение для
функционального питания ω-3:ω-6 жирных кислот - 1:3,0.

{\bfseries Ключевые слова:} растительный напиток, грецкие орехи, пищевая
ценность, биологическая ценность, полиненасыщенные жирные кислоты,
незаменимые аминокислоты, питательные компоненты.

\begin{center}
{\large\bfseries DETERMINATION OF THE NUTRITIONAL AND BIOLOGICAL VALUE OF A VEGETABLE DRINK BASED ON WALNUTS}

{\bfseries \textsuperscript{1}N.E. Alzhaxina\envelope, \textsuperscript{1}A.B. Dalabayev, \textsuperscript{2}A.Zh. Khastayeva}

\textsuperscript{1}Astana branch «Kazakh research institute of
processing and food industry» LTD,

Astana, Kazakhstan,

\textsuperscript{2}K. Kulazhanov Kazakh University of Technology and
Business, Astana, Kazakhstan,

е-mail: alzhaxina@inbox.ru
\end{center}

With the increase in the number of consumers with lactose intolerance,
milk casein, and the physiological preference for vegetable protein,
there is a tendency for the intensive development of the herbal drinks
market. To date, vegetable analogues of milk are becoming popular,
produced from various cereals and oilseeds, nuts, because they have a
pleasant taste, high nutritional and biological value. When improving
existing plant-based beverage technologies, the main factors are taken
into account: the determination of organoleptic parameters, the norms of
nutrient content, the storage capacity of the emulsion system and the
fractional composition of proteins. As part of this study, the
organoleptic parameters and nutritional value of vegetable drink based
on walnuts were determined. When determining the biological value, the
content of all essential amino acids is high. The amino acid score of
the protein is 86.3\%. The protein rationality coefficient is 67.4\%.
The sum of polyunsaturated fatty acids in the total fat mass is 75.8\%.
At the same time, the content of ω-3 fatty acids in 100 g of the drink
was set at 0.86 g, and ω-6 fatty acids at 2.63 g. The optimal ratio for
functional nutrition of ω-3:ω-6 fatty acids is 1:3.0.

{\bfseries Key words:} vegetable drink, walnuts, nutritional value,
biological value, polyunsaturated fatty acids, essential amino acids,
nutritional components.

\begin{multicols}{2}
{\bfseries Кіріспе.} Өсімдік негізіндегі сүт аналогтарының танымалдығы
бірнеше жылдан бері халық арасында қарқын алуда. Осыған байланысты
нарықта ұсыныс көлемі де артып келеді, атап айтқанда, пайдаланылатын
шикізаттың ассортименті кеңеюде, жаңа өндіріс технологиялары әзірленуде,
макро- және микроэлементтерді жақсы сақтау және олардың сіңімділігін
арттыру үшін өсімдік өнімдерін өңдеудің қолданыстағы әдістері
жетілдірілуде. Тұтынушылар әр түрлі себептермен, біріншіден, салауатты
өмір салтын ұстанғысы келгендіктен, екіншіден, қоршаған ортаны қорғау
үшін өсімдік сүтін ұнатады. Өсімдік негізіндегі сүттің аналогтары тамақ
өнеркәсібінде вегетариандық диетаны ұстанатын адамдарға арналған тамақ
өнімдерін өндіруде кеңінен қолданылады. Бұл тағамдық жүйе ағзаның
денсаулығын жақсарту үшін құрамында жануар текті шикізаты бар өнімдерді
толығымен алмастырады {[}1{]}. Өсімдік негізіндегі алмастырғыштар
йогурт, ірімшік, сүт өнімдері мен десерттерге арналған ингредиенттер
болып табылады. Бүгінгі таңда сүт өнімдерінің барлық дерлік түрлері
әртүрлі вариацияда ұсынылған және өсімдік негізінде жасалған деп айта
аламыз {[}2{]}.

Сондай-ақ, соңғы онжылдықта қазіргі реалиядағы халықтың тамақтану
үлгілері айтарлықтай өзгерістерге ұшырағанын атап өткен жөн. Денсаулықты
сақтау үшін тек адекватты тамақтану ғана емес, сонымен қатар оның
профилактикалық және детоксикациялау функциялары да өте маңызды болды.
Барлық осы алғышарттар негізінен теңдестірілген диета құрылымына
заманауи талаптарды анықтайды. Мысалы, жаңғақтарды пайдаланып,
дәрумендермен және аминқышқылдарымен байытылған аралас өнімдерді жасауға
болады. Жаңғақтар -- өсімдік тектес ақуыздар мен майлардың ең бай
көздерінің бірі болып табылады {[}3{]}.

Бұл зерттеулердің мақсаты -- коллоидтық жүйенің жоғары тұрақтылығын,
өсімдік негізіндегі сусынның тағамдық және биологиялық құндылығының
оңтайлы көрсеткіштерін қамтамасыз ету үшін өсімдік шикізатының жекелеген
түрлерінің ақуыздарының фракциялық құрамын анықтау.

Кілегейлі дәмі және көптеген пайдалы қасиеттері бар грек жаңғақ
негізіндегі өсімдік сусыны жүйке жүйесіне тыныштандыратын әсер етеді,
ашушаңдықты азайтады, созылмалы шаршауды басады және денені қуатпен
толтырады. Жаңғақ негізіндегі өсімдік сусыны метаболизм үрдістерінің
және иммундық жүйенің жұмысын қалыпқа келтіреді. Құрамындағы кальций мен
фосфордың арқасында сусынды ішу эмаль мен сүйек тінін нығайтуға
көмектеседі {[}4, 5{]}.

{\bfseries Материалдар мен әдістер.} Негізгі зерттеу нысаны ретінде
жаңғақтар, оның ішінде грек жаңғағы, грек жаңғағы қосылған өсімдік
сусыны және қосымша шикізат ретінде дәнді және бұршақ дақылдары
қарастырылды. Пайдаланылған дақыл үлгілері өсімдік сусынының коллоидтық
жүйесінің тұрақтылығын анықтау үшін 22-24 °C және 0...+4 °C аралығындағы
температура жағдайында сақталды.

Сынамаларды сақтауды аяқтаудың қолайлы критерийі ретінде қатты
бөлшектердің көрінетін тұнбаның пайда болуы қарастырылды. Содан кейін
өсімдік негізіндегі сусынның сұйық және қатты фазаларының бөлінуі
байқалды.

Алынған өсімдік негізіндегі сусындар үлгілері тағамдық құндылығына, атап
айтқанда тағамдық компоненттердің құрамына, май қышқылдарының профиліне,
аминқышқылдарының құрамына бағаланды. Бақылау үлгісі ретінде «Здоровое
меню» маркалы күріш негізіндегі өсімдік сусыны алынды.

Барлық зерттеулер стандартты әдістерді қолдану арқылы жүргізілді: МемСТ
10846-91 бойынша ақуыз мөлшері, МемСТ 10857-64 бойынша май мөлшері,
МемСТ 31675-2012 бойынша тағамдық талшық мөлшері, МемСТ 15113.9-77
бойынша күлділігі (минералды қалдық). М-04-38-2009 әдісі бойынша
аминқышқылдарының құрамы анықталды. Бұл әдіс үлгілерді қышқылдық
(сілтілі) гидролиз арқылы аминқышқылдарының бос түрге ауысуымен
ыдырауына негізделген. Олардың одан әрі бөлінуі капиллярлық электрофорез
арқылы анықталады. Грек жаңғағы негізіндегі өсімдік сусынының
биологиялық құндылығы есептеу әдісімен анықталды.

Алмастырылмайтын аминқышқылдарын (ААҚ) пайдалану коэффициенті (Kу) 1
формула бойынша анықталды:

\begin{equation}
Ky = \frac{Cmin}{Ci}
\end{equation}

мұндағы, Сmin -- аминқышқылдарының ең төменгі жылдамдық көрсеткіші;

Ci -- i-ші алмастырылмайтын аминқышқылы үшін жылдамдық көрсеткіші
{[}6{]}.

Май қышқылдарының құрамы МемСТ 31663-2012 «Өсімдік майлары және
жануарлар майлары. Май қышқылдарының метил эфирлерінің массалық үлесін
газ хроматографиясы арқылы анықтау» бойынша анықталды {[}7{]}.

{\bfseries Нәтижелер және талқылау.} Өсімдік негізіндегі сусын коллоидтық
тұрақтылығы компоненттердің қасиеттері мен құрамымен ғана емес, сонымен
қатар суспензиялы бөлшектердің гранулометриялық сипаттамаларымен де
анықталатын суспензиялы эмульсия {[}8, 9{]}. Өсімдік негізіндегі сусынды
өндіру үшін қолданылатын әдістер, ең алдымен, пайдаланылатын шикізаттан
басқа ақуыздармен салыстырғанда молекулалық массасы төмен және
сәйкесінше жеңіл және толық сіңімділігі бар суда және тұзда еритін ақуыз
фракцияларын алуға негізделген. Бұл фракциялар сусындардың коллоидты
тұрақтылығын қамтамасыз етеді және жаңғақ ақуыздарының құрамында басым
болады (1-кесте).
\end{multicols}

\begin{table}[H]
\caption*{1-кесте -- Өсімдік негізіндегі сусын өндіру үшін қолданылатын өсімдік шикізатының кейбір түрлерінің ақуыздарының фракциялық құрамы}
\centering
\begin{tabular}{|lllll|}
\hline
\multicolumn{1}{|l|}{\multirow{2}{*}{Шикізат}} & \multicolumn{4}{l|}{Ақуыз фракциясының мөлшері, \% жалпы ақуыздың мөлшерінен} \\ \cline{2-5}
\multicolumn{1}{|l|}{} & \multicolumn{1}{l|}{альбуминдер} & \multicolumn{1}{l|}{глобулиндер} & \multicolumn{1}{l|}{глютелиндер} & проламиндер \\ \hline
\multicolumn{5}{|l|}{\textit{Жаңғақтар:}} \\ \hline
\multicolumn{1}{|l|}{Грек жаңғағы} & \multicolumn{1}{l|}{90,1} & \multicolumn{1}{l|}{2,3} & \multicolumn{1}{l|}{7,6} & - \\ \hline
\multicolumn{5}{|l|}{\textit{Дәнді дақылдар:}} \\ \hline
\multicolumn{1}{|l|}{Күріш дәні} & \multicolumn{1}{l|}{8-10} & \multicolumn{1}{l|}{7-10} & \multicolumn{1}{l|}{60-75} & 3-5 \\ \hline
\multicolumn{1}{|l|}{Жүгері дәні} & \multicolumn{1}{l|}{6,6-10,0} & \multicolumn{1}{l|}{3,5-5,0} & \multicolumn{1}{l|}{28,0-40,3} & 28,9-40,0 \\ \hline
\multicolumn{5}{|l|}{\textit{Бұршақ дақылдары:}} \\ \hline
\multicolumn{1}{|l|}{Бұршақ} & \multicolumn{1}{l|}{38,8} & \multicolumn{1}{l|}{53,4} & \multicolumn{1}{l|}{4,6} & - \\ \hline
\multicolumn{1}{|l|}{Ноқат} & \multicolumn{1}{l|}{49,1} & \multicolumn{1}{l|}{40,5} & \multicolumn{1}{l|}{7,3} & - \\ \hline
\end{tabular}
\end{table}

\begin{multicols}{2}
1-кестеде көрсетілгендей, грек жаңғағының құрамында, басқаларымен
салыстырғанда, альбумин фракциясының мөлшері жоғары 90,1\%. Альбуминдер
адам ағзасына пайдалы, тез қорытылады және сіңіріледі {[}10{]}. Зерттеу
нәтижесінде грек жаңғағы қосылған өсімдік сусынының коллоидтық
тұрақытылығы жоғары болатыны анықталды. Себебі, жаңғақ, дәнді және
бұршақ дақылдарын ұнтақтап, оларды сусынға қосып суспензия жасаған
кезде, ақуыз фракцияларының мөлшері коллоидтық жүйенің біркелкілігіне
әсер етеді. Грек жаңғағын қосып өсімдік сүтін дайындаған кезде, ақуыз
фракцияларының көп мөлшері альбуминнен тұратындықтан, сусынның
біркелкілігі артады және сақтау кезінде тұнбаның пайда болу мүмкіншілігі
төмендейді. Сондықтан, зерттеу барысында өсімдік шикізаты ретінде грек
жаңғағын қосу арқылы өсімдік сусыны дайындалды және оның биологиялық
және тағамдық құндылығы анықталды.

Ғылыми дәлелдер өсімдік негізіндегі сусындарда әдетте ақуыздың мөлшері
төмен және алмастырылмайтын аминқышқылдары да жетіспейтінін көрсетеді.
Зерттелетін нысандардың тағамдық құндылығын талдау кезінде грек
жаңғағында ақуыздың, липидтердің және тағамдық талшықтардың жоғары
мөлшері байқалды. Сәйкесінше, грек жаңғағы негізіндегі өсімдік сусынының
құрамында 1,28\%-ға дейін ақуыз, 4,5\%-ға дейін май және 0,78\%-ға дейін
тағамдық талшық бар екендігі анықталды (2-кесте).
\end{multicols}

\begin{table}[H]
\caption*{2-кесте - Грек жаңғағы негізіндегі өсімдік сүтінің тағамдық құндылығы}
\centering
\resizebox{\textwidth}{!} & \multicolumn{2}{p{0.3\textwidth}|}{100 г тұтыну кезіндегі тәуліктік қажеттілікті қанағаттандыру дәрежесі, \%} \\ \cline{2-7}
 & \multicolumn{1}{l|}{ақуыз} & \multicolumn{1}{l|}{май} & \multicolumn{1}{p{0.15\textwidth}|}{тағамдық талшықтар} & Минералды қалдық & \multicolumn{1}{l|}{ақуызбен (88 г)} & Тағамдық талшықтармен (25 г) \\ \hline
Грек жаңғағы & \multicolumn{1}{l|}{21,9±0,2} & \multicolumn{1}{l|}{32,0±0,7} & \multicolumn{1}{l|}{31,8±0,3} & 5,3±0,1 & \multicolumn{1}{l|}{24,9} & 127,2 \\ \hline
Күріш негізіндегі өсімдік сусыны (бақылау үлігісі) & \multicolumn{1}{l|}{0,3±0,1} & \multicolumn{1}{l|}{1,1±0,2} & \multicolumn{1}{l|}{0,15±0,1} & 0,41±0,1 & \multicolumn{1}{l|}{0,3} & 0,6 \\ \hline
Грек жаңғағы негізіндегі өсімдік сусыны & \multicolumn{1}{l|}{1,28±0,1} & \multicolumn{1}{l|}{4,5±0,2} & \multicolumn{1}{l|}{0,78±0,1} & 0,68±0,1 & \multicolumn{1}{l|}{1,6} & 3,5 \\ \hline
\end{tabular}
}
\end{table}

\begin{multicols}{2}
Жоғарыда келтірілгендей, бақылау үлгісімен салыстырғанда грек жаңғағы
негізіндегі өсімдік сусынының тағамдық құндылығы, атап айтқанда, ақуыз
мөлшері 4,2 есе, май мөлшері 4,1 есе, тағамдық талшықтар мөлшері 5,2 есе
және минералдық қалдық мөлшері 1,6 есе жоғары. Өсімдік шикізатынан
жасалған сусындарды қамтитын функционалды тамақ өнімдерін әзірлеу
кезінде маңызды мәселе эссенциалды қоректік заттарды, атап айтқанда,
полиқанықпаған май қышқылдарын (ПҚМҚ) толықтыру болып табылады. ω-3, ω-6
май қышқылдарының организмдегі биологиялық рөлі реттеуші құрылымдардың
синтезі, стресске қарсы және адаптогендік механизмдердің қалыптасуы
зерттеулерде дәлелденген {[}11, 12{]}. Эксперименттік зерттеулер
барысында жаңғақ негіздегі сусындардың липидті фракциясының жоғары
биологиялық тиімділігі анықталды. Осылайша, жалпы май массасындағы ПҚМҚ
мөлшері 75,8\% құрады. Бұл ретте 100 г сусынның құрамындағы ω-3 май
қышқылдарының мөлшері 0,86 г деңгейінде, ω-6 май қышқылдарының мөлшері
ω-3:ω-6 май қышқылдарының қатынасы функционалдық және профилактикалық
тамақтану үшін оңтайлы қатынасқа сәйкес келетін 1:3,0 шегіне жетеді.
Зерттеулердің келесі кезеңінде грек жаңғағы негізіндегі өсімдік
сусынының биологиялық құндылығы анықталды (3-кесте).
\end{multicols}

\begin{table}[H]
\caption*{3-кесте - Грек жаңғағы негізіндегі өсімдік сусынының биологиялық құндылығы}
\centering
\resizebox{\textwidth}{!} & \multirow{2}{=}{Алмастырылмайтын аминқышқылдарын пайдалану коэффициенті (Kу)} \\ \cline{2-3}
 & \multicolumn{1}{l|}{өнімнің г/100г} & ақуыздың г/100г &  &  &  \\ \hline
Валин & \multicolumn{1}{l|}{0,0684} & 5,43 & 4,8 & 105,6 & 0,79 \\ \hline
Изолейцин & \multicolumn{1}{l|}{0,1634} & 11,7 & 3,6 & 118,2 & 0,68 \\ \hline
Лейцин & \multicolumn{1}{l|}{0,1634} & 11,7 & 6,7 & 118,2 & 0,68 \\ \hline
Лизин & \multicolumn{1}{l|}{0,0586} & 4,65 & 5,3 & 86,7 & 1 \\ \hline
Метионин + цистеин & \multicolumn{1}{l|}{0,0746} & 5,63 & 3,2 & 158,6 & 0,59 \\ \hline
Фенилаланин + тирозин & \multicolumn{1}{l|}{0,1283} & 8,36 & 5 & 153,7 & 0,54 \\ \hline
Треонин & \multicolumn{1}{l|}{0,0794} & 5,72 & 3 & 138,8 & 0,63 \\ \hline
Триптофан & \multicolumn{1}{l|}{0,0254} & 1,87 & 1 & 192 & 0,43 \\ \hline
\end{tabular}%
}
\end{table}

\begin{multicols}{2}
Біздің зерттеулерімізде тәжірибелік-есептеу әдістері грек жаңғағы
негізіндегі өсімдік сусынының ақуыздық фракциясы жоғары биологиялық
құндылығымен сипатталатынын, ақуыздың құрамында бір шектеуші амин
қышқылы - лизин, оның аминқышқылдың жылдамдық көрсеткіші 86,3\%,
алмастырылмайтын аминқышқылдарын пайдалану коэффициенті 67,4\% құрайтыны
анықталды.

{\bfseries Қорытынды.} Грек жаңғағы негізіндегі өсімдік сусынының тағамдық
және биологиялық құндылығы зерттелді. Грек жаңғағының құрамында альбумин
фракциясының мөлшері 90,1\% құрады. Аталған сусынның құрамында 1,28\%-ға
дейін ақуыз, 4,5\%-ға дейін май және 0,78\%-ға дейін тағамдық талшық бар
екендігі анықталды. Бақылау үлгісімен салыстырғанда грек жаңғағы
негізіндегі өсімдік сусынының тағамдық құндылығы, атап айтқанда, ақуыз
мөлшері 4,2 есе, май мөлшері 4,1 есе, тағамдық талшықтар мөлшері 5,2 есе
және минералдық қалдық мөлшері 1,6 есе жоғары екендігі белгіленді. Май
массасындағы ПҚМҚ мөлшері 75,8\% құрады. 100 г сусынның құрамында ω-3
май қышқылдары - 0,86 г, ω-6 май қышқылдары - 2,63 г деңгейінде
белгіленеді. Функционалды тамақтану үшін ω-3: ω-6 май қышқылдарының
оңтайлы қатынасы -1: 3,0 екені белгіленді. Лизинның аминқышқылдық
жылдамдық көрсеткіші 86,3\%, алмастырылмайтын аминқышқылдарын пайдалану
коэффициенті 67,4\% құрады. Ақуыздың жоғары биологиялық құндылығы және
грек жаңғағы негізіндегі өсімдік сусының ақуыздық фракциясының
биологиялық тиімділігі дәлелденді.
\end{multicols}

\begin{center}
{\bfseries Әдебиеттер}
\end{center}

\begin{noparindent}
1. Хастаева А.Ж., Бектурганова А.А., Омаралиева А.М., Сериков А.Ж.,
Мухтарханова Р.Б., Байхожаева Б.А. Исследование пищевой и биологической
ценности зерновых напитков // Вестник Алматинского технологического
университета. -- 2023. -- № 1. -- С. 33-40. DOI
10.48184/2304-568X-2023-1-33-40.

2. Тулякова, Т. В., Буравова Н. А., Колесникова А. А. Растительные
альтернативы традиционного молока // Вестник Медицинского института
непрерывного образования. -- 2023. -- Т. 3. - № 1. -- С. 107-114. -- DOI
10.36107/2782-1714\_2023-3-1-107-112.

3. Хастаева А.Ж., Омаралиева А.М., Бектурганова А.А., Кабдолова А.М.
Обоснование выбора сырья для производства растительного молока //
Вестник Алматинского технологического университета. -- 2021. - №4. -- С.
53-57. DOI 10.48184/2304-568X-2021-4-53-57.

4. Косарева О.А., Ситникова К.С. Безлактозное, органическое и
растительное молоко, как альтернатива молоку цельному // Вестник
Национального Института Бизнеса. -- 2024. -- № 1(53). -- С. 125-128.

5. Zhang H. High-pressure treatment effects on proteins in soy milk //
LWT - Food Science and Technology. - 2005. - V. 38. - Р. 7-14.

6. Семянникова Н.Р., Ключко Н.Ю. Изучение возможности совершенствования
технологии аналогов молочных напитков на основе растительного сырья //
Вестник молодежной науки. -- 2023. -- № 5(42). -- 5 с. DOI
10.46845/2541-8254-2023-5(42)-6-6.

7. Носова О.С. Сравнительный анализ органолептических показателей молока
животного и растительного происхождения // Вестник молодежной науки
Алтайского государственного аграрного университета. -- 2023. -- № 1. --
С. 130-133.

8. Шишкина Д. И., Штовхун А. И., Клейн Е. Э., Беркетова Л. В.
Современные технологии производства альтернативного молока из
растительных продуктов // Вестник Воронежского государственного
университета инженерных технологий. -- 2022. -- Т. 84. -№ 4(94). -- С.
141-148. DOI 10.20914/2310-1202-2022-4-141-148.

9. Агутова, С. И., Глотова И. А., Галочкина Н. А. Фортификация пищевой
ценности и функциональных свойств соевого белкового напитка //
Технологии пищевой и перерабатывающей промышленности АПК -- продукты
здорового питания. -- 2022. -- № 4. -- С. 100-106. DOI
10.24412/2311-6447-2022-4-100-106.

10. Afolabi I.S. Production of a new plant-based milk from Adenanthera
pavonina seed and evaluation of Its nutritional and health benefit //
Frontiers in Nutrition. - 2018. - Vol. 5. DOI 10.3389/fnut.2018.00009.

11. Ғани Г. М., Жакипбеков К. С., Датхаев У. М., Аширов М.З., Жакып
Н.А., Кусайнов А.З. Euphorbia humifusa Willd. құрамындағы химиялық
компоненттер және олардың арнайы фармацевтикалық қызметтері // Қазақстан
фармациясы. -- 2022. -- № 2. -- Б. 150-155.

12. Capriotti A.L. Protein profile of mature soybean seeds and prepared
soybean milk // Journal of Agricultural and Food Chemistry. - 2014. - V.
62 (40). - Р. 9893-9899. DOI 10.1021/jf5034152.
\end{noparindent}

\begin{center}
{\bfseries References}
\end{center}

\begin{noparindent}
1. Khastaeva A.Zh., Bekturganova A.A., Omaralieva A.M., Serikov A.Zh.,
Mukhtarkhanova R.B., Baikhozhaeva B.A. Issledovanie pishchevoi i
biologicheskoi tsennosti zernovykh napitkov // Vestnik Almatinskogo

tekhnologicheskogo universiteta. -- 2023. -- № 1. -- S. 33-40. DOI
10.48184/2304-568X-2023-1-33-40. {[}in Russian{]}

2. Tulyakova, T. V., Buravova N. A., Kolesnikova A. A.
Rastitel\textquotesingle nye al\textquotesingle ternativy traditsionnogo
moloka // Vestnik Meditsinskogo instituta nepreryvnogo obrazovaniya. --
2023. -- T. 3. - № 1. -- S. 107-114. -- DOI
10.36107/2782-1714\_2023-3-1-107-112. {[}in Russian{]}

3. Khastaeva A.Zh., Omaralieva A.M., Bekturganova A.A., Kabdolova A.M.
Obosnovanie vybora syr\textquotesingle ya dlya proizvodstva
rastitel\textquotesingle nogo moloka // Vestnik Almatinskogo
tekhnologicheskogo universiteta. -- 2021. - №4. -- S. 53-57. DOI
10.48184/2304-568X-2021-4-53-57. {[}in Russian{]}

4. Kosareva O.A., Sitnikova K.S. Bezlaktoznoe, organicheskoe i
rastitel\textquotesingle noe moloko, kak al\textquotesingle ternativa
moloku tsel\textquotesingle nomu // Vestnik
Natsional\textquotesingle nogo Instituta Biznesa. -- 2024. -- № 1(53).
-- S. 125-128. {[}in Russian{]}

5. Zhang H. High-pressure treatment effects on proteins in soy milk //
LWT - Food Science and Technology. - 2005. - V. 38. - R. 7-14.

6. Semyannikova N.R., Klyuchko N.Yu. Izuchenie vozmozhnosti
sovershenstvovaniya tekhnologii analogov

molochnykh napitkov na osnove
rastitel\textquotesingle nogo syr\textquotesingle ya // Vestnik
molodezhnoi nauki. -- 2023. -- № 5(42). -- 5 s. DOI
10.46845/2541-8254-2023-5(42)-6-6. {[}in Russian{]}

7. Nosova O.S. Sravnitel\textquotesingle nyi analiz organolepticheskikh
pokazatelei moloka zhivotnogo i rastitel\textquotesingle nogo

proiskhozhdeniya // Vestnik molodezhnoi nauki Altaiskogo
gosudarstvennogo agrarnogo universiteta. -- 2023. -- № 1. -- S. 130-133.
{[}in Russian{]}

8. Shishkina D. I., Shtovkhun A. I., Klein E. E., Berketova L. V.
Sovremennye tekhnologii proizvodstva al\textquotesingle

ternativnogo
moloka iz rastitel\textquotesingle nykh produktov // Vestnik
Voronezhskogo gosudarstvennogo universiteta

inzhenernykh tekhnologii. --
2022. -- T. 84. -№ 4(94). -- S. 141-148. DOI
10.20914/2310-1202-2022-4-141-148. {[}in Russian{]}

9. Agutova, S. I., Glotova I. A., Galochkina N. A. Fortifikatsiya
pishchevoi tsennosti i funktsional\textquotesingle nykh svoistv soevogo
belkovogo napitka // Tekhnologii pishchevoi i pererabatyvayushchei
promyshlennosti APK -- produkty zdorovogo pitaniya. -- 2022. -- № 4. --
S. 100-106. DOI 10.24412/2311-6447-2022-4-100-106. {[}in Russian{]}

10. Afolabi I.S. Production of a new plant-based milk from Adenanthera
pavonina seed and evaluation of Its nutritional and health benefit //
Frontiers in Nutrition. - 2018. - Vol. 5. DOI 10.3389/fnut.2018.00009.

11. Ғani G. M., Zhakipbekov K. S., Datkhaev U. M., Ashirov M.Z., Zhakyp
N.A., Kusainov A.Z. Euphorbia humifusa Willd. kuramyndagy khimiyalyk
komponentter zhane olardyn arnaiy farmatsevtikalyk kyzmetterі //
Kazakstan farmatsiyasy. -- 2022. -- № 2. -- B. 150-155. {[}in Kazakh{]}

12. Capriotti A.L. Protein profile of mature soybean seeds and prepared
soybean milk // Journal of Agricultural and Food Chemistry. - 2014. - V.
62 (40). - R. 9893-9899. DOI 10.1021/jf5034152.
\end{noparindent}

\emph{{\bfseries Авторлар туралы мәліметтер}}

\begin{noparindent}
Альжаксина Н.E. - PhD, директорының қ.а., Астана филиалы ЖШС «Қазақ
қайта өңдеу және тағам өнеркәсіптері ғылыми-зерттеу институты», Астана,
Қазақстан, е-mail: alzhaxina@inbox.ru;

A.Б. Далабаев - магистр, аға ғылыми қызметкер, Астана филиалы ЖШС «Қазақ
қайта өңдеу және тағам өнеркәсіптері ғылыми-зерттеу институты», Астана,
Қазақстан, е-mail: dalabaev\_askhat@mail.ru;

A.Ж. Хастаева - PhD, «Технология және стандарттау» кафедрасының
қауым.профессоры; Қ. Құлажанов атындағы Қазақ технология және бизнес
университеті, Астана, Қазақстан, e-mail: gera\_or@mail.ru.
\end{noparindent}

\emph{{\bfseries Information about the authors}}

\begin{noparindent}
Nazym Alzhaxina - PhD, Director of the Astana branch of «Kazakh Research
Institute of Processing and Food Industry», Astana, Kazakhstan; e-mail:
alzhaxina@inbox.ru;

Askhat~Dalabayev - Masters degree, Senior Researcher, Astana branch of
«Kazakh Research Institute of Processing and Food Industry», Astana,
Kazakhstan; е-mail: dalabaev\_askhat@mail.ru;

Aigerim Khastayeva -- PhD, ass. professor of the Department of
Technology and Standardization; K. Kulazhanov Kazakh University of
Technology and Business; Astana, Kazakhstan; e-mail: gera\_or@mail.ru.
\end{noparindent}
