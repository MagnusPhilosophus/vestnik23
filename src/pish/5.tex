\newpage
{\bfseries МРНТИ 65.09.03}
\hfill {\bfseries \href{https://doi.org/10.58805/kazutb.v.2.23-493}{https://doi.org/10.58805/kazutb.v.2.23-493}}

\sectionwithauthors{T.Ch. Tultabayeva, G.N Zhakupova, N.Kokumbekova, A.T.Sagandyk, A.H. Muldasheva, A.T.Akhmetzhanova}{STUDYING EFFECT OF FREEZE DRYING ON THE CHEMICAL COMPOSITION OF
COW COLOSTRUM}

\begin{center}
{\bfseries T.Ch. Tultabayeva, G.N Zhakupova, N.Kokumbekova, A.T.Sagandyk, A.H. Muldasheva, A.T.Akhmetzhanova\envelope}

NJSC "Kazakh Agrotechnical Research University named after S.
Seifullin",

Astana, Kazakhstan

\envelope Corresponding author: aygerim\_talgatqyzy@mail.ru
\end{center}

The article presents studies of the quality of cow colostrum. To date, a
promising direction in the development of food functional products
technology is the processing of cow colostrum as an additional source of
protein, immunoglobulins, lipids, vitamins, minerals and other
biologically active substances. It has been established that the
physico-chemical composition of colostrum depends on the time elapsed
since the calving of the cow. According to the conducted studies, it was
found that the protein content is significantly higher in colostrum
obtained immediately after calving than in colostrum collected after 24
hours and 36 hours after calving. The article provides data on the study
of dry cow colostrum obtained by freeze-drying. In this regard, the
authors of the article investigated the chemical composition of dry
colostrum, namely, the concentration of protein, fat and ash was
determined depending on the time of collection of colostrum. The authors
found that colostrum obtained immediately after calving also has low
humidity, which makes it possible to increase the shelf life. The
article substantiates the ways of its subsequent use in the production
of food products with high nutritional and biological value and possible
immunomodulatory effect.

{\bfseries Key words:} cow colostrum, dry colostrum, biologically active
substances, proteins.

\begin{center}
{\large\bfseries ИССЛЕДОВАНИЕ ВЛИЯНИЯ СУБЛИМАЦИОННОЙ СУШКИ НА ХИМИЧЕСКИЙ СОСТАВ КОРОВЬЕГО МОЛОЗИВА}

{\bfseries T.Ч. Тултабаева, Г. Н. Жакупова, Н. Кокумбекова, А. Т. Сагандык, А. Х. Мулдашева, А. Т. Ахметжанова\envelope}

НАО «Казахский агротехнический исследовательский университет им. Сакена
Сейфуллина» , Астана, Казахстан,

e-mail: aygerim\_talgatqyzy@mail.ru
\end{center}

В представленной статье рассмотрены исследования качества коровьего
молозива. На сегодняшний день перспективным направлением развития
технологии пищевых функциональных продуктов является переработка
молозива коров как дополнительного источника белка, иммуноглобулинов,
липидов, витаминов, минеральных и других биологически активных веществ.
Установлено, что физико-химический состав молозива зависит от времени,
прошедшего с момента отела коровы. В соответствии с проведенными
исследованиями обнаружено, что содержание белка существенно выше в
молозиве, полученном сразу после отела, чем в молозиве собранном после
24 часов и 36 часов с момента отела. В статье приводятся данные по
исследованию сухого коровьего молозива, полученного методом
сублимационной сушки. В связи с этим авторами статьи исследован
химический состав сухого молозива, а именно определена концентрация
белка, жира и золы в зависимости от времени сбора молозива. Авторами
установлено, что молозиво, полученное сразу после отела имеет также
низкую влажность, что дает возможность увеличения сроков хранения. В
статье обоснованы пути последующего его использования в производстве
пищевой продукции с высокой пищевой и биологической ценностью и
возможным иммуномодулирующим действием.

{\bfseries Ключевые слова:} коровье молозиво, сухое молозиво, биологически
активные вещества, белки.
\newpage
\begin{center}
{\large\bfseries СУБЛИМАЦИЯЛЫҚ КЕПТІРУДІҢ СИЫР УЫЗЫНЫҢ ХИМИЯЛЫҚ ҚҰРАМЫНА ӘСЕРІН ЗЕРТТЕУ}

{\bfseries Тултабаева Т.Ч., Жакупова Г.Н., Кокумбекова Н., Сағандық А.Т., Мулдашева А.Х., Ахметжанова А.Т.\envelope}

\textsuperscript{1} КеАҚ «С.Сейфуллин атындағы Қазақ агротехникалық
зерттеу университеті»,

Астана, Қазақстан,

e-mail: aygerim\_talgatqyzy@mail.ru
\end{center}

Ұсынылған мақалада сиыр уызының сапасы туралы зерттеулер қарастырылған.
Бүгінгі таңда тағамдық функционалды өнімдер технологиясын дамытудың
перспективалы бағыты сиырдың уыз сүтін ақуыздың, иммуноглобулиндердің,
липидтердің, дәрумендердің, минералды және басқа да биологиялық белсенді
заттардың қосымша көзі ретінде өңдеу болып табылады. Уыздың
физика-химиялық құрамы сиырды төлдегеннен кейінгі уақытқа байланысты
екендігі анықталды. Жүргізілген зерттеулерге сәйкес, ақуыздың мөлшері
төлдегеннен кейін бірден алынған уыз сүтінде төлдегеннен кейін 24 сағат
36 сағаттан кейін жиналған уыз сүтіне қарағанда айтарлықтай жоғары
екендігі анықталды. Мақалада мұздатып кептіру әдісімен алынған құрғақ
сиыр уызын зерттеу туралы мәліметтер келтірілген. Осыған байланысты
мақала авторлары құрғақ уыздың химиялық құрамын зерттеді, атап айтқанда
уыз сүтін жинау уақытына байланысты ақуыз, май және күл концентрациясы
анықталды. Авторлар төлдегеннен кейін бірден алынған уыздың ылғалдылығы
төмен екенін анықтады, бұл сақтау мерзімін ұзартуға мүмкіндік береді.
Мақалада оны кейіннен тағамдық және биологиялық құндылығы жоғары және
иммуномодуляциялық әсері бар тамақ өнімдерін өндіруде қолдану жолдары
негізделген.

{\bfseries Түйін сөздер:} сиыр уызы, құрғақ уыз, биологиялық белсенді
заттар, ақуыздар.

\begin{multicols}{2}
{\bfseries Introduction.} The nutraceuticals and functional foods market
shows significant progress aimed to meet consumer demand. Nowadays,
people are looking for new and safer food ingredients that become not
only a source of nutrients, but also benefit health and ensure
well-being. This concept draws consumers\textquotesingle{} attention to
dietary biologically active compounds, nutraceuticals and functional
foods.

One of the underestimated biologically active product with great
potential is cow colostrum. Nowadays it is of great interest of
scientists from all over the world {[}1{]}. Cow colostrum is a feed or
abnormal milk, that rich in immunological agents, play the role of
passive immunity in a newborn calf, guarantee of protection and help in
the development of the gastrointestinal system {[}2{]}. Its composition
rich in dry substances, proteins, immunoglobulins, fats and growth
factors, which arouses interest in their use in the development of both
pharmaceuticals and food derivatives {[}3{]}.

A.G. Khramtsov et al. studied protective substances (antibacterial
factors) of cow colostrum. The authors believe that colostrum is useful
not only for newborn calves, but also for children, athletes, the
elderly, patients with tuberculosis, stomach ulcers and diabetes
{[}4{]}. It is known that colostrum contains in its composition a
significant amount of lactoferrin, lactoperoxidase and lysozyme, which
have antimicrobial and antiviral properties. Lactoperoxidase affects the
binding of liposaccharides by regulating bacterial growth, while
lactoferrin has toxic properties for several gram-positive and negative
bacteria, as well as antiviral properties. Meanwhile lysozyme affect to
the health by attacking the peptidoglycan component of gram-positive
bacteria, causing bacterial lysis {[}5{]}. By gaining such positive
effects colostrum allow to destruct certain pathogenic microorganisms
such as E. coli, rotavirus and cryptosporidium {[}6{]}.

The use of colostrum remains limited not only due to insufficient
research of this biologically active substance, but also due to its
short shelf life. The technology of deep processing and obtaining acid
preservation, dry concentrate on its basis, or isolation and
purification of individual fractions without significant changes in
composition and quality are expensive and time-consuming {[}7{]}. Due to
the development of mini-refrigerators, the preservation and
transportation market of colostrum, the processing point is not
technically difficult anymore and ensures the preservation of all
biologically valuable substances {[}8{]}.

Thus, the above discussion approved that the direction of scientific
research as the development of food products based on milk colostrum
becomes relevant.

The aim of the work was to study the physico-chemical parameters of
colostrum of Simmental cows and compare the composition of natural
colostrum and colostrum powder obtained by freeze drying.

{\bfseries Materials and methods.} This part should consist of a
description of the materials, the progress of the work, as well as a
complete description of the used methods. The characterization or
description of the research material includes the presentation of the
specified material in qualitative and quantitative terms. The
characteristic of the material is one of the factors determining the
reliability of the conclusions and research methods.

Objects and methods of research

- colostrum of Simmental cows obtained after calving, after 24 hours and
36 hours

- dry colostrum obtained by freeze-drying in the laboratory.

The research of the physico-chemical composition was carried out in the
research laboratory of the Food Technology and Processing Products
department of the S. Seifullin Kazakh Agrotechnical Research University.

The following methods were used in the process of conducting the study:

- titrated acidity -- according to State Standart 3624-92 "Milk and
dairy products. Methods for determining titrated acidity"; {[}9{]}.

-protein content in natural and dry colostrum -- according to State
Standart 25179-2014 "Milk and dairy products. Methods for determining
the mass fraction of protein";{[}10{]}.

- fat content -- according to State Standart 5867-90 "Milk and dairy
products. Methods for determining fat";{[}11{]}.

- moisture content on moisture meters RADWAG MA -60.3;

- lactose and solids content -- on the TANGO BRUKER spectrophotometer
with a measuring range of 11500-4000 cm\textsuperscript{-1}.

{\bfseries Results and discussion.} In order to determine the
physico-chemical composition of cow colostrum, raw material were taken
from the most common Simmental cow breed in the Akmola region. For the
purity of the experiment, cows kept under equal conditions and received
the same amount of feed while milking.

It has been established that cow colostrum is a breast milk of the
mammary gland, produced after calving for enhanced feeding of the calf
in the first days of life and providing it with a large number of
antibodies for the formation of natural immunity. It is produced in the
postpartum period during the next 7-10 days after calving. Young animals
that received the first portion of colostrum within 1.5 hours after
birth, up to 2 weeks, show relatively better growth rates and are immune
to general dyspepsia and bronchopneumonia {[}12{]}.

The first batch of colostrum was received within 3 hours after calving,
then the second batch after 24 hours and the third batch after 36 hours.
Visually, natural colostrum is a yellow-brown liquid on the first day of
milking, which by the third day approaches the color of milk. The
physico-chemical composition of natural cow colostrum is presented in
Table 1.
\end{multicols}

\begin{table}[H]
\caption*{Table 1- Physico-chemical composition of natural cow colostrum}
\centering
\resizebox{\textwidth}{!}{%
\begin{tabular}{|l|l|lll|}
\hline
№ & Parameters & \multicolumn{1}{l|}{Colostrum after calving} & \multicolumn{1}{l|}{Colostrum after 24 hours} & Colostrum after 36 hours \\ \hline
1 & Titrable acidity, 0 Т & \multicolumn{1}{l|}{58} & \multicolumn{1}{l|}{40} & 32 \\ \hline
2 & рН & \multicolumn{1}{l|}{4,92} & \multicolumn{1}{l|}{6,41} & 6,49 \\ \hline
3 & Dry matters, \% & \multicolumn{1}{l|}{19,17} & \multicolumn{1}{l|}{10,8} & 10,04 \\ \hline
4 & Protein content, \% & \multicolumn{1}{l|}{13,34} & \multicolumn{1}{l|}{7,54} & 3,95 \\ \hline
5 & Lactoze, \% & \multicolumn{1}{l|}{5,54} & \multicolumn{1}{l|}{4,76} & 4,51 \\ \hline
6 & Fat content, \% & \multicolumn{1}{l|}{5,08} & \multicolumn{1}{l|}{4,48} & 2,95 \\ \hline
7 & Moisture, \% & \multicolumn{1}{l|}{63,976} & \multicolumn{1}{l|}{78,588} & 79,731 \\ \hline
8 & Purity (group) & \multicolumn{3}{l|}{No less than II} \\ \hline
9 & Temperature, 0 С & \multicolumn{3}{l|}{4*/-2} \\ \hline
\end{tabular}%
}
\end{table}

\begin{multicols}{2}
According to the data given in Table 1, it was found that the results
comply with State Standart R -- 71167-2023 ``Cow colostrum (raw
materials). Technical conditions''{[}13{]}. The data shows that the
chemical composition of colostrum depends on the time of the cow's
calving. According to the data, after calving, the percentage of
protein, fat and lactose is 13.34:5.08:5.54\% compared with colostrum
obtained on the second day of milking, which is 10.8:4.48:4.76\%.
According to the analysis data and based on the organoleptic properties,
it was found that colostrum obtained after 36 hours after calving is
close to the indicators of cow\textquotesingle s milk and thus its study
has no scientific interest.

Considering that colostrum can be stored at a temperature from 0
\textsuperscript{0}C to 4 \textsuperscript{0}C for up to 8 days, subject
to sanitary requirements, without significantly changing the nutritional
value of colostrum. Colostrum was freeze-dried to preserve the
biological and nutritional properties. Drying was carried out in
laboratory conditions in Vacuum freeze-drier LG -06 at a temperature of
- 50 \textsuperscript{0}C and a pressure of 10 MPa.

In the resulting dry colostrum physico-chemical composition were
assesed. The data are shown in Table 2.
\end{multicols}

\begin{table}[H]
\caption*{Table 2 - Physico-chemical composition of dry cow colostrum}
\centering
\begin{tabular}{|l|l|l|l|l|}
\hline
№ & Parameters & Colostrum after calving & Colostrum after 24 hours & Colostrum after 36 hours \\ \hline
1 & Titrable acidity, 0 Т & 57 & 42 & 35 \\ \hline
2 & Protein content, \% & 37,3 & 32,2 & 29,2 \\ \hline
3 & Lactoze, \% & 5,5 & 4,7 & 4,1 \\ \hline
4 & Ashes, \% & 6,9 & 5,4 & 5,2 \\ \hline
5 & Fat content, \% & 39,7 & 32,2 & 28,3 \\ \hline
6 & Moisture, \% & 3,51 & 3,92 & 3,91 \\ \hline
\end{tabular}
\end{table}

\begin{multicols}{2}
The resulting dry colostrum has real prospects for use as a functional
ingredient in food products.

{\bfseries Conclusions.} The results of studies of the chemical composition
of colostrum prove the prospects of its use in the food industry as a
source of protein of animal origin. However, it should be borne in mind
that the maximum amount of biologically active substances is stored in
colostrum in the first hours after calving.

This work was carried out within the framework of the IRN BR21882184
2PCF-MNVO/24 program "Creation of a set of risk management measures to
ensure food safety and the development of meat and dairy products with
increased biological value"
\end{multicols}

\begin{center}
{\bfseries References}
\end{center}

\begin{noparindent}
1.Bodammer, P., Maletzki, C., Waitz, G., \& Emmrich, J. Prophylactic
  application of bovine colostrum ameliorates murine colitis via
  induction of immunoregulatory cell // Journal of Nutrition. -2011.
  --Vol. 141(6). --P. 1056-1061. DOI 10.3945/jn.110.128702

2.Nikolic, I., Stojanovic, I., Vujicic, M., Fagone, P., Mangano, K.,
  Stosic-Grujicic, S., Nicoletti, F., \& Saksida, T. Standardized bovine
  colostrum derivative impedes development of type 1 diabetes in rodents
  // Immunobiology. -2017. --Vol. 222(2). --P. 272-279. DOI
  10.1016/j.imbio.2016.09.013

3.McSweeney, P. L. H., \& McNamara, J. P. Encyclopedia of Dairy
  Sciences: Third edition. In Encyclopedia of Dairy Sciences: Third
  edition. -2021.

4.Gorbatova K.K. Biokhimiya moloka i molochnykh produktov. - SPb.:
  GIORD, 2001. - 320 s. {[}in Russian{]}

5.Bagwe, S., Tharappel, L. J. P., Kaur, G., \& Buttar, H. S. Bovine
  colostrum: An emerging nutraceutical // In Journal of Complementary
  and Integrative Medicine. -2015. -Vol. 12. --Iss. 3. DOI
  10.1515/jcim-2014-0039

6.Mehra, R., Singh, R., Nayan, V., Buttar, H. S., Kumar, N., Kumar, S.,
  Bhardwaj, A., Kaushik, R., \& Kumar, H. Nutritional attributes of
  bovine colostrum components in human health and disease: A
  comprehensive review // In Food Bioscience. -Vol. 40. DOI
  10.1016/j.fbio.2021.100907

7.Pat. 2275564 Rossiiskaya Federatsiya. MPK F 26 V 9/06 Transfer Faktor
  XF™ Patent SShA №4 816 563. Sposob polucheniya sublimirovannykh
  pishchevykh produktov. -№ 77.99.23.3.U.7085.12.04 ot 10.12.04. {[}in
  Russian{]}

8.Pat.2535877 S1 Rossiiskaya Federatsiya. MPK A23/S 9/123. Cposob
  proizvodstva iogurta s funktsional\textquotesingle nymi svoistvami /
  Polyanskaya Irina Sergeevna, Topal Ol\textquotesingle ga Ivanovna,
  Novokshanova Alla L\textquotesingle vovna, Teraevich Alla Sergeevna. -
  № 2013134767/10, 2013.07.23; zayavl. 23.07.2013; opubl. 20.12.2014
  {[}in Russian{]}

9.Leont\textquotesingle eva S.A., Tikhonov S.L., Tikhonova N.V., Lazarev
  V.A. Korov\textquotesingle e molozivo - perspektivnoe
  syr\textquotesingle e dlya proizvodstva produktov pitaniya //
  Pishchevaya promyshlennost\textquotesingle. -2021. --T. 6. -№2. - S.
  23-33. {[}in Russian{]}

10.GOST 3624-92. Moloko i molochnye produkty. Titrim etricheskie m etody
opredeleniya kislotnosti. --M: Standartinform, 2009. -6s. {[}in
Russian{]}

11. GOST 25179-2014 Moloko i molochnye produkty. Metody opredeleniya
massovoi doli belka. --M:

Standartinform, 2015. -8s. {[}in Russian{]}

12. GOST 5867-90 Moloko i molochnye produkty. Metody opredeleniya zhira.
--M: Standartinform, 2009. -12s. {[}in Russian{]}

13. GOST R-71167-2023. Molozivo korov\textquotesingle\textquotesingle e
(syr\textquotesingle\textquotesingle e) Tekhnicheskie usloviya. --M:
Ross. Inst.standartizatsii. -14 s. {[}in Russian{]}
\end{noparindent}

\emph{{\bfseries Information about the authors}}

\begin{noparindent}
Tultabayeva T.Ch. -- d.t.n., associate professor, Kazakh Agrotechnical
Research University named after

S.Seifullin, Astana, Kazakhstan, e-mail:
tamara\_tch@list.ru;

Zhakupova G.N. -- candidate of technical sciences,Kazakh Agrotechnical
Research University named after

S.Seifullin, Astana, Kazakhstan, e-mail:
gulmira-zhak@mail.ru;

Kokumbekova N.K.- master of technical sciences, Kazakh Agrotechnical
Research University named after

S.Seifullin, Astana, Kazakhstan, e-mail:
kokumbekovanazym@gmail.ru;

Sagandyk A.T. -- master of technical sciences, S.Seifullin Kazakh
Agrotechnical Research University,

Astana, Kazakhstan, e-mail:
assema.bukeyeva@gmail.com;

Muldasheva A.H. -- master of technical sciences Kazakh Agrotechnical
Sesearch University named after

S.Seifullin, Astana, Kazakhstan, e-mail:
Ak91.91@mail.ru;

Akhmetzhanova A.T. -- doctoral student ,Kazakh Agrotechnical Research
University named after

S.Seifullin, Astana, Kazakhstan, e-mail:
aygerim\_talgatqyzy@mail.ru
\end{noparindent}

\emph{{\bfseries Сведения об авторах}}

\begin{noparindent}
Тултабаева T.Ч.- д.т.н, ассоц. профессор, Казахский агротехнический
исследовательский

университет им.С.Сейфуллина, Астана, Казахстан,
e-mail: tamara\_tch@list.ru;

Жакупова Г.Н. -- к.т.н., ассоц. профессор, Казахский агротехнический
исследовательский

университет им.С.Сейфуллина, Астана, Казахстан,
e-mail: gulmira-zhak@mail.ru;

Кокумбекова Н.К.- магистр технических наук, Казахский агротехнический
исследовательский

университет им.С.Сейфуллина, Астана, Казахстан,
e-mail: kokumbekovanazym@gmail.ru;

Сагандык А.Т. -- магистр технических нау Казахский агротехнический
исследовательский

университет им.С.Сейфуллина, младший» Астана,
Казахстан, e-mail: assema.bukeyeva@gmail.com;

Мулдашева А.Х. -- магистр технических наук, Казахский агротехнический
исследовательский

университет им.С.Сейфуллина, Астана, Казахстан,
e-mail: Ak91.91@mail.ru;

Ахметжанова А.Т. -- докторант кафедры Казахский агротехнический
исследовательский

университет им.С.Сейфуллина», Астана, Казахстан,
e-mail: aygerim\_talgatqyzy@mail.ru
\end{noparindent}
