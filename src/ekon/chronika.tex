\newpage
\let\cleardoublepage\clearpage
\part{Хроника}
\emph{{\bfseries ХРОНИКА}}

\sectionwithauthors{А.Е. Даниярова}{ВКЛАД АКАДЕМИКА Е.А. БУКЕТОВА В РАЗВИТИЕ ХИМИЧЕСКОЙ НАУКИ КАЗАХСТАНА}

\begin{center}
{\bfseries А.Е. Даниярова}

Карагандинский технический университет им. Абылкаса Сагинова,

г. Караганда, Казахстан,

e-mail: aina171173@mail.ru
\end{center}

\begin{multicols}{2}
Биографии ученых являются одним из научных направлений в исторических
исследованиях. Биографический жанр, уместный при освещении любых сторон
исторического процесса, особенно плодотворен тогда, когда речь идет о
духовной жизни, в которой на первый план выдвигается индивидуальное
творческое начало. В настоящей статье проведена попытка проанализировать
вклад академика Академии наук Казахской СССР Е.А. Букетова в развитие
химической науки Казахстана, рассматривается его роль как организатора
науки в создании Химико-металлургического института в г. Караганде в
системе АН КазССР, показана деятельность ученого в определении научных
направлений института, формировании научных школ, укреплении
материально-технической базы крупнейшего научно-исследовательского
института Центрально-Казахстанского региона. Основное фундаментальное
направление научной деятельности академика Е.А. Букетова -- химия и
технология халькогенов, халькогенидов, молибдена, ванадия, мышьяка,
глинозема, меди, включало также ряд практических разделов: шахтный обжиг
сырья цветных и редких металлов, кинетика и термодинамика
окислительно-восстановительных процессов в водных средах, химическое
подобие и электронное строение элементов, водородная энергетика на
ферросплавной основе. Главные научные достижения Е.А. Букетова по этим
разделам: разработка теоретических основ шахтного обжига гранулированных
материалов, создание технологических принципов автоматизации теплового
режима шахтного обжига, внедрение окислительного спекания.

XXI век -- это, несомненно, время ускоренного научного прогресса в
котором большое значение для анализа опыта прошлых достижений имеют
биографии людей науки. Наметилась тенденция к осмыслению и определению
роли конкретных ученых в становлении и развитии науки, оценке значения
их творчества, что актуально и необходимо для подготовки современных
исследователей. Научная деятельность того или иного ученого объективно
является фактором культурного развития общества. В этой связи,
специального исследования требует рассмотрение взглядов и разносторонней
деятельности академика Е.А. Букетова, чье творчество дает новый материал
для анализа процесса формирования казахстанской интеллигенции середины и
второй половины ХХ века.

Цель работы - показать вклад академика Е.А. Букетова в развитие
химической науки Казахстана. Предметом исследования является научная и
организаторская деятельность Е.А. Букетова в создании и развитии
Химико-металлургического института АН КазССР. Объектом исследования
является персоналия академика Е.А. Букетова.

Настоящая работа основана на архивных и частично опубликованных
источниках о жизни и творчестве ученого, которые раскрывают новые
стороны многогранного творчества Е.А. Букетова.

Материалы по вопросам научного наследия ученого хранятся в нескольких
архивных фондах: Государственном архиве Карагандинской области (ГАКО).
-- Ф.1484; Архиве Казахского национального исследовательского
технического университета им. К.И. Сатпаева (Каз НИТУ). -- Ф.122; Архиве
Карагандинского университета им. Е.А. Букетова (КарУ). -- Ф.764, Текущем
архиве Химико-металлургического института за 1960 - 1983 гг.; в фондах
Мемориального музея академика Е.А. Букетова и истории университета (фонд
академика Е.А. Букетова, научно-вспомогательный фонд) {[}1{]}.

Многочисленную группу архивных материалов составляют повествовательные
источники, статьи и рукописи Е.А. Букетова с правками самого ученого.
Изучение разного рода подготовительных материалов -- черновиков,
конспектов, выписок и т.д. позволяет проникнуть в творческую лабораторию
ученого. Это необходимо для того, чтобы лучше понять возникновение
творческой мысли. Анализ подготовительных материалов и черновиков
статей, позволил обратить внимание на редакторскую правку и характер
корректировок, сделанные автором, так как подобные исправления резко
увеличивают информативность неопубликованных источников.

Значительную по объему группу материалов составляют эпистолярные
источники. Большинство дошедших до нас писем адресовано Е.А. Букетовым
научным сообществам, а также редакциям газет и журналов. Деловые письма
характеризуют его активную научную и публицистическую деятельность.
Обильную информацию для изучения методики естественнонаучных и
гуманитарных исследований дают дружеские письма Е.А. Букетова к
известным личностям: ученым, писателям и общественным деятелям. Среди
его корреспондентов известные личности: академик В.И. Спицын, О.О.
Сулейменов, Т.А. Сатпаева, М.К. Сатпаева и другие. Эти письма позволяют
выявить истоки зарождения научных идей, литературных замыслов, заглянуть
в творческую мастерскую ученого. Изучение самого круга переписки дает
ценные сведения для характеристики взглядов ученого. Среди эпистолярных
источников значительное место занимают письма, адресованные Е.А.
Букетову, его личная переписка является только дополнением к основному
материалу источников.

Особую ценность представляют мемуарные издания. Эти источники
субъективны по характеру, но позволяют ощутить «колорит» эпохи и
отразить наиболее специфические черты общества. Материалы такого плана
очень важны и необходимы, так как нынешние и будущие исследователи будут
изучать исторический процесс только с позиции ретроспективного анализа.

Важным источником информации о творчестве ученого является его
автобиографическая повесть «Шесть писем другу», позволяющая глубже
раскрыть и прокомментировать творчество Е.А. Букетова {[}2{]}.

Особую значимость представляют воспоминания родного брата ученого - К.А.
Букетова {[}3{]}. Характер и объем книги не дают исчерпывающего
освещения всех сторон деятельности ученого, но все же позволяют получить
цельное представление о личности Е.А. Букетова. Важную часть работы
составляют архивные документы. Весь материал исследования сгруппирован в
хронологическом порядке, применительно к отдельным этапам жизненного и
творческого пути ученого.

Ценную информацию об организаторской и научной деятельности академика
Е.А. Букетова в создании Химико-металлургического института в г.
Караганде содержит работа мемуарного характера доктора технических наук,
профессора В.П. Малышева, ученика-последователя Е.А. Букетова, {[}4{]}.
Она представляет интерес не только как новое свидетельство современника,
к тому же прекрасно ориентирующегося во всех научных изысканиях
наставника, но и как наиболее полное, концентрированное и
специализированное описание личности Е.А. Букетова.

Методологической базой данной статьи послужил комплекс общенаучных и
специально-исторических методов, позволяющий не только изучать
актуальные проблемы истории, но и исследовать деятельность выдающихся
исторических личностей, среди которых, прежде всего, следует выделить
историко-системный и проблемно-хронологический методы. Наряду с ними, в
работе был применен ряд методик: источниковедческого поиска и
исследования, текстологического и концептуального анализа произведений,
на основании которых проводится настоящая работа. В процессе работы
автор руководствовался принципами объективности и историзма. В качестве
общей композиционной основы был использован ретроспективный метод,
направленный на поэтапном переходе исследования от настоящего к
прошлому, от следствия к причине.

Анализ эпистолярного наследия академика Е.А. Букетова, оставленный своим
научным наставникам, малоизвестных его публикаций и других архивных
материалов подтверждает факт, что годы аспирантуры, как и студенческие
годы, были наиболее важными этапами для формирования его мышления,
научных интересов и планов. Одним из доказательств тому является письмо,
адресованное академику В.Д. Пономареву, в котором имеется информация о
начальном периоде становления Е.А. Букетова как научного работника:
«Когда чувствуешь, что одолел собственную малограмотность,
неорганизованность, легкомыслие, это не бог весть какое достижение
становится сладким и значительным. С этой смешной победы над собой во
мне начался научный работник, ибо именно с этого момента я приобрел
боязнь относиться несерьёзно к чему бы то ни было, если это касалось
научной работы» {[}5{]}. Выполняя свои первые исследования, активно
участвуя в заседаниях научного сообщества химиков, металлургов, в
проводимых научных семинарах, выполняя опыты по диссертационной работе,
которые не сразу давали ожидаемых результатов -- в ежедневном кипении
рабочих буден, Е.А. Букетов рос и созревал как учёный. Данные
экспериментов, полученные опытным путем, позволили проанализировать и
обобщить их результаты и опубликовать в научном журнале института. «Ни
одному из дальнейших успехов в научной работе я не радовался так бурно,
так непосредственно, как этому первому моему ученическому успеху, тем
более что удача посетила человека, ещё не охлажденного жизнью и
превратностями земного» {[}6{]}.

Именно первая научная работа вдохнула в него уверенность,
засвидетельствовав, что и у него при обдумывании прочитанного,
услышанного, увиденного тоже могут появиться дельные мысли. «Пережив
длительный ряд раздумий, разочарований, а затем, испытав первые
положительные исходы своего дела, им овладело одно -- всё это нужно,
потому что это будущее, это путь, через который можно утвердить
полезность и необходимость своего бытия» {[}7{]}. Не сразу пришли верные
идеи. Вспышки вдохновения озаряли долгий и кропотливый будничный труд.
Он надеялся только на целеустремлённое и постоянное беспокойство, мечтал
о днях, когда его упрямство будет вознаграждено. «Не будь этих
нескромных, честолюбивых мечтаний молодости, которые принято скрывать, у
меня не было бы и того малого, чем не смотря ни на что горжусь»,
искренне признавался Е.А. Букетов, уже будучи крупным учёным {[}8{]}.

Значительную роль в формировании личности ученого играет
непредубежденность, самокритичность, готовность признать свои ошибки,
отбросив прежние идеи, когда они приходят в противоречие с проверенными
фактами и теориями. Каждая эпоха трансформирует личность ученого
по-своему. Наряду с творческими способностями, настойчивостью,
трудолюбием существенное значение приобретает коммуникабельность,
готовность к сотрудничеству, способность сочетать личные интересы с
задачами научного учреждения, понимания социального назначения науки.
Ученый - субъект не только науки, но и своего времени.

6 июня 1954 г. Е.А. Букетов, молодой ученый, исследующий проблему
извлечения молибдена и его химического анализа, защищает диссертацию на
соискание учёной степени кандидата технических наук на объединенном
заседании Ученого совета Института металлургии и обогащения Академии
Наук КазССР. С этого момента начинается новый этап его творческого пути,
он посвящает себя преподавательской работе в Казахском
горно-металлургическом институте. В связи с окончанием аспирантуры, на
основании приказа №40 от 1 февраля 1954 г. Е.А. Букетова принимают на
должность ассистента кафедры «Металлургия легких и редких металлов», а с
сентября 1956 г. переводят на должность доцента этой же кафедры {[}9{]}.

В процессе преподавательской деятельности, для соответствия уровню и
требованиям высококвалифицированного специалиста первостепенной задачей
молодого ученого стало постоянное самоусовершенствование.
«Преподавательская работа была хороша тем, --- вспоминал Е. Букетов, что
приводила меня к убеждению, как много нужно работать, чтобы быть
достаточно знающим наставником. Студенческая пытливость заставляла меня
с лихорадочной поспешностью осваивать книги, чтобы научная
осведомлённость учителя стала добротным достоянием учащихся» {[}2, с.
187{]}.

Исключительная энергичность, деловитость, настойчивость Е.А. Букетова
обратили на себя внимание руководства института, приказом №149 от 14
июля 1958 г. он был освобожден от работы доцента кафедры и назначен
заместителем директора Горно-металлургического института по учебной
работе {[}10{]}.

В качестве одного из руководителей вуза он начинает посещать занятия
преподавателей; акцентирует свои требования на том, чтобы квалификация
профессорско-преподавательского состава учебного заведения
соответствовала должному уровню; занимается вопросами внедрения новых
технологий обучения в образовательный процесс, в частности, оснащения
лабораторий вуза современными приборами и инструментами.

В этот период, Е.А. Букетов, как сложившийся ученый и педагог, имея
перспективы для дальнейшего творческого роста, хорошо обустроенный быт,
решает начать новую жизнь. Этот поворот произошел в феврале 1960 г.,
после встречи с первым президентом АН Казахской ССР, академиком Канышем
Имантаевичем Сатпаевым, который предлагает Е.А. Букетову от имени
Академии работать в её системе, возглавив открывшийся
Химико-металлургический институт в городе Караганде. При этом президент
Академии наук делает упор на необходимость личного научного роста, без
которого невозможен: «Неподдельный авторитет руководителя научного
учреждения» {[}11{]}.

Свою первую встречу с выдающимся учёным Е.А. Букетов воспроизводит
следующим образом: «Это было состояние внезапного соприкосновения с
чем-то недосягаемым, сказочно высоким, когда ты вдруг чувствуешь, что ты
оказался каким-то образом достоин его, и за этим следует та высокая
ответственность, которая потребует от тебя научного, глубинного
понимания значения твоих будущих замыслов и действий» {[}10, с. 204{]}.
Е.А. Букетов принимает предложение президента АН КазССР и обещает
оправдать оказанное ему доверие.

В соответствии с Постановлением бюро Президиума АН КазССР за №8 от 13
февраля 1960 г. Е.А. Букетова назначают директором
Химико-металлургического института и руководителем лаборатории
металлургии цветных, легких и редких металлов {[}12{]}.

Много сил, энергии, знаний прикладывает новый директор развитию
института. выбрав путь со многими неизвестными; необходимость
определения научного направления, создания научной школы, формирование
сплоченного коллектива, строительство производственной базы и жилья.

Размышляя о том времени, -- периоде становления института, Евней
Арстанович размышлял: «Думаю, не от хорошей жизни я был назначен на эту
должность. Очевидно, моя кандидатура всплыла перед президентом Академии
наук после того, как он убедился, что никого из опытных и достаточно
маститых товарищей не прельщает руководство институтом, коллектив
которого состоял из какой-то сотни лиц, съехавшихся из разных концов
республики. Материальная база была представлена небольшим двухэтажным
зданием бывшего общежития, соседнего учебного заведения, комнаты
которого с редко расставленными канцелярскими столами и шкафами не
напоминали химические лаборатории. Институт, несмотря на солидное
название, таковым еще не являлся. Президент напутствовал перед выездом
сюда на работу обратить внимание на многие трудности, ожидавшие меня
впереди» {[}2, с.211{]}.

Постепенно в коллективе устанавливаются непринуждённые взаимоотношения с
младшими коллегами, что придает упорства в преодолении возникающих
трудностей на пути становления научного учреждения. Поэтапно решаются
проблемы материальной базы института. Соратники, ученики Е.А. Букетова с
теплотой вспоминают, как они с большим энтузиазмом подходили к общему
делу: осваивали, чуть ли не всем институтом, домик под стеклодувную
мастерскую; из котельной, подлежащей сносу, получили прекрасную
лабораторию, с неожиданно высокими потолками и поэтому не требовалось
особых забот по вентиляции. И, наконец, усилия коллектива увенчались
успехом, когда началось строительство современного здания института.

Был у Е.А. Букетова и момент эмоционального колебания, когда он в
отчаянии написал заявление об уходе на имя президента АН КазССР К.И.
Сатпаева. «Я отталкивал от себя трезвые соображения о преодолении
трудностей, потому что был в обиде на себя же самого, что совершенно не
способен на достижение целей, требующих длительной выдержки, упорства,
настойчивости, методичности. Я обнаруживал свою беспомощность. Мне не
оставалось ничто иного, как подать заявление на имя президента об
освобождении от занимаемой должности» {[}13{]}. К.И. Сатпаев не довел
данную информацию до общего обсуждения на бюро президиума АН КазССР, но
вызвал Е.А. Букетова на личную беседу и в завершении разговора заключил:
«Невесёлые у вас дела, но мне импонирует, что вы свои ошибки ни на кого
не сваливаете, вы суровы к себе. Мужество заключается не только в том,
чтобы признать ошибки, это полдела, за признанием следует дело,
требующее упорства, напряжения сил» {[}14{]}. Вспоминая эту встречу,
Евней Арстанович отмечал: «За эти короткие часы я испытал, как слова
могут полосовать кости, тогда как плётка имеет дело лишь с кожей и
мясом. Я чувствовал, что мне нет возврата, и не остается ничего другого,
как напрячь все свои силы, и все своё уменье, чтобы не обманывать
надежды этого человека и всех других хороших людей, делающих на меня
хоть какую-то ставку. Опасно проявление прыти с первого раза» {[}15{]}.

Потребовалось время, опыт и знания, чтобы создать прочный фундамент для
целостного функционирования академического учреждения. Прошли годы, и
институт вышел далеко за пределы Центрального Казахстана. Характерной
чертой в деятельности Е.А. Букетова являлось умелое сочетание науки и
практики. Под руководством Е.А. Букетова Химико-металлургический
институт АН КазССР укреплял творческие связи с производственными
предприятиями, проектными организациями, научными учреждениями, не
только по республике, но и в масштабах всего Советского Союза.
Постоянные контакты были с комбинатрм «Карагандауголь», Балхашским
горно-металлургическим комбинатом, другими предприятиями горнорудной
промышленности и химическими заводами.

Несколько лет продолжались сложные эксперименты, поиски новых путей
извлечения металла -- рения, в котором так нуждалась современная
техника. Огромную работу пришлось проделать, чтобы решить эту задачу.
Надо было связать в единый комплекс медеплавильные печи и непрерывное
производство сернокислотного цеха, чтобы дым стал значительно светлее и
«чище». Творческие искания и напряжённый труд многих людей:
горняков-обогатителей, металлургов и химиков увенчался успехом. Страна
отметила этот научный подвиг -- внедрение технологии комплексной
переработки медных руд на Балхашском горно-металлургическом комбинате
присуждением в 1969 году Государственной премии СССР группе специалистов
и ученых, во главе с профессором Е.А. Букетовым {[}16{]}.

Всю свою энергию, колоссальную работоспособность Е.А. Букетов направил
на установление свойств таких важных и нужных для промышленности страны
металлов как селен, теллур, рений и галлий. В то время свойства и
поведение в различных системах рассеянных металлов, содержащихся в
рудах, концентратах, полупродуктах химического и металлургического
производства в незначительных количествах были мало исследованы и
изучены. Долгое время не было практического применения данных
металлоидов, но по мере увеличения количества проведения
экспериментальных опытов в этой области были получены результаты,
носящие прикладной характер. Высокочистый селен, главным образом,
применялся для изготовления выпрямителей преобразователей тока а также,
для окраски изделий из стекла, в производстве красителей и химических
соединений. Теллур использовался при создании особой лампы, которая дает
непрерывный спектр и нашёл широкое применение в радарных установках, в
автомобильной промышленности. Проведенные в то время работы по селену и
теллуру, на сегодняшний день, являются ценным справочным материалом для
специалистов, работающих с продуктами, содержащими эти металлы. Вопросы
изучения металлоидов (главным образом -- селена и теллура), возможность
получения их в промышленных масштабах, легли в основу научных изысканий
Е.А. Букетова.

Результаты наблюдений, экспериментов и опытов были обобщены в докторскую
диссертацию на тему: «Извлечение селена и теллура из остатков медных
электролитов». 10 октября 1966 года научное исследование было
представлено к защите на Ученом совете Московского ордена Трудового
Красного знамени института стали и сплавов. Работа была высоко оценена и
получила многочисленны положительные отзывы крупных специалистов.
Официальными оппонентами диссертанта являлись видные ученые: доктор
технических наук, профессор Н.Н. Севрюков, доктор технических наук,
профессор Н.А. Суворовская, профессор М.Д. Ивановский, отзыв передового
предприятия был представлен Кыштымским медеэлектролитным заводом. В
выписке из протокола № 3 от 10 октября 1966 года заседания Объединенного
совета при Московском ордена Трудового Красного знамени института стали
и сплавов по присуждению ученых степеней значилось: «Заслушав сообщение
диссертанта, отзывы официальных оппонентов, передового предприятия,
мнения членов Совета и присутствующих и признав опубликованный материал
по диссертации достаточно полным Объединенный совет единогласно при
тайном голосовании постановил считать достойным присуждение ученой
степени доктора технических наук Букетову Е.А. Результаты голосования:
за присуждение -- 12, против --- нет» {[}17{]}.

Решением Высшей Аттестационной комиссии от 25 февраля 1967 года протокол
№9, Евнею Арстановичу Букетову была присуждена ученая степень доктора
технических наук {[}18{]}. В июле 1967 года Президиум АН КазССР возбудил
ходатайство перед Высшей Аттестационной комиссией при Министерстве
высшего и среднего специального образования СССР о присвоении директору
ХМИ АН КазССР, доктору технических наук, Е.А. Букетову ученого звания
профессора. По решению Высшей Аттестационной комиссии от 11 октября 1967
года протокол №555 Е.А. Букетов был утвержден в учёном звании профессора
по специальности: «Металлургия цветных, благородных и редких металлов»
{[}19{]}.

Научное направление в области химии и технологии халькогенов и
халькогенидов, в первую очередь, селена и теллура явилось одним из
главных и результативных направлений в научной деятельности, основанным
академиком Е.А. Букетовым. Академия наук СССР, отмечая вклад академика
Е.А. Букетова, утвердила проведение союзных совещаний по данному
научному направлению в г. Караганде, председателем которых в 1978 и 1982
гг. был назначен Е.А. Букетов.

Коллеги и научные последователи Е.А. Букетова утверждают, что их
руководителю была присуща научная интуиция. Проблемы, разрабатываемые в
ХМИ АН Казахской ССР с 60-х гг. XX века были актуальными в мировом
масштабе, в частности, аналогичные работы проводились в США, Канаде и
Англии (данную информацию сотрудникам удалось почерпнуть позже, из
научной литературы). Доктор технических наук, профессор В.П. Малышев
отмечал: «Являясь учениками Е.А. Букетова, мы занимались множеством
задач в области химии и металлургии. Мы делили себя на «мышьячников»,
«селенщиков», химиков, металлургов, теоретиков, практиков. Е.А. Букетов
был един во всех этих лицах. Именно он объединил все эти достаточно
самостоятельные направления в одно мощное, дав ему имя «Химия и
технология халькогенов и халькогенидов» {[}4, с. 27{]}.

В научной печати Е.А. Букетов выходил с сообщениями о новых методах,
приемах обработки материалов, которые нельзя было отнести к определенной
области, данные методы сочетали в себе синтез наук, то есть содержали
междисциплинарное начало. Научная общественность страны отмечала, что
результаты трудов «букетовской школы» по химии и технологии селена и
теллура, кроме специализированного значения, были интересны еще и тем,
что в них содержалось несколько принципиально новых научных разработок.
Одной из наиболее плодотворных идей являлось использование окиси цинка в
качестве адсорбента окислов селена, рения, мышьяка и некоторых других
элементов. В ряде случаев было установлено, что окись цинка, как
адсорбент, оказывается незаменимым, а регенерация ее не представляет
затруднений. Выбор этого соединения был сделан на основании
экспериментальных данных, что свидетельствовало об интуиции ученого,
крайне необходимой при разработке технологических вопросов. Другая идея
состояла в применении шахтного аппарата с «сухой» разгрузкой для
проведения операций обжига спекания шлама. Этот принцип был
распространен на процессы пирометаллургической подготовки руд и
концентратов: сушка жезказганских концентратов, обжиг молибденового
полупродукта, термическая обработка катализатора для сжигания выхлопных
газов. Большое место в научной деятельности «букетовской школы» занимало
исследование гидрохимической переработки селен-теллур содержащих шламов
и методов получения селена и теллура. Без знания новейших методов
расчета физико-химических констант немыслимо было научное
прогнозирование, Е.А. Букетов со своими аспирантами, вполне глубоко
овладел этими методами, и их исследования изобиловали новыми сведениями
по термодинамическим свойствам селенидов, теллуридов, теллуратов,
селенатов.

За время работы (1960 - 1972 гг.) Е.А. Букетова руководителем
научно-исследовательского Химико-металлургического института были
организованы целый ряд лабораторий, проведены исследования по актуальным
проблемам, связанные с освоением богатейших ресурсов минерального и
химического сырья Центрального Казахстана.

Одной из первоочередных задач был выбор и обоснование научной
проблематики. В 1961 г. Е.А. Букетов определяет два основных направления
работ по химии и технологии селена и теллура. Первое --
совершенствование применяемых на практике пирометаллургических методов:
спекание шламов с содой и обжига шламов с отгонкой диоксида селена. По
данной теме было получено первое для института авторское свидетельство
СССР и осуществлено (1965 г.) первое внедрение в производство --
упрочняющий обжиг гранулированных концентратов в шахтной печи с
наклонной решеткой. В 1969 г. за эту работу Е.А. Букетов был удостоен
Государственной премии СССР, как руководитель освоения технологии
комплексной переработки медных концентратов Балхашского
Горно-металлургического комбината с применением кислорода на стадии
конвертирования. Второе крупное научное направление -- разработка новых
гидрометаллургических щелочных методов извлечения селена и теллура из
медьэлектролитных шламов.

По предложению Е.А. Букетова в 1960 г. при институте была создана
аспирантура, и функционировали курсы по подготовке и сдаче кандидатских
минимумов по философии, иностранным языкам, предметам по специальности
для научных и учебных учреждений, промышленных предприятий Центрального
Казахстана.

Наставник Е.А. Букетова, его единомышленник, ведущий специалист в
области физической химии академик В.И. Спицин отмечал: «Евней Арстанович
Букетов является творческим научным работником, обладающим широкой
интеллектуальной эрудицией, исследователем, внесшим серьезный вклад в
теорию металлургических процессов. На основании этого считаю
целесообразным и необходимым представление кандидатуры Е.А. Букетова для
баллотирования в действительные члены АН Казахской ССР» {[}20{]}. 3
апреля 1975 г. на очередной сессии Академии наук КазССР кандидатура Е.А.
Букетова была утверждена для избрания действительным членом АН Казахской
ССР {[}21{]}.

Стремление к познанию и творческое беспокойство были отправным началом в
деятельности Е.А. Букетова. Многое раскрывается в этом человеке, когда
обращаешься к источникам его эрудиции и обнаруживаешь, что они не только
в разуме, но и во всем мироощущении ученого. Всю свою жизнь он
придерживался идеи: «Истинным, главным двигателем человека на пути к
совершенствованию является недовольство, неудовлетворённость собой
никогда!» {[}22, с. 61{]}.

Творчество складывается из работоспособности, удачи, знания, фантазии,
абсолютного владения своим мастерством. Этот путь творческих исканий
выбрал и выходец из маленького аула «Алыпкаш» Е.А. Букетов, всецело
посвятив себя науке, литературе и педагогике.
\end{multicols}

\begin{center}
{\bfseries Литература}
\end{center}

\begin{noparindent}
1. Государственный архив Карагандинской области. - Ф. 1484; Архив
Казахского национального исследовательского технического университета
им. К.И. Сатпаева. - Ф.122; Архив Карагандинского университета имени
академика Е.А. Букетова. - Ф.764; Текущий архив Химико-металлургического
института за 1960 - 1983 гг.; Фонды Мемориального музея академика
Е.А.Букетова и истории университета (Научно-вспомогательный фонд, Фонд
академика Е.А.Букетова).

2. Букетов Е. Шесть писем другу. - Алма-Ата: Жалын, 1989. -- 288 с.

3. Букетов К.А. Друг мой, брат мой. -- Караганда: Изд. КарГУ, 1994. -
118 с.

4. Малышев В.П. Поступью командора и пророка. - Караганда: Полиграфия,
1994. - 50с.

5. Государственный архив Карагандинской области (далее ГАКО). -Ф.1484.-
Оп.1. --Д.101.-Л.11

6. ГАКО. -- Ф.1484.- Оп.1 .--Д.74.-Л.21

7. ГАКО. -- Ф.1484.- Оп.1. --Д.83.-Л.9

8. Текущий Архив Химико-металлургического института 1980
г.-Оп.2.-Д.132.-Л.225

9. Архив Казахского национального исследовательского технического
университета им. К.И. Сатпаева (далее
КазНИТУ).-Ф.122.-Оп.2.-Д.2.788.-Л.3.7

10. Архив КазНТУ. - Ф.122.-Оп.2.-Д.2.788.-Л.5

11. ГАКО.-Ф.1484.- Оп.1. --Д.72.-Л.54

12. Текущий Архив ХМИ. 1960 г. - Оп.2. -Д.132. -Л.253

13. ГАКО.-Ф.1484.- Оп.1. --Д.230.-Л.7

14. ГАКО.-Ф.1484.- Оп.1. --Д.230.-Л.9

15. ГАКО.-Ф.1484.- Оп.1. --Д.230.-Л.5

16. ГАКО.-Ф.1484.- Оп.1. --Д.20.-Л.6

17. Текущий Архив ХМИ 1966 г. - Оп.2.-Д.132.-Л.232

18. Текущий Архив ХМИ 1967 г. - Оп.2.-Д.324.-Л.439

19. Текущий Архив ХМИ 1967 г. -Оп.2.-Д.324.-Л.438

20. ГАКО.-Ф.1484.- Оп.1. --Д.20.-Л.3

21. ММБ 163

22. Букетов Е. Грани творчества. - Алма-Ата: Жазушы, 1977.
\end{noparindent}
