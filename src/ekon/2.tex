\newpage
\phantomsection
{\bfseries МРНТИ 82.33.10}
\hfill {\bfseries \href{https://doi.org/10.58805/kazutb.v.2.23-428}{https://doi.org/10.58805/kazutb.v.2.23-428}}

\sectionwithauthors{Х. Батцэнгэл}{ЭКОНОМИКА В «ДУХОВНОМ ИЗМЕРЕНИИ» И НОВАЯ «БИЗНЕС МОДЕЛЬ»}

\begin{center}
{\bfseries Х. Батцэнгэл}

Монгольский университет поствысшего образования, Улан-Батор, Монголия,

e-mail: esbise@yahoo.com
\end{center}

В современных условиях экономика как система существенно изменяется, что
обуславливает ее переход к новой экономике. Экономические исследования
нуждаются в разработке новых теоретических и методологических проблем,
приложении их к развитию бизнес модели и созданию новой, где не только
экономические выгоды, но духовное начало становятся основой деятельности
человека и организации. Духовность заложена в экономике, но более
глубокий ее анализ возможен на основе современных теорий систем и
использования методологических возможностей, представленных достижениями
естественных наук в анализе нелинейней, неустойчивой, сложной системы.
Таковой представляется современная экономика, основанная на знаниях и
духовных ценностях.

{\bfseries Ключевые слова:} знание, духовность, потециал человека, система,
прерывность, нелинейность, неустойчивость

\begin{center}
{\large\bfseries «РУХАНИ ӨЛШЕМДЕГІ» ЭКОНОМИКА ЖӘНЕ ЖАҢА «БИЗНЕС-МОДЕЛЬ»}

{\bfseries Х. Батцэнгэл}

Моңғол жоғары оқу орнынан кейінгі білім беру университеті, Ұлан-Батор қ,
Моңғолия,

e-mail: esbise@yahoo.com
\end{center}

Қазіргі жағдайда экономика жүйе ретінде айтарлықтай өзгереді, бұл оның
жаңа экономикаға көшуіне әкеледі. Экономикалық зерттеулер жаңа теориялық
және әдіснамалық мәселелерді әзірлеуді, оларды бизнес-модельді дамытуға
және жаңасын құруға қолдануды қажет етеді, мұнда экономикалық пайда ғана
емес, рухани бастама адам мен ұйымның негізіне айналады. Руханилық
экономикада жатыр, бірақ оны тереңірек талдау жүйелердің заманауи
теориялары мен сызықтық емес, тұрақсыз, күрделі жүйені талдауда
жаратылыстану ғылымдарының жетістіктерімен ұсынылған әдістемелік
мүмкіндіктерді пайдалану негізінде мүмкін болады. Бұл білім мен рухани
құндылықтарға негізделген заманауи экономика.

{\bfseries Түйін сөздер:} білім, руханият, адамның әлеуеті, жүйе, үзіліс,
сызықтық емес, тұрақсыздық

\begin{center}
{\large\bfseries THE ECONOMY IN THE "SPIRITUAL DIMENSION" AND THE NEW "BUSINESS MODEL"}

{\bfseries Kh. Battsengel}

Graduate University of Mongolia, Ulanbaatar, Mongolia,

e-mail: esbise@yahoo.com
\end{center}

In modern conditions, the economy as a system is changing significantly,
which causes its transition to a new economy. Economic research needs to
develop new theoretical and methodological problems, apply them to the
development of a business model and create a new one, where not only
economic benefits, but the spiritual principle become the basis of human
and organizational activity. Spirituality is embedded in economics, but
a deeper analysis of it is possible on the basis of modern theories of
systems and the use of methodological possibilities presented by the
achievements of natural sciences in the analysis of a nonlinear,
unstable, complex system. Such is the modern economy based on knowledge
and spiritual values.

{\bfseries Keywords:} knowledge, spirituality, human potential, system,
discontinuity, nonlinearity, instability

\begin{multicols}{2}
{\bfseries Введение.} Проблема духовности в основном рассматривается в
философских и других аспектах, чем в экономическом. Хотя на различных
этапах развития экономической теории вплоть до ее современных
направлений проблема эта так или иначе присутсвовала в экономических
исследованиях или по крайней мере предполагалась как общий фон,
поскольку экономику просто нельзя понять отдельно от человека, его
внутреннего мира. Последний в самом общем виде можно представить как
духовность, где потенциал человека во всем своем многообразии и развитии
получает воплощение и проявление. Если знание, даже самое глубокое,
ограничено в реализации потенциала человека, то духовность способствует
более полному его раскрытию и развитию, выражая прежде всего целостность
и взаимообогащение всех составляющих внутренного мира человека.

Знание, его воплощение в разных формах, по сути своей охватывает не весь
спектр потенциала человека к труду, созданию и накоплению богатства, в
том числе духовного. Что касается духовности, то она пронизывая все
заложенные в человеке способности, обеспечивает их духовную целостность,
тем самым позволяет более полной их реализации в экономике. Духовность в
этом контексте можно определить как качество человека, позволяющее
обеспечить как бы ``синергию'' различных способностей и энергии человека
как целостной личности. В этой связи экономику нужно рассматривать не
только с точки зрения затрат и результатов деятельности, но и
потребления энергий, заложенных в человеке. Потенциал человека в таком
понимании можно представить как ``сгусток'' энергии самого различного
типа, начиная с энергии, создающие экономику как самоорганизующуюся и
саморазвивающуюся систему.

{\bfseries Материалы и методы.} В приложении к экономике идеи ее
энергетической природы выдвигали исследователи и раньше. Один из первых,
кто последовательно разработал данную проблему был экономист, философ
С.А. Подолинский(1850-1891), рассматривающий труд как особенный процесс,
который ``улавливает'' тот или иной поток энергии, превращает энергии
природы и человека в экономическое и духовное богатство. Тем самым
становится усилителем потенциала человека, основой раскрытия и развития
его способностей, что позволяет рассмотреть экономику, ее трудовую
природу и как духовно-нравственную категорию. По определению С.А.
Подолинского, ``Труд есть такое потребление механической и психической
работы, накопленной в организме, которое имеет результатом увеличение
количества превратимой энергии на земной поверхности'' {[}1{]}. В его
понимании ``увеличение количества превратимой энергии'' происходит через
увеличение и духовной энергии человека, без которой не могут быть
раскрыты и развиты его самые разносторонние способности к труду,
созданию капитала, других благ.

Не даваяся в суть его концепции экономики и ее энергетической природы
отметим, что он представлял труд как всякая полезная деятельность,
особенный процесс превращения энергии в конкретные блага, материальные,
духовные и т.д. Это уже возвышает понимание экономики, ее духовности на
новый уровень в отличии от классической экономической теории, где труд
рассматривается прежде всего как промышленный, полезность его в создании
благ, приносящих прибыль. Однако классическое понимание труда до сих пор
остается методологической основой и многих современных направлений
экономической теории без существенных изменений и дополнений, разве что
отмечается влияние технологической структуры современной экономики на
содержание труда, динамику и структуру занятости, более того и на всю
институциональную систему.

Осмысление труда и экономики в контексте их энергетической природы
позволяет представить, что полезность не ограничивается лишь чисто
экономической выгодой. Поскольку она есть результат потребления энергии,
заложенных не только в человеке, но и в природе, по определению
С.А.Подолинского, и космической энергии. Поэтому бизнес модель в силу
своих духовных составляющих, вполне может быть ориентирована на духовные
ценности помимо экономической выгоды. Возможность переориентации бизнес
модели или создания новой обусловлена тем, что духовность изначально
присуща экономике как системе. Другой вопрос -- почему на ней не
ставится акцент в целеполагании бизнес модели, каков механизм ее
функционирования и раскрытия в экономике, как задействовать данный
механизм в полную ``мощность''.

{\bfseries Результаты и обсуждение.} В современной бизнес модели все
отчетливо просматривается тенденция сочетать экономические выгоды с
другими ориентирами, в часности, решеиием социальных вопросов,
повышением социальной ответственности и т.д. Это очень важные изменения,
происходящие в бизнес модели самых различных типов. Но ``сидит'' на двух
стульях одновременно просто невозможно. Акцент должен ставиться на
одном, а другие должны из него исходить или ему подчиняться. Такова
логика ``функционирования'' духовности в экономике. До сих пор в рамках
существующей системы экономических координат мы старались сочетать
разные векторы развития, что давало лишь краткосрочные результаты.

В целополагании бизнес модели, где и как ставить акцент -- важный
вопрос, от решения которого во многом зависит переориентация бизнес
модели в нужном нам направлении. Неслучайно, что все значительные идеи
бизнеса, инновационные начинания, создание новых продуктов и производств
возникают как будто спонтанно, лишь от тех идей и мыслей, что постоянно
хотелось делать и воплотить в жизнь, создать и произвести, от
создаваемого внутри самого себя, в своей ``душе'' ``интелектуального
образа''. Таким образом, вначале создается образное представление, как
их реализовать и осуществить на деле, т.е. ``технологию'' их создания и
производства внутри себя. Такая внутренняя духовная, интелектуальная
работа во многом предшествует возникновению и формированию новых идей,
новой бизнес модели. Не экономическая необходимость, выгода, а именно
такое духовное начало определяет основную линию поведения многих
успешных, тем более инновационных компаний.

Экономика как общая система в каждой отдельно взятой своей единице по
особому превращает космическую энергию, по Подолинскому, посредством
человеческого труда в те или иные виды богатства, самые разнообразные
духовные ценности, в различные сочетания бизнес моделей и даже методов
управления, типов менежмента. Понятие ``различные сочетания'' в данном
случае не означает, что в какой то бизнес модели или системе управления
можно одновременно задействовать все силы и факторы без выделения
исходной точки, откуда начинается их действие и расходятся все
координаты. Такой точкой является духовное начало в экономической
системе, которое представляется как мысль, возникшая, образно говоря, от
``сердца'', в душе человека.

Мысль эта начинается с восприятия, которое не является просто актом
собирания информации, восприятия разнообразных сигналов извне, а прежде
всего результатом активного контакта с внутренним миром человека,
трансформации составляющих его мыслей и их взаимодействия. Одним словом
``хаос'' мыслей, если применить идеи И.Пригожина относительно механизма
``хаоса'' в системе, переходит к выделению основной линии, координата
динамики системы, т.е. возникновению и оформлению главной мысли, от
которой сталкиваются или расходятся другие{[}2{]}. Отсюда вывод, что для
новой бизнес модели важен способ улучшения возможностей восприятия
экономических агентов, которые должны развивать необходимую
чувствительность, находить способы развития своей духовной,
интелектуальной памяти о прошлом, настоящем и будущем. На этой основе
экономические агенты претерпевают внутренние структурные изменения
убеждений, идей и духовных установок, решат, что в каком бизнесе будут
участвовать или свой бизнес начинут своим умом и сердцем, а не только в
силу экономической необходимости, лишь опираясь на ``разум'', знание и
опыт. Такое взаимодействие с окружающей средой и своим внутренним миром
возможно на духовной основе, что заставляет их раскрыть полнее свои
способности и таланты, бизнес организациям выживать в конкурентной среде
и развивать свой потенциал, приспосабливаться к меняющему миру.

Поэтому от того, какая мысль стала началом построения той или иной
бизнес модели зависит, каким будет ее содержание и цель, направленность
деятельности. Если из всего многообразия и ``хаоса'' возникших во
внутреннем мире человека мыслей, их взаимодействия вы будуте следовать
мысли о своей выгоде, то у вас строится бизнес модель, где не человек, а
потребители становятся главным объектом приложения энергии, в том числе,
духовной. Когда мысль о том, чтобы работать не просто на обеспечение
потребностей человека, прежде всего на самого человека становится
основной, то строится другая бизнес модель, которую можно называть как
``новую''. Здесь духовность определяет основные координаты в развитии
экономики, что не противоречит экономическому развитию как таковому.
Просто последнее подчиняется первому, что делает саму экономическую
систему действительно динамичной, самоорганизующейся.

В современных условиях экономика как система существенно изменяется, что
довольно интенсивно и с разных теоретико-методологических позиций
анализируется учеными-экономистами. Во многих исследованиях изменение
системы рассматривается как переход к новой экономике, где основным
фактором развития становится знание. Знание более других факторов
связано с духовным началом экономики, что делает влияние духовности на
экономику значимым и важным. Это проявляется прежде всего в
переориентации экономического развития на всех уровнях, что отражается
на целевых установках экономических агентов. Однако проблема духовности
в такой постановке не всегда последовательно проработывается
исследователями в теоретическом анализе происходящих изменений в самой
экономической системе, так и в осмыслении глубиннных причин эволюции
экономической теории, которая при всем своем развитии и изменении, все
еще остается продолжением теории ``материальной цивилизации'', суть
которой, определенная французским историком Фернан Броделью до сих пор
остается основной парадигмой развития общества{[}3{]}. Даже во многом
теоретические разработки так называемой ``новой'' экономики, основанной
на знании, идут в русле той же парадигмы экономического развития,
возникшей в эпоху индустриальную. И сегодня мы живем при всех
кардинальных изменениях в той же ``материальной цивилизации'', в
обществе индустриальном, за их рамки по существу не входят все эволюция
и развитие экономики, даже переход к экономике, основанной на знаниях.

Теоретическое осмысление духовности в экономике дает возможность
взглянуть за рамки экономики, созданной ``материальной цивилизацией'',
индустриальным обществом и по новому представить переход к экономике,
основанной на знаниях. В данной постановке, переход этот означает
глубинные изменения, происходящие в самой основе экономики, где духовное
начало становится все более значимым. Это трансформируется и в
постепенную переориентацию существующей бизнес модели и создание новой.
В этой связи проблему измерения следует разработать с учетом такой
переориентации бизнес модели. Для этого важно исходит из положения, что
духовность в самом первом приближении измеряется тем, насколько полнее
раскрыт и реализован человеческий потенциал, потенциал бизнес
организации, какое количество энергий, заложенных в человеке и природе
превращено человеческим трудом в конкретные блага, какова полезность
труда и деятельности организации не только для потребителей, а именно в
том, насколько и в какой мере она направлена на человека.

Постановка проблемы в такой плоскости требует других методологических
подходов к ее разработке, поскольку теоретическое объяснение которой
возможно с точки зрения рассмотрения экономики как ``живой'' системы.
Хотя традиционное трехмерное представление экономики как взаимодействие
таких факторов как земля, труд и капитал, последовательно сменяется
теоретическим обоснованием экономики знаний. Однако даже это не может в
полной мере объяснено лишь происходящим переходом к экономике знаний и
тем, что знание становится основным фактором экономического развития. В
основе происходящего перехода лежат более глубинные причины и системные
преобразования общества, которые правомерно анализировать в контексте
современных теорий систем, используя новые возможности, представленные
развитием методологии системных исследований прежде всего в естественных
науках {[}4{]}. Эти методологические возможности следует последовательно
приложить к анализу проблемы экономики в ``духовном измерении'' и
``новой'' бизнес модели.

Забегая впредь отметим, что в экономической теории до сих пор
сохраняется традиционный подход к пониманию непрерывности движения как
основной характеристики развития. Однако надо признать, что в
современных исследованиях делаются попытки переосмысления данной
проблемы в контексте экономики как динамической системы, более того
признания прерывности и перехода к другой системе. Непрерывность
движения в экономике есть частность в линейнем развитии, что
противоречит динамической характеристике системы. Неслучайно, на
практике и в обосновании экономической политики такое понимание
экономики всегда допускает неизбежность волнообразного поведения
экономики, кризисных явлений и цикличного развития, соответственно и
строятся различные бизнес модели. Что касается прерывность развития как
состояние нелинейнной модели экономического развития, то такой подход к
анализу экономики как системы позволяет выяснить внутренний механизм
экономического развития, где духовное начало становится точкой раскрытия
заложенных в экономике энергий, если использовать определение А.С.
Подолинского.

Такое положение вещей объясняется в современной теории систем как
``хаос'' переходит к порядку и наоборот. Тот же механизм действует
внутри экономической системы, их различных моделей в разных странах и
регионах. Различие в духовных ценностях, системах мышления и
мировозрения, культурах и нравственных принципах народов как ``хаос'' в
миросистеме определяет все богатство и разнообразие духовности в мировой
экономике, что в каждой отдельно взятой единице экономики, то конкретный
человек или организация выдвигает на первый план те или иные координаты
и направления действий. Не случайно, одна и та же бизнес модель, один и
тот же тип менеждмента, одни и те же методы управления в разных странах
и регионах, на различных предприятиях и организациях имеют свои
особенности, что чрезвычайно важно учесть при всей их общности. Отсюда
вывод -- достичь каких то видимых, долгосрочных результатов не возможно
на основе механического использования даже самых лучших бизнес моделей,
методов и форм управления, более того моделей экономичекого равития без
тщательной предварительной духовной, интелектуальной работы у себя, их
приспособления соотвественно своей духовной, интелектуальной структуре
мышления.

Проникновение в экономику новых идей и методов из областей системных
исследований и развития методологии, более развитых в естественных
науках, в частности, физической и биологической науки позволяет
определить нелинейнность, прерывность, неустойчивость как основные
характеристики ``живой'' экономической системы. Такое понимание
экономической динамики существенно расширает горизонты экономического
анализа за рамки традиционных подходов и позиций, где свое ``законное
место'' может занять и духовность в экономике.

При этом нужно отметить, что динамика системы делает любой переход от
одного состояния в другое чрезвычайно противоречивым и сложным. Это
следует учесть в анализе проблем преобразования современной экономики и
перехода к новой бизнес модели в частности. На такое противоречие,
присущее нелинейной, сложной системе указывал один из основателей
нелинейной науки В.И. Арнольд. Как писал В.И.Арнольд, в нелинейной
системе при переходе из одного устойчивого состояния в другое, более
предпочтительное: 1) постепенное движение к лучшему состоянию сразу
приводит к ухудшению и его скорость при равномерном движении
увеличивается, при этом сопротивление изменению растет; 2) сопротивление
достигает максимального значения раньше перехода через наихудшее
состояние; 3) величина ухудшения, необходимая для перехода в лучшее
состояние, сравнима с финальным улучшением {[}5{]}.

В этой связи и духовность следует рассматривать как свойство экономики,
понимаемой как нелинейной, сложной системы. Поэтому в духовности
присутствует не только ``хаос'' самых различных мыслей и идей, исходящих
из внутренного мира человека, а также их превращение в реальные
действия, те или иные материальные, духовные и другие блага представляет
на деле такой же противоречивый переход от одного устойчивого состояния
в другое. Духовность не только опирается на разум, но прежде всего на
гармонию всех составляющих потенциала и способностей человека, что
обуславливает внутренний мир, внутреннюю культуру. Соответственно бизнес
модель получает другое содержание, где упор ставится не просто на
потребностей как таковых, а именно на развитии человека и т.д. Вместо
прибыльности, доходов, бизнесменов интересует не просто экономические
параметры, но и духовные {[}6{]}.

Экономические целеуказания дополняются духовными ориентирами,
направленностью, которые в свою очередь постепенно делают одухотворенной
саму экономическую деятельность, начиная с индивидуального труда, бизнес
организаций и т.д. Здесь духовным началом такой переориентации
становится свободный выбор деятельности по ``зову сердца'', от души, от
понимания того, что человеку или бизнес организации хочется делать,
какая деятельность позволит им саморазвиваться и совершить то, что
хочется, а не в зависимости от чисто экономической выгоды. Такие
возможности в условиях перехода к экономике, основанной на знаниях
существенно увеличиваются и данная тенденция в дальнейшем усилится еще
больше, хотя бы в силу развития новых технологий, дающих все более
широкие поля для самостоятельного, свободного от навязанных традиционным
разделением труда выбора деятельности и создания собственных,
неповторимых, инновационных форм бизнес творчества.

{\bfseries Выводы.} В этой связи изменяется и бизнес модель, где упор будет
сделан на ориентации бизнес деятельности сначало на оптимальное
сочетание экономических и социальных составляющих в цели,
направленности, что в традиционной бизнес модели не всегда
обеспечивается. На следующем этапе происходит подчинение экономических
целей, задач или ориентации социальным. Это затронет само содержание
деятельности экономических агентов и делает их ответственныи за духовное
развитие.

В современной экономической науке недостаточно обращает внимание на то,
что исследование проблемы духовности в экономике открывает новые
концептуальные, теоретические возможности глубже понимать экономику как
``живую'' систему. В этой связи, вместо линейной модели экономического
развития, детально разработанной, удобной и понятной, с определенностью
и предсказуемостью поведения, приходит другая с неопределенностью и
ограниченной предсказуемостью. Это требует создания новой бизнес модели,
приспособленной и устойчивой к неопределенности и постоянным изменениям
экономики, где духовность как синергия всех составляющих потенциала
человека и организации становится основой такой модели.
\end{multicols}

\begin{center}
{\bfseries Литература}
\end{center}

\begin{noparindent}
1. Подолинский С.А. Труд человека и его отношение к распределению
энергии:учебник. - М. Амрита, 2019. -132 с.

2. Пригожин И., Стенгерс И. Порядок из хаоса: Новый диалог человека с
природой / Пер. с англ. Общ. ред. В. И. Аршинова, Ю. Л. Климонтовича и
Ю. В. Сачкова. - М.: Прогресс, 1986. - 432 с.

3. Бродель Фернан. Материальная цивилизация, экономика и капитализм.
XV-XVIII вв. Учебник. -М.Альма-Матер, 2023. -- 2258 с.

4. Сорокин П.А. Социальная и культурная динамика / Пер. с англ.,
ввступ.ст. и коммент. В.В. Сапова. - М.: Академический проект, 2017. -
964 с.

5. Арнольд В.И. Математическое понимание природы: Очерки удивительных
физических явлений и их понимания математиками. - 3-е изд., стереотип. -
М.: МЦНМО, 2011. -- 144 с.

6. Морозов, В. А. Духовная экономика общества: учебное пособие. -
Москва: Дашков и К, 2023. - 2-е изд. - 458 с. - ISBN 978-5-394-05244-6.
- URL: https://znanium.com/catalog/product/1923198 (дата обращения
05.03.2024)
\end{noparindent}

\begin{center}
{\bfseries References}
\end{center}

\begin{noparindent}
1. Podolinskii S.A. Trud cheloveka i ego otnoshenie k raspredeleniyu
energii: uchebnik. - M. Amrita, 2019. -132 s. {[}in Russian{]}

2. Prigozhin I., Stengers I. Poryadok iz khaosa: Novyi dialog cheloveka
s prirodoi / Per. s angl. Obshch. red. V. I. Arshinova, Yu. L.
Klimontovicha i Yu. V. Sachkova. - M.: Progress, 1986. - 432 s. {[}in
Russian{]}

3. Brodel\textquotesingle{} Fernan. Material\textquotesingle naya
tsivilizatsiya, ekonomika i kapitalizm. XV-XVIII vv. Uchebnik.
-M.Al\textquotesingle ma-Mater, 2023. -- 2258 s. {[}in Russian{]}

4. Sorokin P.A. Sotsial\textquotesingle naya i
kul\textquotesingle turnaya dinamika / Per. s angl., vvstup.st. i
komment. V.V. Sapova. - M.: Akademicheskii proekt, 2017. - 964 s. {[}in
Russian{]}

5. Arnol\textquotesingle d V.I. Matematicheskoe ponimanie prirody:
Ocherki udivitel\textquotesingle nykh fizicheskikh yavlenii i ikh
ponimaniya matematikami. - 3-e izd., stereotip. - M.: MTsNMO, 2011. --
144 s. {[}in Russian{]}

6. Morozov, V. A. Dukhovnaya ekonomika obshchestva: uchebnoe posobie. -
Moskva: Dashkov i K, 2023. - 2-e izd. - 458 s. - ISBN 978-5-394-05244-6.
- URL: https://znanium.com/catalog/product/1923198 (data obrashcheniya
05.03.2024) {[}in Russian{]}
\end{noparindent}

\emph{{\bfseries Сведения об авторах}}

\begin{noparindent}
Батцэнгэл Хуухээ -- доктор экономических наук, профессор, директор
Института современных исследований, Монгольский университет поствысшего
образования, e-mail: esbise@yahoo.com
\end{noparindent}

\emph{{\bfseries Information about the author}}

\begin{noparindent}
Khukhe Battsengel -- Doctor of Economics, Professor, Director of the
Institute of Modern Studies, Graduate University of Mongolia, e-mail:
esbise@yahoo.com
\end{noparindent}
