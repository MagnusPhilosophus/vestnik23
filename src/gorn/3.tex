\newpage
{\bfseries МРНТИ 52.47.01}
\hfill {\bfseries \href{https://doi.org/10.58805/kazutb.v.2.23-324}{https://doi.org/10.58805/kazutb.v.2.23-324}}

\sectionwithauthors{Г.А. Исенгалиева, А.М. Балгынова, Ж.С. Саркулова, М.М. Темирханова}{РЕШЕНИЕ ПРОБЛЕМ ЭКОЛОГИИ С ПОМОЩЬЮ ОЧИСТКИ НЕФТИ И ГАЗА ОТ
СЕРОСОДЕРЖАЩИХ СОЕДИНЕНИЙ}

\begin{center}
{\bfseries Г.А. Исенгалиева\envelope, А.М. Балгынова, Ж.С. Саркулова, М.М. Темирханова}

Актюбинский региональный университет им.К. Жубанова, г.Актобе,
Казахстан.

\envelope Корреспондент-автор: isengul@mail.ru
\end{center}

Соединения серы по своему отрицательному воздействию на окружающую среду
занимают одно из первых мест среди приоритетных загрязняющих веществ.
Негативные последствия этого воздействия проявляются не только вблизи
источников выбросов, но и на весьма значительных расстояниях от них. О
влиянии техногенной эмиссии на степень загрязнения серой атмосферного
воздуха свидетельствуют многие данные.

Данная работа посвящена способам эффективной очистки нефти и газа от
серосодержащих соединений, влияния различных модифицированных
адсорбентов на эффективность очистки, а также определен качественный
состав получаемой серы - отхода сероочистки углеводородного сырья с
целью дальнейшего использования в различных отраслях промышленности.

В этой статье рассматривается решение экологических проблем путем
очистки нефти и газа от серосодержащих соединений. С целью применения
экологически чистых возобновляемых местных источников энергии для
очистки нефти и газа от сернистых соединений исследованы свойства
гелиосиликата - отходов металлургического производства.

{\bfseries Ключевые слова:} Нефть, экология, способы очистки, отходы,
сероочистка, адсорбент, меркаптаны, сероводород, сера.

\begin{center}
{\large\bfseries SOLVING ENVIRONMENTAL PROBLEMS BY CLEANING OIL AND GAS FROM SULFUR-CONTAINING COMPOUNDS}

{\bfseries G.A. Isengalieva\envelope, A.M. Balgynova, Zh.S. Sarkulova, M.M. Temirkhanova}

Aktobe Regional University named after K. Zhubanov, Aktobe, Kazakhstan

e-mail: isengul@mail.ru
\end{center}

In terms of their negative impact on the environment, sulfur compounds
occupy one of the first places among priority pollutants. The negative
consequences of this impact appear not only near emission sources, but
also at very significant distances from them. Many data indicate the
influence of technogenic emissions on the degree of sulfur pollution in
atmospheric air.

This work is devoted to methods for effective purification of oil and
gas from sulfur-containing compounds, the influence of various modified
adsorbents on the purification efficiency, and also determined the
qualitative composition of the resulting sulfur - a waste product from
the desulfurization of hydrocarbon raw materials for the purpose of
further use in various industries.

This article discusses the solution of environmental problems by
cleaning oil and gas from sulfur-containing compounds. In order to use
environmentally friendly renewable local energy sources to purify oil
and gas from sulfur compounds, the properties of heliosilicate, waste
from metallurgical production, have been studied.

{\bfseries Keywords:} Oil, ecology, purification methods, waste,
desulfurization, adsorbent, mercaptans, hydrogen sulfide, sulfur.

\begin{center}
{\large\bfseries ҚҰРАМЫНДА КҮКІРТІ БАР ҚОСЫЛЫСТАРДАН МҰНАЙ МЕН ГАЗДЫ ТАЗАРТУ АРҚЫЛЫ ЭКОЛОГИЯЛЫҚ МӘСЕЛЕЛЕРДІ ШЕШУ}

{\bfseries Г.А. Исенгалиева\envelope, А.М. Балгынова, Ж.С. Саркулова, М.М. Темирханова}

Қ. Жұбанов атындағы Ақтөбе өңірлік университеті, Ақтөбе қ., Қазақстан,

e-mail: isengul@mail.ru
\end{center}

Күкірт қосылыстары қоршаған ортаға теріс әсер етуі бойынша басым
ластаушы заттардың арасында бірінші орында. Бұл әсердің теріс салдары
тек шығарындылар көздерінің жанында ғана емес, олардан айтарлықтай
қашықтықта да көрінеді. Көптеген деректер техногендік эмиссияның сұр
атмосфералық ауаның ластану дәрежесіне әсерін көрсетеді. Бұл жұмыс
құрамында күкірт бар қосылыстардан мұнай мен газды тиімді тазарту
әдістеріне, әр түрлі модификацияланған адсорбенттердің тазарту
тиімділігіне әсеріне, сондай - ақ өндірілетін күкірттің сапалық
құрамы-өнеркәсіптің әр түрлі салаларында одан әрі пайдалану мақсатында
көмірсутек шикізатын күкірт тазартудың қалдықтарына арналған.

Бұл мақалада мұнай мен газ құрамында күкірт бар қосылыстардан тазарту
арқылы экологиялық мәселелерді шешу қарастырылған. Мұнай мен газды
күкіртті қосылыстардан тазарту үшін экологиялық таза жаңартылатын
жергілікті энергия көздерін қолдану мақсатында гелиосәулеленген
алюмосиликаттың - металлургия өндірісінің қалдықтарының қасиеттері
зерттелді.

{\bfseries Түйінді сөздер:} Мұнай, экология, тазарту әдістері, қалдықтар,
күкіртті тазарту, адсорбент, меркаптан, күкіртті сутегі, күкірт.

\begin{multicols}{2}
{\bfseries Введение.} Наличие в основной массе углеводородного сырья
большинства месторождений Западного Казахстана агрессивных
серусодержащих соединений создают трудности при добыче, транспортировке,
хранении и его переработке, что делает особо актуальной проблему
обессеривания нефти и нефтепродуктов.

Под экологической безопасностью в качестве составной части национальной
безопасности понимается состояние защищенности прав и жизненно важ-ных
интересов человека, общества и государства от угроз, возникающих в
результате антропогенных и природных воздействий на окружающую среду
{[}1{]}. Необходимым элементом процесса обеспечения экологической
безопасности в стране является определение и управление экологическими и
другими рисками. Экологический риск -- это оценка вероятности появления
негативных изменений в окружающей природной среде на всех уровнях, от
точечного до глобального, вызванных антропогенным или иным воздействием.

В процессах очистки нефти и серусодержащих газов образуется, в
результате их окисления по общепринятой технологии, элементарная сера.
Даже с учетом частичной реализации, запасы серы продолжают
увеличиваться. Серные терриконы являются возрастающей угрозой
экологической безопасности региона.

Сероводород -- весьма нежелательный спутник сернистых нефтей,
освобождение от которого требует значительного расхода реагентов,
строительства специальных установок и т.д. Он может присутствовать в
попутном газе, сопровождающем сернистые нефти, в растворенном состоянии
в самих нефтях, в продуктах первичной перегонки нефти (газах, бензиновых
дистиллятах и других светлых нефтепродуктах) или в продуктах вторичных
термических процессов (термический и каталитический крекинг,
каталитический риформинг, гидроочистка и др.) {[}2-3{]}.

Одним из экологически неблагоприятных районов Актюбинской области с
загрязненным атмосферным воздухом и почвенно-растительным покровом
является территория Жанажольского нефтегазового месторождения. По данным
наблюдений концентрация диоксида азота в атмосферном воздухе
месторождения превышает предельно допустимые концентрации в 1,5 раза.
Основными источниками загрязнения служат сырая нефть, буровой шлам,
нефтяной газ. К загрязняющим химическим веществам {[}4{]} относятся
сероводород, оксиды серы, азота, углерода, меркаптаны.

Существенной проблемой является и наличие сероводорода на месторождениях
Жанажол и Тенгиз. Отсутствие механизмов обработки и применения серы
ведет к серьезным экологическим неувязкам. Наибольшую известность
приобрели скопления серы, получаемые при очистке нефти. При годовой
производительности 3 млн. тонн стабильной сырой нефти ежедневно
вырабатывается около 1000 тонн серы. Неизбежным следствием этого
является техногенное воздействие скопившейся элементарной серы и
сероводорода на объекты окружающей среды.

В целях применения экологически чистых возобновляемых локальных
источников энергии для очистки нефтяного газа были исследованы свойства
облученного гелиоизлучением алюмосиликата - отхода металлургического
производства.

Адсорбционные свойства алюмосиликата испытывались в химической
лаборатории Жанажольского газоперерабатывающего завода. Изучены свойства
трех образцов адсорбента: 1- контроль (необработанный), 2- обработанный
гелиоизлучением в обычной стеклянной трубке, 3- обработанный
гелиоизлучением в кварцевой трубке (таблица 1).

Сероводород и меркаптаны в нефтяном газе определяли йодометрическим
методом, суть которого заключается в поглощении сероводорода и
меркаптанов из газов подкисленными растворами CdCl\textsubscript{2} и
последующем йодометрическом титровании образовавшегося CdS.

За исходный был взят газ с концентрацией сероводорода 31
г/м\textsuperscript{3}.

{\bfseries Материалы и методы.} Пробу испытуемого газа из пипетки вытесняли
10-15-кратным объемом вытеснительного газа через поглотительные склянки.
В начале продувки устанавливали скорость газа 1-2 пузырька в секунду.
Когда основная часть газа была вытеснена в раствор, скорость постепенно
увеличивали до 20 дм\textsuperscript{3}/ч.

После окончания пропуска газа анализировали содержимое поглотительных
склянок. Содержимое первой поглотительной склянки переводили
количественно в коническую колбу для титрования, тщательно ополаскивали
стенки и трубки склянки дистиллированной водой и сливали ее в ту же
колбу.

В колбу пипеткой приливали 10 см\textsuperscript{3} раствора йода
рекомендуемой концентрации и, убедившись в его избытке по бурой окраске
раствора, титровали избыток йода раствором тиосульфата натрия
Na\textsubscript{2}S\textsubscript{2}O\textsubscript{3} соответствующей
концентрации до светло-желтого цвета. Затем приливали 1
см\textsuperscript{3} раствора крахмала и продолжали титровать до
исчезновения синей окраски.

Содержимое второй поглотительной склянки анализировали аналогично
содержанию первой.

Параллельно с проведением анализа пробы испытуемого газа проводили
контрольный опыт так же, как описано выше, но без пропускания газа.

Обработка результатов

Концентрацию сероводорода в газе X, г/м\textsuperscript{3}, вычисляли по
формуле:

\begin{equation}
X = \frac{\left( V - V_{1} \right)C17}{V_{2}K} \text{г/м\textsuperscript{3}}
\end{equation}

где V- объем титрованного раствора
Na\textsubscript{2}S\textsubscript{2}O\textsubscript{3}, израсходованный
на титрование поглотительного раствора без пропускания газа (контрольный
опыт), см\textsuperscript{3};

V\textsubscript{1}- объем титрованного раствора
Na\textsubscript{2}S\textsubscript{2}O\textsubscript{3}, израсходованный
на титрование поглотительного раствора после пропускания испытуемого
газа, см\textsuperscript{3};

V\textsubscript{2} -- объем газа, измеренный газовым счетчиком,
дм\textsuperscript{3};

С -- концентрация титрованного раствора
Na\textsubscript{2}S\textsubscript{2}O\textsubscript{3}, моль/
дм\textsuperscript{3};

К -- коэффициент приведения объема газа к стандартным условиям --
температуре 20 \textsuperscript{0}С и давлению 101, 325 кПа;

17 -- масса сероводорода, соответствующая 1 см\textsuperscript{3}
титрованного раствора
Na\textsubscript{2}S\textsubscript{2}O\textsubscript{3} концентрации
точно 1 моль/ дм\textsuperscript{3}, мг.

Было пропущено 0,5 дм\textsuperscript{3} нефтяного газа. После
пропускания газа через склянки с сорбентами № 2 и 3 наблюдалось
поглощение влаги в газе.
\end{multicols}

\begin{table}[H]
\caption*{Таблица 1 - Содержание сероводорода в газе при пропускании через алюмосиликат}
\centering
\begin{tabular}{|l|l|l|}
\hline
Образец алюмосиликата & № поглотительной склянки & Концентрация H\tsb{2}S, г/м\tsp{3} \\ \hline
\multirow{3}{*}{№1 Контроль-необработанный} & 1 & 16,4 \\ \cline{2-3}
 & 2 & 3,2 \\ \cline{2-3}
 & 3 & 2,5 \\ \hline
\multirow{3}{*}{№2 Обработанный в обычном стекле} & 1 & 15,83 \\ \cline{2-3}
 & 2 & 2,2 \\ \cline{2-3}
 & 3 & 1,36 \\ \hline
\multirow{3}{*}{№3 Обработанный в кварце} & 1 & 13,2 \\ \cline{2-3}
 & 2 & 0,17 \\ \cline{2-3}
 & 3 & 0,068 \\ \hline
\end{tabular}
\end{table}

\begin{multicols}{2}
Как следует из таблицы 1, после пропускания газа через склянку с
адсорбентом №3 содержание сероводорода было меньше, что свидетельствует
о положительном эффекте от облучения адсорбента солнечным излучением.

Таким образом, использование для активации алюмосиликата гелиоизлучение
в обычном стекле позволило уменьшить содержание сероводорода в нефтяном
газе в 23 раза, а в кварцевом стекле-до 456 раз.

Далее были изучены поглотительная способность и сорбционная емкость
углеродного адсорбента на основе рисовой шелухи для очистки газа и нефти
от серосодержащих соединений.

Испытания проводились в химико-аналитической лаборатории Жанажольского
газоперерабатывающего завода. Содержание сероводорода и меркаптанов в
газе определяли при нормальных условиях по международному стандарту ISO
6326-4. За исходный был взят газ с концентрацией сероводорода 58,89
г/м\textsuperscript{3}, меркаптанов 447 г/м\textsuperscript{3} (таблица
2).

\emph{Проведение эксперимента.} Адсорбент массой 2 г помещали в колонку,
нижнюю часть которой через резиновую трубку подсоединяли к газовому
барабанному счетчику (тип ГСБ-400 кл, Р \textsubscript{раб} 5885 Па) для
регулирования объема и скорости прохождения очищаемого газа. С другой
стороны, к счетчику подключали пробоотборник с газом. Верхнюю часть
колонки, через которую выходил очищенный газ, подсоединяли к
хроматографу. Газ пропускали через нижнюю часть колонки со скоростью 1
л/мин.

Использование углеродного адсорбента позволило очистить 16 литров газа и
значительно уменьшить содержание меркаптанов в нем (почти в 5 раз).

Содержание сероводорода и меркаптанов в нефти определяли с помощью
газовой хроматографии по ГОСТ Р 50802-95. В качестве исходной была взята
нефть плотностью 0, 8027 г/см\textsuperscript{3} (таблица 2).

\emph{Проведение эксперимента.} Нефть объемом 500 мл со скоростью 60
капель/мин пропускали через слой адсорбента массой 5 г.

Применение углеродного адсорбента позволило уменьшить концентрацию
низших меркаптанов в нефти в 3-10 раз и полностью очистить от
сероводорода.

{\bfseries Результаты и обсуждение.} Из результатов анализа следует, что
углеродный адсорбент способен поглощать из газа и нефти сероводород и
меркаптаны.

В работе были также проведены исследования применения модифицированного
углеродного адсорбента для очистки бензина от меркаптанов.
Модифицирование адсорбента проводилось 10 \% растворами солей магния
(MgCl\textsubscript{2}) и меди (CuCl\textsubscript{2}).
\end{multicols}

\begin{table}[H]
\caption*{Таблица 2 - Результаты очистки нефти с помощью углеродного адсорбента}
\centering
\resizebox{\textwidth}{!}{%
\begin{tabular}{|l|llllll|}
\hline
\multirow{3}{*}{№ Пробы} & \multicolumn{6}{l|}{Содержание, мг/л} \\ \cline{2-7}
 & \multicolumn{3}{l|}{До очистки} & \multicolumn{3}{l|}{После очистки} \\ \cline{2-7}
 & \multicolumn{1}{l|}{сероводород} & \multicolumn{1}{l|}{метилмеркаптан} & \multicolumn{1}{l|}{этилмеркаптан} & \multicolumn{1}{l|}{сероводород} & \multicolumn{1}{l|}{метилмеркаптан} & этилмеркаптан \\ \hline
1 & \multicolumn{1}{l|}{46,62} & \multicolumn{1}{l|}{55,57} & \multicolumn{1}{l|}{283} & \multicolumn{1}{l|}{Отс.} & \multicolumn{1}{l|}{4,92} & 76,7 \\ \hline
2 & \multicolumn{1}{l|}{44,03} & \multicolumn{1}{l|}{50,33} & \multicolumn{1}{l|}{252} & \multicolumn{1}{l|}{Отс.} & \multicolumn{1}{l|}{4,68} & 65,6 \\ \hline
3 & \multicolumn{1}{l|}{45,89} & \multicolumn{1}{l|}{54,45} & \multicolumn{1}{l|}{281} & \multicolumn{1}{l|}{Отс.} & \multicolumn{1}{l|}{4,76} & 69,8 \\ \hline
\end{tabular}
}
\end{table}

\begin{multicols}{2}
Очистка бензина от меркаптановой серы осуществлена в динамических
условиях. В стеклянную колонку диаметром 10 мм помещали 1 г исследуемого
адсорбента и пропускали бензин со скоростью 0,5 мл/мин. Пробы отбирались
по 25 мл. В анализируемых фракциях определяли остаточное содержание
меркаптановой серы потенциометрическим титрованием аммиакатом серебра.

1. Исходный адсорбент (углеродный). Пропущено через адсорбент 500 мл
бензина с содержанием меркаптановой серы 0,0004 \%. Поглощено 0,32 мг/г
(0,01 мг-экв/г) (таблица 3).
\end{multicols}

\begin{table}[H]
\caption*{Таблица 3- Результаты очистки бензина от меркаптанов исходным углеродным адсорбентом}
\centering
\begin{tabular}{|l|l|l|l|}
\hline
№ фракции & Остаточная меркаптановая сера, мг & Поглощение, мг/г & Поглощение, \% \\ \hline
1-14 & 0,056 & 0,019 & 25,34 \\ \hline
15 & 0,075 & 0 & 0 \\ \hline
\end{tabular}
\end{table}

2. Адсорбент, модифицированный CuCl\textsubscript{2}. Пропущено 500 мл
бензина с содержанием меркаптановой серы 0,0003\%. Поглощено 0,236 мг/г
(0,007 мг-экв/г) (таблица 4).

\begin{table}[H]
\caption*{Таблица 4 - Результаты очистки бензина от меркаптанов углеродным адсорбентом, модифицированным CuCl\textsubscript{2}}
\centering
\begin{tabular}{|l|l|l|l|}
\hline
№ фракции & Остаточная меркаптановая сера, мг & Поглощение, мг/г & Поглощение, \% \\ \hline
1-7 & 0,0375 & 0,0185 & 33 \\ \hline
8-15 & 0,045 & 0,011 & 19,6 \\ \hline
16 & 0,056 & 0 & 0 \\ \hline
\end{tabular}
\end{table}

3. Адсорбент, модифицированный MgCl\textsubscript{2}. Пропущено 350 мл
бензина с содержанием меркаптановой серы 0,0006\%. Поглощено 1,14 мг/г
(0,034 мг-экв/г) (таблица 5).

\begin{table}[H]
\caption*{Таблица 5 - Результаты очистки бензина от меркаптанов углеродным адсорбентом, модифицированным MgCl\textsubscript{2}}
\centering
\begin{tabular}{|l|l|l|l|}
\hline
№ фракции & Остаточная меркаптановая сера, мг & Поглощение, мг/г & Поглощение, \% \\ \hline
1 & 0,0375 & 0,075 & 67 \\ \hline
2-6 & 0,056 & 0,056 & 50 \\ \hline
7-11 & 0,075 & 0,0375 & 33 \\ \hline
12-18 & 0,093 & 0,0188 & 16,7 \\ \hline
19 & 0,099 & 0 & 0 \\ \hline
\end{tabular}
\end{table}

Как видно из таблицы 5, очистка идет эффективнее для первых фракций.

\begin{table}[H]
\caption*{Таблица 6 - Очистка бензина от меркаптанов углеродным адсорбентом}
\centering
\begin{tabular}{|p{0.2\textwidth}|p{0.15\textwidth}|p{0.15\textwidth}|p{0.15\textwidth}|p{0.15\textwidth}|}
\hline
Образец адсорбента & Количество пропущенного бензина, мл & Исходное содержание меркаптанов в бензине, \% & Эффективность поглощения, мг/г (мг-экв/г) & Эффективность поглощения (макс.), \% \\ \hline
Исходный & 500 & 0,0004 & 0,32 (0,01) & 25,3 \\ \hline
Модифицированный CuCl2 & 500 & 0,0003 & 0,236 (0,007) & 33 \\ \hline
Модифицированный MgCl2 & 350 & 0,0006 & 1,14 (0,034) & 67 \\ \hline
\end{tabular}
\end{table}

\begin{multicols}{2}
Из экспериментальных данных следует, что наибольшее поглощение
наблюдается для адсорбента, модифицированного MgCl\textsubscript{2}, --
до 67 \% (таблица 6).

Тенгизское (Атырауская область) и Жанажолское (Актюбинская область)
месторождения нефти и газа характеризуются весьма высоким содержанием
сернистых соединений. Объемы извлекаемого углеводородного сырья
составляют десятки миллионов тонн в год {[}5,6{]}.

Актуальным на пути к решению этой проблемы встает вопрос поиска
материалов, пригодных для изготовления сорбентов, предназначенных как
для сбора нефти с поверхности воды, так и очистки сточных промышленных
вод {[}7{]}.Основные требования к оптимальному сорбенту для сбора нефти
и нефтепродуктов с поверхности воды таковы: наличие высокой
нефтепоглощающей способности, возможность регенерации вместе с
утилизацией собранной нефти, низкая стоимость и др. Основой таких
веществ являются кремнийорганические соединения, обладающие высокими
гидрофобными и сорбционными свойствами. Кремнийорганические соединения
содержатся во многих материалах, в том числе и в тех, которые уже
являются побочным результатом того или иного промышленного производства
{[}8{]}.

{\bfseries Выводы.} Особенностью меркаптансодержащего нефтяного сырья
является наличие в нем практически всего гомологического ряда
меркаптанов, от самых токсичных метил- и этилмеркаптанов до
высокомолекулярных с разветвленным строением. Поскольку для условий
транспортировки и хранения сернистых нефтей достаточно удаления из них
только сероводорода и суммы метил-, этилмеркаптанов {[}9{]}, эта задача
может быть успешно решена путем селективного извлечьния их щелочным
раствором или селективным окислением меркаптанов молекулярным
кислородом.

Таким образом, показано, что углеродный адсорбент эффективно поглощает
из углеводородного газа такие агрессивные сернистые соединения, как
сероводород и меркаптаны. Определена эффективность предложенного
углеродного адсорбента при сероочистке нефти, применение которого
позволило уменьшить концентрацию низших меркаптанов в 3-10 раз и
полностью очистить от сероводорода {[}10-12{]} В результате проведенных
экспериментов разработаны новые приемы подготовки активных адсорбентов
для очистки нефтяных углеводородов от сернистых соединений, определена
эффективность углеродного адсорбента при сероочистке нефти и природного
газа.
\end{multicols}

\begin{center}
{\bfseries Литература}
\end{center}

\begin{noparindent}
1.
  Кодекс Республики Казахстан от 2 января 2021 года № 400-VI
  «Экологический кодекс Республики Казахстан» (с изменениями и
  дополнениями от 27.12.2021 г.) {[}Code of the Republic of Kazakhstan
  dated January 2, 2021№ 400-VI "Environmental Code of the Republic of
  Kazakhstan" (amended and supplemented on December 27, 2021){]}
  /Батталова Ш.Б., Курбангалиева Г.В., Сакиева З.Ж. О сероочистке нефтей
  и нефтепродуктов // Нефть и газ. -2001. -№2. -С. 46-56.

2.
  Киреев М.А., Надиров Н.К. Экологические проблемы нефтедобывающей
  отрасли Казахстана и пути их решения // Нефть и газ Казахстана. -1998.
  -№ 4. -С. 59-62.

3.
  Архипова О.В., Обухова С.А., Везиров Р.Р., Теляшев Э.Г. Использование
  природных минеральных сорбентов в прцессах очистки нефтепродуктов //
  Нефть и газ. - 2003. - №1. -С. 58-66.

4.
  Базарбаева С.М., Сарсенов А.М. Социально-экологические и экономические
  проблемы разработки месторождений нефти на Каспии // Сб. тр. Междунар.
  семинара: «Третьи Международные Надировские чтения». - Алматы, 2005. -
  С.436-439.

5.
  Грушко Я.М. Вредные неорганические соединения в промышленных выбросах
  в атмосферу. -Л.: Химия, 1987. -192 с.

6.
  Пономарев В.Г., Иоакимис Э.Г., Монгайт И.Л. Очистка сточных вод
  нефтеперерабатывающих заводов. - М.: Химия, 1985. - 256 с.

7.
  Kudaybergenov K., Ongarbaev E., Mansurov Z., Comparison of the
  Adsorbent Performance between Carbonized Rice Husk and Abricot Stone
  According to their Structural Differences // 4 - th KKU International
  Engineering Conference (KKU - IENC 2012). - Thailand, 2012. - P.
  127-129.

8.
  ГОСТ Р 51858-2002. Нефть. Общие технические условия. --Введ.
  2002-01-08. --М: Стандартинформ. - 12 с.

9.
  Мазгаров А.М., Вильданов А.Ф., Сухов С.Н. и др. Новый процесс очистки
  нефтей и газоконденсатов от низкомолекулярных меркаптанов // Химия и
  технология топлив и масел,1996.- № 6.- С. 11--12.

10.
  Караулова Е.Н. Химия сульфидов нефти. - М.: Наука, 1970. - 202 с.

11.
  Zhadуrassyn Sarkulova, G. Lo Papa, Carmelo Dazzi, Farida Kozybaeva,
  Gulzhan Beiseyeva. Morphogenetic characteristics of chernozem leached
  in mining enterprises pollution conditions // EurAsian Journal of
  BioSciences. Eurasia J Biosci. -2019. --Vol. 13. -- Iss. 2. - P.
  1931-1941.

12.
  Abdirashit A., Makhambetov YE., Sarkulova Zh., Yerzhanov A.
  Large-scale laboratory tests for smelting medium-carbon ferromanganese
  using jezda manganese ore and simn17 silicomanganese fines//
  Metalurgija. -2023. --Vol. 62(1). -P. 139-141.
\end{noparindent}

\begin{center}
{\bfseries References}
\end{center}

\begin{noparindent}
1.
  Kodeks Respubliki Kazakhstan ot 2 yanvarya 2021 goda № 400-VI
  «Ekologicheskii kodeks Respubliki

  Kazakhstan» (s izmeneniyami i
  dopolneniyami ot 27.12.2021 g.) {[}Code of the Republic of Kazakhstan
  dated January 2, 2021№ 400-VI "Environmental Code of the Republic of
  Kazakhstan" (amended and supplemented on December 27, 2021){]}
  /Battalova Sh.B., Kurbangalieva G.V., Sakieva Z.Zh. O seroochistke
  neftei i nefteproduktov // Neft\textquotesingle{} i gaz. -2001. -№2.
  -S. 46-56. {[}in Russian{]}

2.
  Kireev M.A., Nadirov N.K. Ekologicheskie problemy neftedobyvayushchei
  otrasli Kazakhstana i puti ikh resheniya // Neft\textquotesingle{} i
  gaz Kazakhstana. -1998. -№ 4. -S. 59-62. {[}in Russian{]}

3.
  Arkhipova O.V., Obukhova S.A., Vezirov R.R., Telyashev E.G.
  Ispol\textquotesingle zovanie prirodnykh mineral\textquotesingle nykh
  sorbentov v prtsessakh ochistki nefteproduktov //
  Neft\textquotesingle{} i gaz. - 2003. - №1. -S. 58-66. {[}in
  Russian{]}

4.
  Bazarbaeva S.M., Sarsenov A.M.
  Sotsial\textquotesingle no-ekologicheskie i ekonomicheskie problemy
  razrabotki mestorozhdenii nefti na Kaspii // Sb. tr. Mezhdunar.
  seminara: «Tret\textquotesingle i Mezhdunarodnye Nadirovskie
  chteniya». -- Almaty, 2005. - S.436-439. {[}in Russian{]}

5.
  Grushko Ya.M. Vrednye neorganicheskie soedineniya v promyshlennykh
  vybrosakh v atmosferu. -- L.: Khimiya, 1987. -192 s. {[}in Russian{]}

6.
  Ponomarev V.G., Ioakimis E.G., Mongait I.L. Ochistka stochnykh vod
  neftepererabatyvayushchikh zavodov. - M.: Khimiya, 1985. - 256 s.
  {[}in Russian{]}

7.
  Kudaybergenov K., Ongarbaev E., Mansurov Z., Comparison of the
  Adsorbent Performance between Carbonized Rice Husk and Abricot Stone
  According to their Structural Differences // 4 - th KKU International
  Engineering Conference (KKU - IENC 2012). - Thailand, 2012. - P.
  127-129.

8.
  GOST R 51858-2002. Neft\textquotesingle. Obshchie tekhnicheskie
  usloviya. --Vved. 2002-01-08. --M: Standartinform. - 12 s. {[}in
  Russian{]}

9.
  Mazgarov A.M., Vil\textquotesingle danov A.F., Sukhov S.N. i dr. Novyi
  protsess ochistki neftei i gazokondensatov ot

  nizkomolekulyarnykh
  merkaptanov // Khimiya i tekhnologiya topliv i masel. 1996 № 6 S.
  11--12. {[}in Russian{]}

10.
  Karaulova E.N. Khimiya sul\textquotesingle fidov nefti. - M.: Nauka,
  1970. - 202 s. {[}in Russian{]}

11.
  Zhadуrassyn Sarkulova, G. Lo Papa, Carmelo Dazzi, Farida Kozybaeva,
  Gulzhan Beiseyeva. Morphogenetic characteristics of chernozem leached
  in mining enterprises pollution conditions // EurAsian Journal of
  BioSciences. Eurasia J Biosci. -2019. --Vol. 13. -- Iss. 2. - P.
  1931-1941.

12.
  Abdirashit A., Makhambetov YE., Sarkulova Zh., Yerzhanov A.
  Large-scale laboratory tests for smelting medium-carbon ferromanganese
  using jezda manganese ore and simn17 silicomanganese fines//
  Metalurgija. -2023. --Vol. 62(1). --P. 139-141.
\end{noparindent}

\emph{{\bfseries Сведения об авторах}}

\begin{noparindent}
Исенгалиева Г.А. - кандидат технических наук, доцент, Актюбинский
Региональный университет им.

К.Жубанова, Актобе, Казахстан, e-mail:
isengul@mail.ru;

Балгынова А.М. - кандидат технических наук, доцент, Актюбинский
Региональный университет им.

К.Жубанова, Актобе, Казахстан, e-mail:
moldir\_merei66@mail.ru;

Саркулова Ж.С. - PhD доктор, доцент, Актюбинский Региональный
университет им. К.Жубанова, Актобе, Казахстан, e-mail:
zhadi\_0691@mail.ru;

Темирханова М.М. - магистрант, Актюбинский Региональный университет им.
К.Жубанова, Актобе, Казахстан, e-mail: temirkhanova.madina@inbox.ru.
\end{noparindent}

\emph{{\bfseries Information about the authors}}

\begin{noparindent}
Isengalieva G.A. - Candidate of Technical Sciences, Associate Professor,
Aktobe Regional University named after K.Zhubanova. - Aktobe,
Kazakhstan, e-mail: isengul@mail.ru;

Balgynova A.M. - Candidate of Technical Sciences, Associate Professor,
Aktobe Regional University named after K.Zhubanova, Aktobe, Kazakhstan,
e-mail: moldir\_merei66@mail.ru;

Sarkulova Zh.S. - PhD Doctor, Associate Professor, Aktobe Regional
University named after K.Zhubanova, Aktobe, Kazakhstan, e-mail:
zhadi\_0691@mail.ru;

Temirkhanova M.M. - Master\textquotesingle s student, Aktobe Regional
University named after K.Zhubanova, Aktobe, Kazakhstan, e-mail:
temirkhanova.madina@inbox.ru.
\end{noparindent}
