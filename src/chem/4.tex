\newpage
{\bfseries МРНТИ 61.53.15}

\sectionwithauthors{Н.У.Нургалие, Ж.Б., Искакова, А.Колпек, Е.К.Айбульдинов, А.С.Сабитов, Э.Е.Копишев, Р.М.Салихов, М.С.Петров, Г.Ж Алжанова, Г.Г.Абдиюсупов, М.Т. Өмірзақ}{CОВМЕСТНЫЙ ПИРОЛИЗ НИЗКОСОРТНОГО ТОПЛИВА И ПРИРОДНОГО БИТУМА}

\begin{center}
{\bfseries \textsuperscript{1,2}Н.У.Нургалие, \textsuperscript{1}Ж.Б., Искакова, \textsuperscript{3}А.Колпек, \textsuperscript{1}Е.К.Айбульдинов, \textsuperscript{3}А.С.Сабитов, \textsuperscript{3}Э.Е.Копишев, \textsuperscript{1,4}Р.М.Салихов, \textsuperscript{1,4}М.С.Петров, \textsuperscript{1}Г.Ж Алжанова, \textsuperscript{1,5}Г.Г.Абдиюсупов, \textsuperscript{1,6}М.Т. Өмірзақ}

\textsuperscript{1}Научно-исследовательский институт Новых химических
технологий, Евразийский национальный университет им. Л.Н. Гумилева,
Астана, Казахстан,

\textsuperscript{2}Казахский университет технологии и бизнеса им. К.
Кулажанова, Астана, Казахстан,

\textsuperscript{3}Евразийский национальный университет им. Л.Н.
Гумилева, Астана, Казахстан,

\textsuperscript{4} ООО «ТТУ ЛТД», Санкт-Петербург,Россия,
\textsuperscript{5}CCS Services - Central Asia, Алматы, Казахстан,

\textsuperscript{6}Sauda Exports\&Import, Казахстан,

Корреспондент-автор: nurgaliev\_nao@mail.ru, zhanariskakova@mail.ru,
elaman\_@mail.ru
\end{center}

Углеводородные энергоре­сурсы являются основой экономики Казахстана,
среди которых особо выделяются нефть, уголь, газ. Казахстан входит в
топ-10 стран по доказанным запасам угля в размере 29,4 млрд тонн (около
2,4 \% мировых запасов), где 2/3 приходится на бурый уголь, 1/3 -- на
каменный уголь {[}1{]}. Особенно актуальным для угольной отрасли
Казахстана остается вопрос глубокой переработки низкосортного
углеводородного сырья (угольная мелочь, высокозольный уголь) и
«нетрадиционного углеводородного сырья» (высоковязкие нефти, природные
битумы, смолы и др.). В данной статье проведено исследование совместного
низкотемпературного пиролиза высокозольного угля месторождения «Борлы» с
природным битумом (при разных соотношениях угля и смолы угля с битумом)
на реторте Фишера. Основным продуктом пиролиза является полукокс, а
также в меньших концентрациях присутствует смола и горючий газ.
Приведены результаты элементного анализа и теплотворной способности
образцов исходного сырья и продуктов их пиролиза, а также результаты
анализа физико-химических показателей смолы, полученной из исследуемых
образцов.

{\bfseries Ключевые слова:} пиролиз, уголь, природный битум, смола,
полукокс, горючий газ.

\begin{center}
{\large\bfseries ТӨМЕН СОРТТЫ ОТЫН МЕН ТАБИҒИ БИТУМНЫҢ БІРЛЕСКЕН ПИРОЛИЗІ}

{\bfseries \textsuperscript{1,2}Н.У.Нургалиев, \textsuperscript{1}Ж.Б
Искакова, \textsuperscript{3}А.Колпек., \textsuperscript{1}Айбульдинов, \textsuperscript{3}А.С.Сабитов, \textsuperscript{3}Э.Е.Копишев, \textsuperscript{1,4}Р.М.Салихов, \textsuperscript{1,4} М.С.Петров, \textsuperscript{1}Г.Ж.Алжанова, \textsuperscript{1,5} Г.Г.Абдиюсупов, \textsuperscript{1,6}
М.Т.Өмірзақ}

\textsuperscript{1}Жаңа химиялық технологиялар ғылыми-зерттеу институты,
Л.Н. Гумилев атындағы Еуразия ұлттық университеті, Астана, Қазақстан,

\textsuperscript{2}Қ.Құлажанов атындағы технология және бизнес
университеті, Астана, Қазақстан,

\textsuperscript{3}Л.Н. Гумилев атындағы Еуразия ұлттық университеті,
Астана, Қазақстан,

\textsuperscript{4} ООО «ТТУ ЛТД», Санкт-Петербург, Россия,
\textsuperscript{5}CCS Services -- Central Asia, Алматы, Қазақстан,

\textsuperscript{6}Sauda Exports\&Import, Алматы, Қазақстан,

e-mail:nurgaliev\_nao@mail.ru, zhanariskakova@mail.ru, elaman\_@mail.ru
\end{center}

Көмірсутек энергиясы ресурстар Қазақстан экономикасының негізі болып
табылады, олардың ішінде мұнай, көмір, газ ерекше ерекшеленеді.
Қазақстан 29,4 млрд тонна мөлшеріндегі көмірдің дәлелденген қорлары
бойынша (әлемдік қорлардың 2,4\%-ға жуығы) 10 елдің қатарына кіреді,
онда 2/3 - қоңыр көмірге, 1/3 - тас көмірге тиесілі {[}1{]}.
Қазақстанның көмiр саласы үшiн төменгi сортты көмiрсутегi шикiзатын
(көмiр ұсақ, жоғары күлдi көмiр), сондай-ақ «дәстүрлi емес көмiрсутегi
шикiзатын» (өте тұтқыр мұнай, табиғи битумдар, шайырлар және т.б.) терең
өңдеу мәселесi әсiресе өзектi болып қалуда. Бұл мақалада Фишер
ретортында «Борлы» кен орнының жоғары күлді көмірінің табиғи битуммен
(көмір мен көмір шайырының битуммен әртүрлі арақатынасы кезінде)
бірлескен төмен температуралы пиролизіне зерттеу жүргізілді. Пиролиздің
негізгі өнімі жартылай кокс болып табылады, сондай-ақ шайыр мен жанғыш
газ аз концентрацияларда бар. Бастапқы шикізат пен олардың пиролиз
өнімдері үлгілерінің элементтік талдау және жылу шығару қабілетінің
нәтижелері, сондай-ақ зерттелетін үлгілерден алынған шайырдың
физикалық-химиялық көрсеткіштерін талдау нәтижелері келтірілген.

{\bfseries Түйін сөздер:} пиролиз, көмір, табиғи битум, шайыр, жартылай
кокс, жанғыш газ.

\begin{center}
{\large\bfseries COMBINED PYROLYSIS OF LOW-GRADE FUEL AND NATURAL BITUMEN}

{\bfseries \textsuperscript{1,2} N.U.Nurgaliyev, Zh.B.Iskakova,
\textsuperscript{3}А.Kolpek, \textsuperscript{1}Ye.K.Aibuldinov, \textsuperscript{3}A.S.Sabitov, \textsuperscript{3}E.Ye Kopishev, \textsuperscript{1,4}R.M.Salikho, \textsuperscript{1,4} M.S. Petrov, \textsuperscript{1}G.Zh.Alzhanova, \textsuperscript{1,5} G.G.Abdiyussupov G., \textsuperscript{1,6}
М.Т. Omirzak}

\textsuperscript{1} Research Institute of New Chemical Technologies,
L.N. Gumilyov Eurasian National University, Astana,

Kazakhstan,

\textsuperscript{2} Kazakh University of Technology and Business named
after K. Kulazhanov, Astana, Kazakhstan,

\textsuperscript{3} L.N. Gumilyov Eurasian National University, Astana,
Kazakhstan,

\textsuperscript{4} TTU LTD, St. Petersburg, 192283, Russia,
\textsuperscript{5}CCS Services -- Central Asia, Almaty, Kazakhstan,

\textsuperscript{6}Sauda Exports\&Import, Almaty, Kazakhstan,

e-mail:nurgaliev\_nao@mail.ru, zhanariskakova@mail.ru, elaman\_@mail.ru
\end{center}

Hydrocarbon energy resources are the basis of the economy of Kazakhstan,
among which oil, coal, and gas stand out. Kazakhstan is among the top 10
countries in terms of proven coal reserves of 29.4 billion tons (about
2.4\% of world reserves), where 2/3 is brown coal, 1/3 is hard coal
{[}1{]}. The issue of deep processing of low-grade hydrocarbon raw
materials (fine coal, high-ash coal) and ``unconventional hydrocarbon
raw materials'' (high-viscosity oils, natural bitumens, resins, etc.)
remains especially relevant for the coal industry of Kazakhstan. This
article conducts a study of joint low-temperature pyrolysis of high-ash
coal from the Borly deposit with natural bitumen (at different ratios of
coal and coal tar with bitumen) on a Fischer retort. The main product of
pyrolysis is semi-coke, and tar and flammable gas are also present in
smaller concentrations. The results of elemental analysis and calorific
value of samples of the initial raw materials and their pyrolysis
products are presented, as well as the results of an analysis of the
physicochemical parameters of the resin obtained from the studied
samples.

{\bfseries Keywords:} pyrolysis, coal, natural bitumen, resin, semi-coke,
flammable gas.

\begin{multicols}{2}
{\bfseries Введение.} В настоящее время потребление нефтяных ресурсов,
невозобновляемых запасов ископаемого топлива постоянно растут {[}2{]}, в
то время как их запасы уменьшаются, что приводит к серьезным
экологическим проблемам {[}3{]}. Вместе с тем, темпы роста мировой
экономики привели к увеличению спроса на углеводородные энергоре­сурсы,
что повлияло на развитие альтернативных нефтяных источников энергии
{[}4{]}.

Будучи ценным углеводородным ископаемым, уголь остается мировым лидером
по использованию в топливно-энергетическом комплексе и применяется для
получения металлургического кокса, смолы, углеродных материалов,
гуминовых кислот, сырья для химической промышленности (бензол, толуол,
ксилол и др.) {[}5-7{]}. Для эффективного извлечения из угля
высокоценных жидких и газообразных топлив необходимо полное
использование структуры и реакционной способности угля {[}8,9{]}.~
Органическую структуру угля принято считать сложным полимером с высокой
степенью сшивки, включающим ароматические и алифатические компоненты
{[}10{]}, {[}11{]}. Имеются существенные различия в органическом
строении углей разной степени метаморфизма {[}12{]}, а также очевидные
различия в промышленном применении.~Тщательное знание структуры угля
различной степени метаморфизма необходимо для эффективного использования
угольных ресурсов.

Одним из перспективных видов углеводородного сырья для получения
различных полезных продуктов (горючий газ, смола и др.) является
«нетрадиционное углеводородное сырье»: высоковязкие нефти, природные
битумы и др. Это обусловлено тем, что на их долю приходится в настоящее
время практически весь прирост мировых разведанных запасов
углеводородов. Подтвержденные запасы «нетрадиционного углеводородного
сырья» составляют около тысячи млрд тонн. Порядка 30\% от общей массы
ежегодных поставок энергоносителей на мировой нефтяной рынок составляет
освоение нетрадиционного углеводородного сырья {[}13{]}. Тяжелые нефти и
природные битумы характеризуются высоким содержанием ароматиче­ских
углеводородов, смолисто-ас­фальтеновых веществ, высокой кон­центрацией
металлов и сернистых соединений, высокими значениями плотности и
вязкости, повышенной коксуемостью {[}14{]}.

В настоящее время среди существующих методов термопереработки угля
пиролиз является наиболее перспективным и исследуемым термическим
направлением переработки таких отходов, как низкосортные угли,
нефтешламы, битумы и др. {[}15{]}. Пиролиз представляет собой общую
стадию многих процессов, таких как сжигание, сжижение, карбонизация,
газификация, которые обычно работают в тесных системах в инертной,
восстановительной или окислительной атмосфере при различных давлениях и
времени пребывания {[}16,17{]}. Среди ценных продуктов (получаемых из
угля) смола является основным продуктом пиролиза и может использоваться
в качестве важного сырья для получения олефинов~{[}18,19{]},
ароматических соединений с добавленной стоимостью {[}20{]}, и материалов
на основе каменноугольной смолы {[}18{]}.~

В связи с вышесказанным, определенный интерес представляет исследование
совместной термопереработки угля и природного битума.

Целью данной работы является исследование процесса низкотемпературного
пиролиза смеси высокозольного угля (месторождение «Борлы») с природным
битумом (месторождение «Беке») с определением их физико-химических
свойств и продуктов пиролиза. Продуктами пиролиза в данной работе
являются смола, полукокс и горючий газ.

Получаемые продукты являются ценным сырьем. Например, из смолы выделяют
толуол, бензол, фенол, ксилолы, другие гомологи бензола, фенол, нафталин
и другие ароматические углеводороды, которые имеют широкий спектр
применений в различных отраслях промышленности. Горючий газ, как
известно, используют в качестве топлива для получения тепловой и
электрической энергии, а полукокс используют как энергетическое и
бытовое топливо, восстановитель для химической промышленности,
металлургии.

{\bfseries Материалы и методы.} В качестве объектов исследования
использовали следующие образцы: I -- уголь месторождения «Борлы»; II --
битум; III -- смесь угля с битумом в соотношении 85/15; IV -- смесь угля
с битумом в соотношении 70/30; V -- смесь смолы (полученной от пиролиза
угля) с битумом в соотношении 20/80; VI -- смесь смолы (полученной от
пиролиза угля) с битумом в соотношении 50/50.

Для проведения анализа готовили аналитические пробы. Для оценки
химического состава исходного сырья и продуктов пиролиза приготовлены
пробы в количестве 10 грамм.

Для проведения процесса пиролиза в алюминиевой реторте Фишера
предварительно была отобрана аналитическая проба угля весом 0,6 кг и
подготовлена усредненная проба. Образцы высушивали на воздухе до
достижения приблизительного равновесия между влажностью пробы и
окружающей атмосферы. Пробы сырья были осторожно измельчены так, чтобы
не менее 90\% ее проходило через сито с отверстием размером 1 мм и не
более чем 50\% через сито 0,2 мм. Подготовленные пробы хранили в
герметически закупоренной емкости. Навеску пробы (50 г) нагревали в
реторте до 500 °С со скоростью нагрева 10 °С. Продукты разложения
направляли в приемник, охлаждаемый водой со льдом. Смола и вода
конденсировались.

Жидкие продукты, полученные в процессе пиролиза, подвергали дистилляции
с отбором фракций с температурами кипения до 200 °С, 200-360 °С и свыше
360°C.

Для определения влажности, зольности, серы, выходов продуктов
полукоксования, плотности, температур вспышки, элементного состава и
др., использовали методы в соответствии с ГОСТ 11014-2001, ГОСТ 11022-95
(ISO 1171-97), ГОСТ 1437-75, ГОСТ 3168-93, ГОСТ 3900-85, ГОСТ 4333-87,
ГОСТ 10538-87, ГОСТ 8606-93 (ISO 334-92), ASTM D 5291 (Standard Test
Methods for Instrumental Determination of Carbon, Hydrogen, and Nitrogen
in Petroleum Products and Lubricants).

Теплоту сгорания исходного сырья Q\textsubscript{н} (низшая) определяли
по формуле Д.И. Менделеева:

Q\textsubscript{н} = 81·\emph{С\textsuperscript{r}}+
246·\emph{H\textsuperscript{r}} ̶ 26·(\emph{O\textsuperscript{r}} ̶
\emph{S\textsuperscript{r}}) ̶
6·\emph{W\textsubscript{t}\textsuperscript{r}}

где \emph{С\textsuperscript{r}}- содержание в рабочей массе углерода, \%
масс; \emph{H\textsuperscript{r}}- содержание в рабочей массе водорода,
\% масс; \emph{O\textsuperscript{r}}- содержание в рабочей массе
кислорода, \% масс; \emph{S\textsuperscript{r}} - содержание в рабочей
массе летучей серы, \% масс;
\emph{W\textsubscript{t}\textsuperscript{r}}- влажность рабочей массы
топлива, \% масс.

{\bfseries Результаты и их обсуждение.} Результаты элементного анализа
образцов приведены в таблицах 1, 2.
\end{multicols}

\begin{table}[H]
\caption*{Таблица 1 - Результаты технического анализа исследуемых образцов}
\centering
\begin{tabular}{|l|l|l|l|l|}
\hline
№ & Образец & Влага, Wtr, \% & Зольность, Ar, \% & Qн*, кДж/кг (ккал/кг) \\ \hline
I & уголь & 8,3 & 59,5 & 9630,5 (2300,2) \\ \hline
II & битум & 3,2 & 80,2 & 4732,8 (1130,4) \\ \hline
III & уголь / битум (85-15) & 7 & 64,5 & 8539,0 (2039,5) \\ \hline
IV & уголь/ битум (70-30) & 5,7 & 68,4 & 7946,5 (1898,0) \\ \hline
V & смола угля / битум (20/80) & 2,4 & 64,2 & 11423,7 (2728,5) \\ \hline
VI & смола угля / битум (50/50) & 1,9 & 42,6 & 19280,2 (4605,0) \\ \hline
\end{tabular}
\end{table}
\emph{* ̶ низшая теплота сгорания}

\begin{table}[H]
\caption*{Таблица 2 - Элементный анализ исследуемых образцов (на органическую часть)}
\centering
\begin{tabular}{|l|l|l|l|l|l|l|}
\hline
№ & Образец & C, \% & H, \% & N, \% & O, \% & S, \% \\ \hline
I & уголь & 23,6 & 2,3 & 0,4 & 5,4 & 0,5 \\ \hline
II & битум & 11 & 1,4 & 0,1 & 3,7 & 0,4 \\ \hline
III & уголь / битум (85-15) & 20,7 & 2,1 & 0,4 & 4,8 & 0,5 \\ \hline
IV & уголь/ битум (70-30) & 19 & 2 & 0,3 & 4,2 & 0,4 \\ \hline
V & смола угля / битум (20/80) & 25,3 & 3,2 & 0,3 & 4,1 & 0,5 \\ \hline
VI & смола угля / битум (50/50) & 46,4 & 3,9 & 0,5 & 4,3 & 0,4 \\ \hline
\end{tabular}
\end{table}

\begin{multicols}{2}
Полученные данные показали (табл. 1), что исследуемые образцы обладают
высокими значениями зольности, особенно битум (80,2 \%). Борлинский
уголь является низкосортным углем, с высоким содержанием зольности (≈ 60
\%) и относительно низкой калорийностью. Резкое отличие по элементному
составу угля от битума в основном наблюдается по более высокому
содержанию углерода (23,6 \% и 11,0 \%), а также несущественному
превышению концентраций остальных элементов H, N, O у угля (табл. 2).
Это существенно отражается на их теплотворной способности (табл. 1),
т.к. Q\textsubscript{н} угля фактически в 2 раза превышает
Q\textsubscript{н} битума.
\end{multicols}

\begin{table}[H]
\caption*{Таблица 3 - Состав минеральной части исследуемых образцов}
\centering
\begin{tabular}{|l|l|l|l|l|l|l|l|l|l|}
\hline
№ & Образец & SiO2 & Al2O3 & Fe2O3 & CaO & MgO & Sобщ & K2O & TiO2 \\ \hline
I & уголь & 54,2 & 35,2 & 2,4 & 1,2 & 0,7 & 0,9 & 1 & 1,3 \\ \hline
II & битум & 72,5 & 8,7 & 5,9 & 4,3 & 1,4 & 1,2 & 2 & 2,7 \\ \hline
III & уголь / битум (85-15) & 59,5 & 10,8 & 7,2 & 4,6 & 2,9 & 2,6 & 1,4 & 2,6 \\ \hline
IV & уголь / битум (70-30) & 55,9 & 10,8 & 6,3 & 4,8 & 3,3 & 2,5 & 2 & 3,2 \\ \hline
\end{tabular}
\end{table}

\begin{multicols}{2}
Из таблицы 3 видно, что основную минеральную часть образцов составляют
SiO\textsubscript{2} и Al\textsubscript{2}O\textsubscript{3}. Значения
показателей минеральной части Борлинского угля в целом сопоставимы с
аналогичными данными, полученными в работах {[}21-23{]}, в которых
содержание SiO\textsubscript{2} и Al\textsubscript{2}O\textsubscript{3}
составляют соответственно 50,75-62,10 \% и 34,50-39,50 \%. Среди
основных элементов минеральной части исследуемых образцов
(SiO\textsubscript{2}, Al\textsubscript{2}O\textsubscript{3},
Fe\textsubscript{2}O\textsubscript{3}, CaO, MgO) наибольшая концентрация
SiO\textsubscript{2} наблюдается у битума (72,5 \%). Вместе с тем, уголь
обладает относительно высоким содержанием
Al\textsubscript{2}O\textsubscript{3} (35,2 \%)\textsubscript{.}

Полученные результаты низкотемпературного пиролиза исследуемых образцов
показали (табл. 4), что основными продуктами пиролиза являются полукокс,
смола и горючий газ. Причем в наибольшем количестве извлекается
полукокс, содержание которого в угле (80,8 \%) и битуме (83,2)
составляет более 80 \%. Уменьшение содержания угля на 15 \% и
аналогичное одновременное увеличение доли битума в образце IV (по
сравнению с образцом III) мало приводит к изменению содержания продуктов
пиролиза. Но повышение содержания смолы угля на 30 \% с таким же
одновременным снижением битума в образце VI (по сравнению с образцом V)
приводит к существенному повышению содержания смолы (с 23,6 до 44,5 \%)
и существенному снижению полукокса (с 69,7 до 47,4 \%), за счет высокого
содержания полукокса в битуме.

Таким образом, добавление смолы угля в битум приводит к существенному
уменьшению содержания полукокса. А для получения наибольшего количества
такого ценного продукта как смола (44,5 \%) необходимо, чтобы
соотношение: смола угля / битум составляло 50/50. Вместе с тем,
изменение соотношений в смесях (уголь с битумом, смола угля с битумом)
мало влияет на содержание газообразных продуктов.
\end{multicols}

\begin{table}[H]
\caption*{Таблица 4 - Выход продуктов пиролиза исследуемых образцов}
\centering
\begin{tabular}{|l|l|l|l|l|}
\hline
№ & Образец & Смола, \% & Полукокс, \% & Газ и потери, \% \\ \hline
I & уголь & 9,8 & 80,8 & 7,9 \\ \hline
II & битум & 8,8 & 83,2 & 6,2 \\ \hline
III & уголь / битум (85-15) & 10,7 & 80 & 7,8 \\ \hline
IV & уголь/ битум (70-30) & 9,8 & 81,1 & 7,4 \\ \hline
V & смола угля / битум (20/80) & 23,6 & 69,7 & 5,2 \\ \hline
VI & смола угля / битум (50/50) & 44,5 & 47,4 & 7 \\ \hline
\end{tabular}
\end{table}

\begin{table}[H]
\caption*{Таблица 5 - Элементный анализ полученной смолы из исследуемых образцов}
\centering
\begin{tabular}{|l|l|llllll|}
\hline
\multirow{2}{*}{№} & \multirow{2}{*}{Образец} & \multicolumn{6}{l|}{Смола} \\ \cline{3-8}
 &  & \multicolumn{1}{l|}{C, \%} & \multicolumn{1}{l|}{H, \%} & \multicolumn{1}{l|}{N, \%} & \multicolumn{1}{l|}{O, \%} & \multicolumn{1}{l|}{S, \%} & Qн, кДж/кг (ккал/кг) \\ \hline
I & уголь & \multicolumn{1}{l|}{84,6} & \multicolumn{1}{l|}{10,5} & \multicolumn{1}{l|}{0,9} & \multicolumn{1}{l|}{3,2} & \multicolumn{1}{l|}{0,8} & 39243,7 (9373,2) \\ \hline
II & битум & \multicolumn{1}{l|}{85,4} & \multicolumn{1}{l|}{12,4} & \multicolumn{1}{l|}{0,1} & \multicolumn{1}{l|}{1,4} & \multicolumn{1}{l|}{0,7} & 41657,0 (9949,6) \\ \hline
III & уголь / битум (85-15) & \multicolumn{1}{l|}{84,7} & \multicolumn{1}{l|}{10,9} & \multicolumn{1}{l|}{0,6} & \multicolumn{1}{l|}{3} & \multicolumn{1}{l|}{0,8} & 39711,4 (9484,9) \\ \hline
IV & уголь/ битум (70-30) & \multicolumn{1}{l|}{85,2} & \multicolumn{1}{l|}{11,3} & \multicolumn{1}{l|}{0,5} & \multicolumn{1}{l|}{2,1} & \multicolumn{1}{l|}{0,9} & 40401,8 (9649,8) \\ \hline
V & смола угля / битум (20/80) & \multicolumn{1}{l|}{85} & \multicolumn{1}{l|}{11,9} & \multicolumn{1}{l|}{0,2} & \multicolumn{1}{l|}{1,7} & \multicolumn{1}{l|}{1,2} & 41028,1 (9799,4) \\ \hline
VI & смола угля / битум (50/50) & \multicolumn{1}{l|}{86,1} & \multicolumn{1}{l|}{11,2} & \multicolumn{1}{l|}{0,3} & \multicolumn{1}{l|}{1,6} & \multicolumn{1}{l|}{0,8} & 40647,5 (9708,5) \\ \hline
\end{tabular}
\end{table}

\begin{multicols}{2}
Элементный анализ смолы и газа, полученных из исследуемых образцов
(таблицы 5, 6) показал, что изменение соотношений уголь/битум и смола
угля/битум в исследуемых образцах незначительно влияет на сам элементный
состав смолы и газа. Однако элементный анализ полукокса в исследуемых
образцах (таблица 7) показал, что изменение соотношения -- смола
угля/битум (с 20/80 на 50/50) в образцах V и VI, приводит к
существенному повышению доли углерода, что приводит почти к двойному
повышению калорийности полученного полукокса (с 487,8 ккал/кг до 954,5
ккал/кг).
\end{multicols}

\begin{table}[H]
\caption*{Таблица 6 - Элементный анализ полученного газа из исследуемых образцов}
\centering
\begin{tabular}{|l|l|llllll|}
\hline
\multirow{2}{*}{№} & \multirow{2}{*}{Образец} & \multicolumn{6}{l|}{Газ и потери} \\ \cline{3-8}
 &  & \multicolumn{1}{l|}{C, \%} & \multicolumn{1}{l|}{H, \%} & \multicolumn{1}{l|}{N, \%} & \multicolumn{1}{l|}{O, \%} & \multicolumn{1}{l|}{S, \%} & Qн, кДж/кг (ккал/кг) \\ \hline
I & уголь & \multicolumn{1}{l|}{39,1} & \multicolumn{1}{l|}{2,7} & \multicolumn{1}{l|}{1,1} & \multicolumn{1}{l|}{55,7} & \multicolumn{1}{l|}{1,3} & 10119,1 (2416,9) \\ \hline
II & битум & \multicolumn{1}{l|}{28,1} & \multicolumn{1}{l|}{0,4} & \multicolumn{1}{l|}{0,1} & \multicolumn{1}{l|}{70,8} & \multicolumn{1}{l|}{0,6} & 2299,8 (549,3) \\ \hline
III & уголь / битум (85-15) & \multicolumn{1}{l|}{35,4} & \multicolumn{1}{l|}{1,9} & \multicolumn{1}{l|}{1} & \multicolumn{1}{l|}{60,6} & \multicolumn{1}{l|}{1,1} & 7485,2 (1787,8) \\ \hline
IV & уголь/ битум (70-30) & \multicolumn{1}{l|}{34,8} & \multicolumn{1}{l|}{1,8} & \multicolumn{1}{l|}{1,2} & \multicolumn{1}{l|}{61,2} & \multicolumn{1}{l|}{1} & 7102,5 (1696,4) \\ \hline
V & смола угля / битум (20/80) & \multicolumn{1}{l|}{40} & \multicolumn{1}{l|}{2,2} & \multicolumn{1}{l|}{0,8} & \multicolumn{1}{l|}{55,9} & \multicolumn{1}{l|}{1,1} & 9865,8 (2356,4) \\ \hline
VI & смола угля / битум (50/50) & \multicolumn{1}{l|}{41} & \multicolumn{1}{l|}{2,7} & \multicolumn{1}{l|}{1,8} & \multicolumn{1}{l|}{53,7} & \multicolumn{1}{l|}{0,8} & 10926,7 (2609,8) \\ \hline
\end{tabular}
\end{table}

\begin{table}[H]
\caption*{Таблица 7 - Элементный анализ полученного полукокса из исследуемых образцов}
\centering
\begin{tabular}{|l|l|llllll|}
\hline
\multirow{2}{*}{№} & \multirow{2}{*}{Образец} & \multicolumn{6}{l|}{Полукокс} \\ \cline{3-8}
 &  & \multicolumn{1}{l|}{C, \%} & \multicolumn{1}{l|}{H, \%} & \multicolumn{1}{l|}{N, \%} & \multicolumn{1}{l|}{O, \%} & \multicolumn{1}{l|}{S, \%} & Qн, кДж/кг (ккал/кг) \\ \hline
I & уголь & \multicolumn{1}{l|}{15,4} & \multicolumn{1}{l|}{1,1} & \multicolumn{1}{l|}{0,4} & \multicolumn{1}{l|}{14,6} & \multicolumn{1}{l|}{0,7} & 4089,7 (976,8) \\ \hline
II & битум & \multicolumn{1}{l|}{2,1} & \multicolumn{1}{l|}{0,1} & \multicolumn{1}{l|}{0,1} & \multicolumn{1}{l|}{1} & \multicolumn{1}{l|}{0,3} & 739,0 (176,5) \\ \hline
III & уголь / битум (85-15) & \multicolumn{1}{l|}{12,3} & \multicolumn{1}{l|}{0,9} & \multicolumn{1}{l|}{0,3} & \multicolumn{1}{l|}{15,1} & \multicolumn{1}{l|}{0,5} & 3508,9 (838,1) \\ \hline
IV & уголь/ битум (70-30) & \multicolumn{1}{l|}{10,9} & \multicolumn{1}{l|}{0,7} & \multicolumn{1}{l|}{0,2} & \multicolumn{1}{l|}{12,8} & \multicolumn{1}{l|}{0,4} & 3176,5 (758,7) \\ \hline
V & смола угля / битум (20/80) & \multicolumn{1}{l|}{5,4} & \multicolumn{1}{l|}{0,3} & \multicolumn{1}{l|}{0,3} & \multicolumn{1}{l|}{1,3} & \multicolumn{1}{l|}{0,4} & 2042,3 (487,8) \\ \hline
VI & смола угля / битум (50/50) & \multicolumn{1}{l|}{10,7} & \multicolumn{1}{l|}{0,6} & \multicolumn{1}{l|}{0,2} & \multicolumn{1}{l|}{2,8} & \multicolumn{1}{l|}{0,5} & 3996,3 (954,5) \\ \hline
\end{tabular}
\end{table}

\begin{table}[H]
\caption*{Таблица 8 - Физико-химические показатели смолы из исследуемых образцов}
\centering
\begin{tabular}{|ll|p{0.1\textwidth}|p{0.1\textwidth}|l|l|p{0.11\textwidth}|}
\hline
\multicolumn{2}{|l|}{Образец} & BOB - 200°C, \% & 200°C - 360°C, \% & \textgreater{}360°C, \% & ρ, кг/м3 & температура вспышки, °С \\ \hline
\multicolumn{1}{|l|}{I} & уголь & 4,6 & 26,7 & 68,7 & 896 & 308 \\ \hline
\multicolumn{1}{|l|}{II} & битум & 1,2 & 18,6 & 80,2 & 1039 & 345 \\ \hline
\multicolumn{1}{|l|}{III} & уголь / битум (85-15) & 3,2 & 24,6 & 72,2 & 901 & 317 \\ \hline
\multicolumn{1}{|l|}{IV} & уголь/ битум (70-30) & 2,3 & 23,1 & 74,6 & 909 & 321 \\ \hline
\multicolumn{1}{|l|}{V} & смола угля / битум (20/80) & 3,5 & 20,4 & 76,1 & 1020 & 337 \\ \hline
\multicolumn{1}{|l|}{VI} & смола угля / битум (50/50) & 9,8 & 29,7 & 60,5 & 910 & 297 \\ \hline
\end{tabular}
\end{table}

\begin{multicols}{2}
Анализ физико-химических показателей получаемой смолы (из исследуемых
образцов) показал, что по сравнению с битумом, уголь характеризуется
более высокой концентрацией фракций с температурой кипения до 360 °С
(легкая, фенольная, нафталиновая и поглотительная фракция и антраценовая
фракция первая), и меньшей концентрацией фракций, кипящих свыше 360 °С
(антраценовая фракция вторая). У смолы из всех исследуемых образцов в
преобладающем количестве присутствует высококипящая (\textgreater360°C)
антраценовая фракция вторая (80,2 \%). Концентрация последней фракции
существенно уменьшается (с 76,1 до 60,5 \%) при изменении соотношения -
смола угля/битум (с 20/80 на 50/50) в образцах V и VI, что по-видимому
связано с меньшим содержанием данной фракции у смолы угля по сравнению с
таковой у битума. Однако при этом наблюдается заметное повышение доли
фракций с температурой кипения до 360 °С (до 200°C ̶ с 3,5 \% до 9,8 \%;
200°C-360°C ̶ от 20,4 \% до 29,7 \%). Вместе с тем, добавление битума до
30 \% у образца IV (по сравнению с образцом III) особо не влияет на
выход углеводородных фракций.

{\bfseries Выводы.} Результаты пиролиза смеси угля с битумом показали, что
варьированием их массовых соотношений можно получать в наибольшем
количестве те или иные продукты. Например, при наибольшем содержании
смолы угля (при соотношении: смола угля/битум равно 50/50) получается
максимальное количество смолистых продуктов (44,5 \%). Таким образом,
практическое значение полученных результатов состоит в том, что данное
исследование показало возможность вовлечения природного битума (обычно
применяемого в строительной индустрии и др.) в совместную термическую
переработку с низкосортным углем для получения таких продуктов с высокой
добавленной стоимостью, как смола, полукокс, горючий газ, что является
одной из актуальной проблем в энергетической отрасли -- расширению
сырьевой углеводородной базы.

\emph{{\bfseries Финансирование.} Настоящая работа выполнена при финансовой
поддержке Комитета науки Министерства науки и высшего образования
Республики Казахстан (№ BR21882171 «ЦУР 9.4: Развитие «зеленой»
экономики Казахстана путем переработки минерального сырья и отходов
методом пиролиза»).}

Авторы выражают благодарность за выделенное грантовое финансирование.
\end{multicols}

\begin{center}
{\bfseries Литература}
\end{center}

\begin{noparindent}
1. Институт экономических исследований Казахстана. Текущее состояние
угольной отрасли в Казахстане {[}Электронный ресурс{]}. URL:
https://economy.kz/ru/Mnenija/id=133 - Дата обращения: 19.05.2023

2.Lincoln SF. Fossil fuels in the 21st century // AMBIO A Journal of the
Human Environment, 2005.- Vol.34(8).- P.621-627. DOI
10.1639/0044-7447(2005)034{[}0621:FFITSC{]}2.0.CO;2

3.Lee XJ, Ong HC, Gan YY, Chen W-H, Mahlia TMI. State of art review on
conventional and advanced pyrolysis of macroalgae and microalgae for
biochar, bio-oil and biosyngasproduction. //Energy Conversion
Management, 2020.-Vol.210(1):112707.

DOI 10.1016/j.enconman.2020.112707

4.Guangyan Liu, Pengliang Sun, YaxiongJi, Yuanhao Wang, Hai Wang,
Xinning You{\bfseries .} Текущее состояние и энергетический анализ
процессов пиролиза горючих сланцев в мире (обзор) // Нефтехимия,
2021.-Т. 61.- № 2. -С. 138-156.

5.Михайлова Е.С., Исмагилов 3.Р., Шикина Н.В. Исследование
физико-химических свойств катализаторов в реакции озонолиза
каменноугольного сырого бензола// Химия уст. разв., 2016.-Т.24.(3) - С.
369-377. DOI:~10.15372/KhUR20160312

6.Кузнецов П.Н., Маракушина Е.Н., Бурюкин Ф.А., Исмагилов 3.Р.

Получение альтернативных пеков из углей//Химия уст.разв.-2016.-Т.24(3)-
С.325-333.\\
DOI:~10.15372/KhUR20160307

7.Жеребцов С.И., Малышенко Н.В., Смотрина О.В., Брюховецкая Л.В.,
Исмагилов 3.Р. Сорбция катионов меди нативными и модифицированными
гуминовыми кислотами

//Химия уст. разв., 2016.-Т. 24(3)- С. 399-403.
DOI:~10.15372/KhUR20160316

8.C.~Ma,~Y.~Zhao,~T.~Lang,~C.~Zou,~J.~Zhao,~Z.~Miao. Pyrolysis
characteristics of low-rank coal in a low-nitrogen pyrolysis atmosphere
and properties of the prepared chars// Energy, Elsevier, 2023.-Vol.~277:
127524. DOI 10.1016/j.energy.2023.127524

9.P.R.~Solomon,~M.A.~Serio,~E.M.~Suuberg. Coal pyrolysis: experiments,
kinetic rates and mechanisms.// Progress in Energy and Combustion
Science{\bfseries ,} 1992.-Vol.18(2).- P.133-220

https://doi.org/10.1016/0360-1285(92)90021-R

10.M.J.~Fabianska\emph{~}et al\emph{.} Biomarkers, aromatic hydrocarbons
and polar compounds in the neogene lignites and gangue sediments of the
Konin and Turoszow Brown coal basins (Poland)// International Journal of
Coal Geology, 2013.- Vol.107.- P.24 - 44.
https://doi.org/10.1016/j.coal.2012.11.008

11.M.X.~Liu\emph{~}et al. The radical and bond cleavage behaviors of 14
coals during pyrolysis with 9,10-

dihydrophenanthrene //Fuel, 2016.-
Vol.182.- P.480-486. DOI 10.1016/j.fuel.2016.06.006

12.M.~Baysal\emph{~}et al. Structure of some western Anatolia coals
investigated by FTIR, Raman,~\textsuperscript{13}C solid state NMR
spectroscopy and X-ray diffraction// International Journal of Coal
Geology{\bfseries ,}

2016.-Vol.163.- P.166-176{\bfseries .}
https://doi.org/10.1016/j.coal.2016.07.009

13.Калыбай А.А., Нәдіров Н.К., Бодыков Д.У., Абжали А.К. Высоковязкие
нефти, природные битумы, нефтяные остатки и переработка их
вакуумно-волновой гидроконверсией // Нефть и газ, 2019.-№2 (110). - С.
100-119.

14.Полетаева О.Ю., А.Ю. Леонтьев А.Ю. Тяжелые, сверхвязкие,
битуминозные, металлоносные нефти и нефтеносные песчаники //
НефтеГазоХимия, 2019.- №1.- С. 19-24. DOI:
10.24411/2310-8266-2019-10103.

15.Шантарин В.Д. Безальтернативный метод утилизации углеродосодержащих
отходов. Научное обо­зрение. Технические науки, 2016.- № 2.- С. 71-74.

16.Исламов С.Р., Степанов С.Г. Глубокая переработка угля: введение в
проблему выбора технологии // Уголь, 2007. - № 10 (978). - С. 53--58.

17.Пиролиз каменного угля: понятие и продукты {[}Электронный ресурс{]}.
URL:

https://ztbo.ru/o-tbo/stati/piroliz/piroliz-kamennogo-uglya-ponyatie-i-produkti-Дата
обращения: 02.10.2018.

18.Y.~Liu,~Q.~Yao,~M.~Sun,~X.~Ma. Catalytic fast pyrolysis of coal tar
asphaltene over zeolite catalysts to produce high-grade coal tar: an
analytical Py-GC/MS study// Journal of Analytical and Applied Pyrolysis,

2021-Vol.156:105127. DOI 10.1016/j.jaap.2021.105127

19.Y.~Che,~K.~Shi,~Z.~Cui,~H.~Liu,~Q.~Wang,~W.~Zhu,~\emph{et al.}
Conversion of low temperature coal tar into high value-added chemicals
based on the coupling process of fast pyrolysis and catalytic cracking
//Energy, Elsevier, 2023- Vol. 264(C):126169. DOI:
10.1016/j.energy.2022.126169

20.Z.-H.~Ma,~X.-Y.~Wei,~G.-H.~Liu,~F.-J.~Liu,~Z.-M.~Zong. Value-added
utilization of high-temperature coal tar: a review //Fuel, Elsevier
2021- Vol.~292:119954.

https://doi.org/10.1016/j.fuel.2020.119954

21.Мухамбетгалиев Е.К., Есенжулов А.Б., Рощин В.Е. Получение
комплексного сплава из высококремнистой марганцевой руды и высокозольных
углей Казахстана // Известия высших учебных заведений. Черная
металлургия, 2018.-Том 61.-№ 9.- С. 695-701.

22.Орлов А.С. Исследование и разработка технологии выплавки сплава
алюминий-хром-кремний с использованием в качестве восстановителя
Борлинских высокозольных углей: дисс. \ldots{} PhD: 6D070900 -
Караганда, 2020. -105 с.

23.Махамбетов Е.Н. Разработка технологии выплавки комплексных
кальцийсодержащих ферросплавов из отвальных металлургических шлаков и
высокозольных углей: дисс. \ldots{} PhD: 6D070900 -Караганда, 2021.-125
с.
\end{noparindent}

\begin{center}
{\bfseries References}
\end{center}

\begin{noparindent}
1.Economic Reserch Institute (Institut jekonomicheskih issledovanij
Kazahstana). Tekushhee sostojanie ugol\textquotesingle noj otrasli v
Kazahstane {[}Jelektronnyj resurs{]}. URL:
https://economy.kz/ru/Mnenija/id133 - Data obrashhenija:

19.05.2023

2.Lincoln SF. Fossil fuels in the 21st century // AMBIO A Journal of the
Human Environment, 2005.- Vol.34(8).- P.621-627. DOI
10.1639/0044-7447(2005)034{[}0621:FFITSC{]}2.0.CO;2

3.Lee XJ, Ong HC, Gan YY, Chen W-H, Mahlia TMI. State of art review on
conventional and advanced pyrolysis of macroalgae and microalgae for
biochar, bio-oil and biosyngasproduction. //Energy Conversion
Management, 2020.-Vol.210(1):112707.

DOI 10.1016/j.enconman.2020.112707

4.Guangyan Liu, Pengliang Sun, YaxiongJi, Yuanhao Wang, Hai Wang,
Xinning You{\bfseries .} Tekushhee sostojanie i jenergeticheskij analiz
processov piroliza gorjuchih slancev v mire (obzor) // Neftehimija,
2021.-T. 61.- № 2. -S. 138-156.{[}in Russian{]}

5.Mihajlova E.S., Ismagilov 3.R., Shikina N.V. Issledovanie
fiziko-himicheskih svojstv katalizatorov v reakcii ozonoliza
kamennougol\textquotesingle nogo syrogo benzola// Himija ust. razv.-
2016.-T.24(3) - S. 369-377.

DOI: 10.15372/KhUR20160312. {[}in Russian{]}

6.Kuznecov P.N., Marakushina E.N., Burjukin F.A., Ismagilov 3.R.
Poluchenie al\textquotesingle ternativnyh pekov iz uglej// Himija ust.
razv.- 2016. -T. 24(3)- S.325-333. DOI: 10.15372/KhUR20160307. {[}in
Russian{]}

7.Zherebcov S.I., Malyshenko N.V., Smotrina O.V., Brjuhoveckaja L.V.,
Ismagilov 3.R. Sorbcija kationov medi nativnymi i modificirovannymi
guminovymi kislotami//Himija ust. razv.- 2016.-

T. 24(3)- S. 399-403. DOI: 10.15372/KhUR20160316. {[}in Russian{]}

8.C.~Ma,~Y.~Zhao,~T.~Lang,~C.~Zou,~J.~Zhao,~Z.~Miao. Pyrolysis
characteristics of low-rank coal in a low-nitrogen pyrolysis atmosphere
and properties of the prepared chars// Energy, Elsevier, 2023.-Vol.~277:
127524. DOI 10.1016/j.energy.2023.127524

9.P.R.~Solomon,~M.A.~Serio,~E.M.~Suuberg. Coal pyrolysis: experiments,
kinetic rates and mechanisms.// Progress in Energy and Combustion
Science{\bfseries ,} 1992.-Vol.18(2).- P.133-220

https://doi.org/10.1016/0360-1285(92)90021-R

10.M.J.~Fabianska\emph{~}et al\emph{.} Biomarkers, aromatic hydrocarbons
and polar compounds in the neogene lignites and gangue sediments of the
Konin and Turoszow Brown coal basins (Poland)// International Journal of
Coal Geology, 2013.- Vol.107.- P.24 - 44.
https://doi.org/10.1016/j.coal.2012.11.008

11.M.X.~Liu\emph{~}et al. The radical and bond cleavage behaviors of 14
coals during pyrolysis with 9,10-

dihydrophenanthrene //Fuel, 2016.-
Vol.182.- P.480-486. DOI 10.1016/j.fuel.2016.06.006

12.M.~Baysal\emph{~}et al. Structure of some western Anatolia coals
investigated by FTIR, Raman,~\textsuperscript{13}C solid state NMR
spectroscopy and X-ray diffraction// International Journal of Coal
Geology{\bfseries ,}

2016.-Vol.163.- P.166-176{\bfseries .}
https://doi.org/10.1016/j.coal.2016.07.009

13.Kalybaj A.A., Nәdіrov N.K., Bodykov D.U., Abzhali A.K. Vysokovjazkie
nefti, prirodnye bitumy, neftjanye ostatki i pererabotka ih
vakuumno-volnovoj gidrokonversiej // Neft\textquotesingle{} i gaz,
2019.-№2 (110). - S. 100-119. {[}in Russian{]}

14.Poletaeva O.Ju., A.Ju. Leont\textquotesingle ev A.Ju. Tjazhelye,
sverhvjazkie, bituminoznye, metallonosnye nefti i neftenosnye peschaniki
// NefteGazoHimija, 2019.- №1.- S. 19-24. DOI:
10.24411/2310-8266-2019-10103. {[}in Russian{]}

15.Shantarin V.D. Bezal\textquotesingle ternativnyj metod utilizacii
uglerodosoderzhashhih othodov. Nauchnoe obo¬zrenie.

Tehnicheskie nauki,
2016.- № 2.- S. 71-74. {[}in Russian{]}

16.Islamov S.R., Stepanov S.G. Glubokaja pererabotka uglja: vvedenie v
problemu vybora tehnologii // Ugol\textquotesingle, 2007. - № 10 (978).
- S. 53-58. {[}in Russian{]}

17. 17.Piroliz kamennogo uglja: ponjatie i produkty {[}Jelektronnyj
resurs{]}. URL:

https://ztbo.ru/o-tbo/stati/piroliz/piroliz-kamennogo-uglya-ponyatie-i-produkti-Data
obrashhenija: 02.10.2018 -{[}in Russian{]}

18.Y.~Liu,~Q.~Yao,~M.~Sun,~X.~Ma. Catalytic fast pyrolysis of coal tar
asphaltene over zeolite catalysts to produce high-grade coal tar: an
analytical Py-GC/MS study// Journal of Analytical and Applied Pyrolysis,

2021-Vol.156:105127. DOI 10.1016/j.jaap.2021.105127

19.Y.~Che,~K.~Shi,~Z.~Cui,~H.~Liu,~Q.~Wang,~W.~Zhu,~\emph{et al.}
Conversion of low temperature coal tar into high value-added chemicals
based on the coupling process of fast pyrolysis and catalytic cracking
//Energy, Elsevier, 2023- Vol. 264(C):126169. DOI:
10.1016/j.energy.2022.126169

20.Z.-H.~Ma,~X.-Y.~Wei,~G.-H.~Liu,~F.-J.~Liu,~Z.-M.~Zong. Value-added
utilization of high-temperature coal tar: a review //Fuel, Elsevier
2021- Vol.~292:119954.

https://doi.org/10.1016/j.fuel.2020.119954

21.Muhambetgaliev E.K., Esenzhulov A.B., Roshhin V.E. Poluchenie
kompleksnogo splava iz vysokokremnistoj margancevoj rudy i
vysokozol\textquotesingle nyh uglej Kazahstana // Izvestija vysshih
uchebnyh zavedenij. Chernaja metallurgija, 2018.-Tom 61.-№ 9.- S.
695-701. {[}in Russian{]}

22.Orlov A.S. Issledovanie i razrabotka tehnologii vyplavki splava
aljuminij-hrom-kremnij s ispol\textquotesingle zovaniem v kachestve
vosstanovitelja Borlinskih vysokozol\textquotesingle nyh uglej: diss.
\ldots{} PhD: 6D070900 - Karaganda, 2020. -105 s. {[}in Russian{]}

23.Mahambetov E.N. Razrabotka tehnologii vyplavki kompleksnyh
kal\textquotesingle cijsoderzhashhih ferrosplavov iz
otval\textquotesingle nyh metallurgicheskih shlakov i
vysokozol\textquotesingle nyh uglej: diss. \ldots{} PhD: 6D070900
-Karaganda, 2021.-125 s. {[}in Russian{]}
\end{noparindent}

\emph{{\bfseries Сведения об авторах}}

\begin{noparindent}
Нургалиев Н.У.- кандидат химических наук, ассоциированный профессор,
Казахский университет технологии и бизнеса имени К. Кулажанова,
Астана,Казахстан, e-mail: nurgaliev\_nao@mail.ru;

Искакова Ж.Б. - кандидат химических наук, ассоциированный профессор
Научно-исследовательский институт Новых химических технологий,
Евразийский национальный университет им. Л.Н. Гумилева, Астана,
Казахстан, e-mail: zhanariskakova@mail.ru;

Колпек А. - кандидат химических наук, ассоциированный профессор,
Евразийский национальный университет им. Л.Н. Гумилева,
Астана,Казахстан, e-mail: aynagulk@mail.ru;

Айбульдинов Е.К.-доктор PhD, Научно-исследовательский институт Новых
химических технологий, Евразийский национальный университет им. Л.Н.
Гумилева,Астана, Казахстан, e-mail: elaman\_@mail.ru;

Сабитов А.С.- докторант, Евразийский национальный университет им. Л.Н.
Гумилева, Астана,Казахстан, e-mail: sawy552@gmail.com;

Копишев Э.Е.- кандидат химических наук, Евразийский национальный
университет им. Л.Н. Гумилева, Астана, Казахстан, e-mail:
eldar\_kopishev@mail.ru;

Салихов Р.М.-главный инженер, ООО «ТТУ ЛТД», Российская Федерация, г.
Санкт-Петербург, e-mail: info.galotar@gmail.com;

Петров М.С.- главный инженер, ООО «ТТУ ЛТД», Российская Федерация, г.
Санкт-Петербург, e-mail: info.galotar@gmail.com;

Алжанова Г.Ж. - докторант, Научно-исследовательский институт Новых
химических технологий, Евразийский национальный университет им. Л.Н.
Гумилева, Астана, Казахстан, e-mail: galiya.alzhanova@gmail.com;

Абдиюсупов Г.Г.-менеджер, ТОО «CCS Services -- Central Asia»,
Астана,Казахстан, e-mail: gaziz\_86@inbox.ru;

Өмірзақ М.Т.-доктор PhD, ТОО «Sauda Exports\&Import»,Астана, Казахстан,
e-mail: madi.omirzak@gmail.com
\end{noparindent}

\emph{{\bfseries Information about the authors}}

\begin{noparindent}
Nurgaliyev N.U.- Candidate of Chemical Science, Associate Professor
Kazakh University of Technology and Business, Kazakhstan, Astana,
e-mail: nurgaliev\_nao@mail.ru;

Iskakova Zh.B.-Candidate of Chemistry Sciences, Associate Professor,
Research Institute of New Chemical Technologies, L.N. Gumilyov Eurasian
National University, Kazakhstan, Astana, e-mail: zhanariskakova@mail.ru;

Kolpek A.-Candidate of Chemistry Sciences, Associate Professor, L.N.
Gumilyov Eurasian National University, Kazakhstan, Astana, e-mail:
aynagulk@mail.ru;

Aybuldinov E.K.-PhD, Research Institute of New Chemical Technologies,
L.N. Gumilyov Eurasian National University, Kazakhstan, Astana, e-mail:
elaman\_@mail.ru;

Sabitov A.S.- Doctoral Student, L.N. Gumilyov Eurasian National
University,Kazakhstan, Astana, e-mail:

sawy552@gmail.com;

Kopishev E.Ye - .Candidate of Chemistry Sciences, L.N. Gumilyov Eurasian
National University, Astana,

Kazakhstan, e-mail:
eldar\_kopishev@mail.ru;

Salikhov R.V.-Chief Engineer, LLC "TTU LTD", Russian Federation, St.
Petersburg, e-mail:

info.galotar@gmail.com;

Petrov M.S.-Chief Engineer, LLC "TTU LTD", Russian Federation, St.
Petersburg, e-mail:

info.galotar@gmail.com;

Alzhanova G. Zh. - Doctoral Student, Research Institute of New Chemical
Technologies, L.N. Gumilyov Eurasian National University, Kazakhstan,
Astana,

e-mail: galiya.alzhanova@gmail.com;

Abdiyussupov G.G.- Manager, CCS Services -- Central Asia LLP,
Kazakhstan, Astana, e-mail: gaziz\_86@inbox.ru;

Ómirzak M.T.-PhD, Sauda Exports\&Import LLP, Kazakhstan, Astana, e-mail:
madi.omirzak@gmail.com
\end{noparindent}
