\newpage
{\bfseries МРНТИ 31.17.29}
\hfill {\bfseries \href{https://doi.org/10.58805/kazutb.v.2.23-434}{https://doi.org/10.58805/kazutb.v.2.23-434}}

\sectionwithauthors{Л.А. Кусепова , Ф.О. Суюндикова}{ИК СПЕКТРОСКОПИЧЕСКОЕ И КВАНТОВО-ХИМИЧЕСКОЕ ИССЛЕДОВАНИЕ
СОЕДИНЕНИЙ АМИДОВ И ТИОАМИДОВ С СОЛЯНОЙ И ГЕКСАФТОРКРЕМНИЕВОЙ КИСЛОТАМИ
И СОЛЯМИ МЕДИ (II)}

\begin{center}
{\bfseries Л.А. Кусепова , Ф.О. Суюндикова}

Евразийский национальный университет имени Л.Н. Гумилева, Астана,
Казахстан,

Автор-корреспондент е-mail: kusepova71@mail.ru
\end{center}

Синтез новых координационных соединений на основе солей d-металлов с
протонированными амидами и тиоамидами, изучение их строения,
физико-химических свойств, закономерностей образования и их
идентификация является проблемной задачей современной химии. В данной
статье рассмотрены квантово-химические и ИК спектроскопические
характеристики соединений амидов и тиоамидов с соляной и
гексафторокремниевой кислотами и солями меди (II). Ионы металла образуют
координационную связь с кислородом непротонированной молекулы карбамида,
а в случае ее отсутствия с атомом азота аминогруппы, что подтверждается
квантово-химическими расчетами. Квантово-химическим методом по программе
РМ3 рассчитаны межатомные расстояния, валентные углы, заряды и
координаты атомов полученного координационного соединения. Энергетически
и геометрически более выгодным для комплексов меди (II)является
образование искаженной октаэдрической структуры, лигандами которых
являются карбамид, протонированный карбамид и анионы кислот. На основе
ИК-спектров амидокомплексов доказано протонирование амидов по атому
кислорода (серы) карбонильной группы амида (тиоамида). Кроме того, у
соединений тиоамидов в пользу S -- протонирования говорит смещение
ввысокочастотную область полосы поглощения валентных колебаний связей
С-N.

Координационные соединения меди (II) содержат во внутренней сфере
одновременно молекулы карбамида и протонированного карбамида, причем
первые связаны с комплексообразователем через атом кислорода
карбонильной группы, а вторые -- через атом азота амидной группы.
Получeнныe экспериментальные данные и установленные закономерности
кислотно-основного взаимодействия компонентов, строение полученных
комплексных соединений на основе d-металла с протонированными амидами
являются теоретической основой химии амидокомплексов, их справочным
материалом в области химии координационных соединений.

{\bfseries Ключевые слова:} координационные соединения, амиды,
квантово-химические характеристики, ИК-спектроскопия, протонирование,
разнолигандный комплекс.

\begin{center}
{\large\bfseries ТҰЗ ЖӘНЕ ГЕКСАФТОРКРЕМНИЙ ҚЫШҚЫЛДАРЫМЕН ЖӘНЕ МЫС ТҰЗДАРЫМЕН
АМИДТЕР МЕН ТИОАМИДТЕР ҚОСЫЛЫСТАРЫН ИҚ-СПЕКТРОСКОПИЯЛЫҚ ЖӘНЕ
КВАНТ-ХИМИЯЛЫҚ ЗЕРТТЕУ (II)}

{\bfseries Л.А. Кусепова, Ф.О. Суюндикова}

Л.Н.Гумилев атындағы Еуразия ұлттық университеті, Астана, Қазақстан,

е-mail: kusepova71@mail.ru
\end{center}

Протондалған амидтермен және тиамидтермен d-металл тұздары негізінде
жаңа координациялық қосылыстарды синтездеу, олардың құрылысын,
физика-химиялық қасиеттерін, түзілу заңдылықтарын зерттеу және анықтау
қазіргі химияның өзекті міндеті болып табылады. Бұл мақалада амидтер мен
тиоамидтердің тұз және гексафторкремний қышқылдары және мыс (II)
тұздарымен қосылыстарының квантты-химиялық және ИҚ-спектроскопиялық
сипаттамалары қарастырылған. Металл иондары протондалмаған карбамид
молекуласының оттегісімен, ал ол болмаған кезде амин тобының азот
атомымен координациялық байланыс түзеді, бұл квантты - химиялық
есептеулермен расталады. PM3 бағдарламасының көмегімен квантты-химиялық
әдісті қолдана отырып, атомаралық қашықтық, валенттік бұрыштар, алынған
координациялық қосылыс атомдарының зарядтары мен координаталары
есептелді. Мыс (II) комплекстері үшін лигандтары карбамид, протондалған
карбамид және қышқыл аниондары болып табылатын бұрмаланған октаэдрлік
құрылымның түзілуі энергетикалық және геометриялық тұрғыдан қолайлыболып
табылады. Амидокомплекстерінің ИҚ-спектрлері негізінде амидтің (тиоамид)
карбонил тобының оттегі (күкірт) атомындағы амидтердің протондануы
дәлелденді. Сонымен қатар, тиоамидті қосылыстарда C-N байланыстарының
валенттік тербелістерінің жұтылу аймағының жоғары жиілікті аймағына
ығысуы S-протондалуға қолайлы болады.

Мыс (II) координациялық қосылыстарында ішкі сферада карбамидтың және
протондалған карбамидтің молекулалары да біруақытта болады, олардың
біріншісі комплекс түзушіге карбонил тобының оттегі атомы арқылы, ал
екіншісі амид тобының азот атомы арқылы байланысады. Алынған тәжірибелік
мәліметтер және компоненттердің қышқылды-негіздік әрекеттесуінің
белгіленген заңдылықтары, протондалған амидтермен d-металл негізінде
түзілетін комплексті қосылыстардың құрылымы амидті комплекстер
химиясының теориялық негізі, олардың координациялық қосылыстар химиясы
саласындағы анықтамалық материалы болып табылады.

{\bfseries Түйін сөздер:} координациялық қосылыстар, амидтер,
квантты-химиялық сипаттамалар, ИҚ-спектроскопия, протонирлеу,
әртүрлілигандты комплекс.

\begin{center}
{\large\bfseries IR SPECTROSCOPIC AND QUANTUM-CHEMICAL STUDY OF COMPOUNDS OF AMIDES AND THIOAMIDES WITH HOLARIC AND HEXAFLUOROSILIC ACIDS AND COPPER (II) SALTS}

{\bfseries L.A. Kusepova, F.O.Suyndikova}

L.N.Gumilyov Eurasian National University, Astana, Kazakhstan,

е-mail: kusepova71@mail.ru
\end{center}

The synthesis of new coordination compounds based on d-metal salts with
protonated amides and thioamides, along with the study of their
structure, physicochemical properties, formation patterns, and
identification, poses a challenging task in modern chemistry. This
article examines the quantum chemical and IR spectroscopic
characteristics of compounds of amides and thioamides with hydrochloric
and hexafluorosilicic acids and copper (II) salts. Metal ions form a
coordination bond with the oxygen of the unprotonated urea molecule, and
in its absence, with the nitrogen atom of the amino group, as confirmed
by quantum chemical calculations.Using the quantum chemical method with
the PM3 program, interatomic distances, bond angles, charges, and
coordinates of the atoms of the resulting coordination compound were
calculated. Energetically and geometrically more favorable for copper
(II) complexes is the formation of a distorted octahedral structure, the
ligands of which are urea, protonated urea and acid anions. Based on the
IR spectra of amido complexes, the protonation of amides at the oxygen
(sulfur) atom of the carbonyl group of the amide (thioamide) has been
demonstrated. In addition, in thioamide compounds, S-protonation is
favored by a shift to the high-frequency region of the absorption band
of stretching vibrations of C-N bonds.

Copper (II) coordination compounds contain both urea and protonated urea
molecules in the inner sphere, the former being connected to the
complexing agent through the oxygen atom of the carbonyl group, and the
latter through the nitrogen atom of the amide group. The experimental
data obtained and the established patterns of acid-base interaction of
the components provide the theoretical basis for the chemistry of amido
complexes, serving as reference material in the field of coordination
compound chemistry.

{\bfseries Keywords:} coordination compounds, amides, quantum chemical
characteristics, IR spectroscopy, protonation, mixed-ligand complex.

\begin{multicols}{2}
{\bfseries Введение.} Одной из актуальных проблем химии амидов является
синтез новых химических соединений, поиск возможных областей
практического применения полученных координационных соединений.
Образуемые амидокислоты с солями d-металла совмещают свойства исходных
компонентов с вновь приобретенными, они используются как удобрения,
пестициды, кормовые добавки, в синтезе лекарственных препаратов,
полимеров, важны в синтетической органической химии {[}1-5{]}. С
теоретической точки зрения интересным является изучение
кислотно-основного взаимодействия в системах амид -- кислота,
установление места протонирования и расшифровка структур полученных
соединений. Карбамид и ацетамид - это лиганды, которые могут
присоединяться к комплексообразователю как через атом кислорода
карбонильной группы, так и через атом азота амидной группы.

На кафедре химии ЕНУ им. Л. Н. Гумилева ведутся исследования процессов
взаимодействия различных солей с амидами и тиоамидами в
четырехкомпонентных системах, содержащих соли d-металлов -- амид --
кислоту -- воду {[}6-8{]}. Методом растворимости изучено
кислотно-основное взаимодействие между компонентами системы, характер их
взаимодействия, установлен состав образующихся при этом комплексных
соединений, для которых разработаны условия синтеза в кристаллическом
виде {[}9{]}. Для идентификации и расшифровки структуры некоторых
полученных соединений амидов и тиоамидов с соляной и
гексафторокремниевой кислотами и солями меди (II) были проведены
квантово-химическое и ИК спектроскопическое исследования.

Целью работы является квантово-химическое и ИК спектроскопическое
исследования соединений амидов и тиоамидов с соляной и
гексафторокремниевой кислотами и солями меди (II).

В соответствии с целью были поставлены следующие задачи:
интерпретирование данных по квантово-химическому расчету и анализ
результатов ИК-спектров исследуемых соединений.

{\bfseries Материалы и методы.} Для определения устойчивости комплексов
хлоридамеди (II) с протонированным карбамидом в зависимости от числа
протонированных молекул карбамида проведены квантово-химические расчеты
комплексаCuCl\textsubscript{2}\textsuperscript{.}2CO(NH\textsubscript{2})\textsubscript{2}\textsuperscript{.}HCl.
Для решения поставленной задачи был выбран полуэмпирический метод
квантовой химии РМ3, входящий в программный блок HyperChem 6.0 {[}10{]}.

Строение синтезированных соединений установлены при помощи анализа
положения в ИК -- спектрах характеристических полос поглощения различных
функциональных групп амидов {[}11{]}. ИК-спектры поглощения записывали в
области 400-4000 см\textsuperscript{-1} на спектрометре ИК Фурье IR 20 с
применением методики прессования образцов с КВr.

{\bfseries Результаты и обсуждение.} Квантово-химические расчеты ранее
изученных координационных соединений показывают заметное изменение
электронных характеристик, как атомов молекулы амида (С, N, O), так и в
молекулах комплекса в целом {[}12{]}. Например, анализ полученных данных
для комплекса
NiCl\textsubscript{2}\textsuperscript{.}4CO(NH\textsubscript{2})\textsubscript{2}\textsuperscript{.}2НCl
показывает, что наиболее энергетически и геометрически выгодным для
никеля является октаэдрический разнолигандный комплекс
{[}Ni(CO(NH\textsubscript{2})\textsubscript{2})\textsubscript{2}(CO(NH\textsubscript{2})\textsubscript{2}H\textsuperscript{+})\textsubscript{2}Cl\textsubscript{2}{]}Cl\textsubscript{2},
причем 2 молекулы карбамида протонированы и находятся во внутренней
сфере. Координация осуществляется по двум атомам кислорода и двум атомам
азота четырех молекул карбамида, а также двум ионам хлора.

В комплексе
CuCl\textsubscript{2}\textsuperscript{.}2CO(NH\textsubscript{2})\textsubscript{2}\textsuperscript{.}HCl
наблюдается димеризация атомов меди. Кристаллы имеют слоистое строение:
расстояние между соседними атомами Cu (II), ориентированных вдоль оси z,
равно 2,278 Å. В координации с атомом металла принимают участие атом
кислорода молекулы карбамида, атом азота протонированного карбамида и
три иона хлора. Координационный полиэдр Cu имеет форму слегка
искаженного октаэдра. Атомы кислорода и азота (по одному от каждого)
занимают положения на более близких расстояниях от атома меди в
сравнении с ионами хлора (табл. 1).
\end{multicols}

\begin{table}[H]
\caption*{Таблица 1 - Геометрические параметры связей Cu ˗˗ O, Cu˗˗N, Cu˗˗Cl и Cu˗˗Cu}
\centering
\begin{tabular}{|l|l|}
\hline
Связь & Расстояние (d,Å) \\ \hline
Cu ˗ O (14) & 2,076 \\ \hline
Cu ˗ N (12) & 1,916 \\ \hline
Cu ˗ Cl (18) & 2,246 \\ \hline
Cu ˗ Cl (19) & 2,22 \\ \hline
Cu ˗ Cl (20) & 2,358 \\ \hline
Cu ˗ Cu & 2,278 \\ \hline
\end{tabular}
\end{table}

Длина связей O(2) ˗ H⁺(26) равна 1,007Å.

Распределение заряда и положение атомов в пространстве представлены в
(табл.2).

\begin{table}[H]
\caption*{Таблица 2 - Заряды (q)и координаты (x,y,z) атомов в долях осей элементарной ячейки комплекса CuCl\textsubscript{2}\textsuperscript{.}2CO(NH\textsubscript{2})\textsubscript{2}\textsuperscript{.}HCl}
\centering
\begin{tabular}{|l|l|l|l|l|}
\hline
Атом & q & x & y & z \\ \hline
Cu (21) & -0.228 & -1.215 & 0.942 & -0.459 \\ \hline
Cu (22) & -0.957 & 0.658 & -0.298 & -0.075 \\ \hline
N (12) & 0.726 & -0.191 & 2.283 & -1.366 \\ \hline
O (14) & -0.337 & -1.456 & 2.019 & 1.299 \\ \hline
Cl (16) & 4.019 & 2.185 & -1.562 & 0.143 \\ \hline
Cl (18) & -0.413 & -3.013 & 2.181 & -0.987 \\ \hline
Cl (19) & -0.220 & -2.597 & -0.529 & 0.468 \\ \hline
Cl (20) & -0.006 & -1.579 & -0.113 & -2.537 \\ \hline
H⁺ & 0.284 & -2.173 & 3.742 & -0.965 \\ \hline
\end{tabular}
\end{table}

Искажения октаэдрической структуры координационного соединения
доказывают значения валентных углов в
CuCl\textsubscript{2}\textsuperscript{.}2CO(NH\textsubscript{2})\textsubscript{2}\textsuperscript{.}HCl
(табл.3)

\begin{table}[H]
\caption*{Таблица 3 - Валентные углы (ώ, град.) в комплексе CuCl\textsubscript{2}\textsuperscript{.}2CO(NH\textsubscript{2})\textsubscript{2}\textsuperscript{.}HCl}
\centering
\begin{tabular}{|l|l|}
\hline
Угол & ώ,град. \\ \hline
N(12) – Cu – O(14) & 95,69 \\ \hline
O(14) – Cu – Cl(19) & 85,3 \\ \hline
Cl(19) – Cu – Cl(20) & 88,59 \\ \hline
N(12) – Cu – Cl(20) & 88,79 \\ \hline
Cl(18) – Cu –Cl(20) & 85,2 \\ \hline
Cl(18) – Cu – N(12) & 85,99 \\ \hline
Cl(18) – Cu –O(14) & 79,61 \\ \hline
Cl(18) – Cu – Cl(19) & 88 \\ \hline
Cl(16) – Cu(22) – Cu(21) & 173,02 \\ \hline
Cl(18) – Cu(21) – Cu(22) & 175,98 \\ \hline
\end{tabular}
\end{table}

Энергетические параметры квантово-химического расчета приведены в
(табл.4)

\begin{table}[H]
\caption*{Таблица 4 - Энергетические характеристики (кДж/моль) комплекса CuCl\textsubscript{2}\textsuperscript{.}2CO(NH\textsubscript{2})\textsubscript{2}\textsuperscript{.}HCl}
\centering
\begin{tabular}{|l|l|l|l|l|}
\hline
Комплекс & Eобщая & Eатом & Eэлектрон & ΔHобр \\ \hline
CuCl\tsb{2}.2CO(NH\tsb{2})\tsb{2}\tsp{.}HCl & -7,229×10\tsp{5} & -7,084×10\tsp{5} & -5,110×10\tsp{6} & -1,537×10\tsp{3} \\ \hline
\end{tabular}
\end{table}

\begin{multicols}{2}
Энергия связывания атомов равна 4,387×10\textsuperscript{7}кДж/моль.

Принимая во внимание результаты работы, можно утверждать об искажении
октаэдрического строения комплекса
CuCl\textsubscript{2}\textsuperscript{.}2CO(NH\textsubscript{2})\textsubscript{2}\textsuperscript{.}HCl.
Комплекс хлорида меди с протонированным карбамидом является димером
следующей структуры {[}Cu
(CO(NH\textsubscript{2})\textsubscript{2}H⁺{]}CO(NH\textsubscript{2})\textsubscript{2}Cl\textsubscript{3}.

Определяющим фактором для образования соединения с определенным
количеством лигандов является природа металла-комплексообразователя.
Структурные исследования координационных соединений методом
квантово-химического расчета дает возможность утверждать, что более
энергетически выгодным для меди является образование комплексов с
координационным числом шесть. Получение таких характеристик для новых
соединений при помощи расчетного квантово-химического метода позволит
пополнить банк термодинамических характеристик, которые могут быть
использованы в качестве индексов при оценке их относительной реакционной
способности и справочных данных.

Соединения солей меди с протонированным карбамидом содержат во
внутренней сфере одновременно молекулы карбамида и протонированного
карбамида, причем первые связаны с комплексообразователем через атом
кислорода карбонильной группы, а вторые - через атом азота аминной
группы.

Протонирование у амидов и тиоамидов может осуществляться либо по атому
кислорода (серы) карбонильной группы, либо по атому азота амидной
группы. Анализ ИК спектров поглощения соединений амидов и тиоамидов с
соляной и фторокомплексными кислотами покажет, который из двух
атомов-доноров электронов протонируется {[}13{]}. Для определения центра
протонирования были записаны ИК спектры диметилацетамида, ацетамида,
бензамида, тиокарбамида, диацетилтиокарбамида и их соединений с соляной
и фторокомплексными кислотами, а также ссолями меди (II).

Ацетилкарбамид и фенилкарбамид являются замещенными карбамида, поэтому
увеличение числа метильных групп, обладающих положительным индуктивным
эффектом, введение фенильной группы в молекулу карбамида должно вызывать
деформацию CO и NH\textsubscript{2} групп, которая приводит к некоторому
уменьшению суммарной электронной плотности на связи С=О и уменьшению ее
энергии. Отнесение частот в спектрах соединений фенилкарбамида,
ацетилкарбамида с гексафторокремниевой кислотой приведено в таблице 5.
\end{multicols}

\begin{table}[H]
\caption*{Таблица 5 - Значения характеристических частот (см\textsuperscript{-1}) в ИК спектрах поглощения соединений фенилкарбамида, ацетилкарбамида, бензамида с гексафторокремниевой кислотой}
\centering
\resizebox{\textwidth}{!}{%
\begin{tabular}{|l|llllll|}
\hline
\multirow{2}{*}{Отнесение} & \multicolumn{6}{l|}{Соединение} \\ \cline{2-7}
 & \multicolumn{1}{l|}{C\tsb{6}H\tsb{5}NHCONH\tsb{2}} & \multicolumn{1}{l|}{2C\tsb{6}H\tsb{5}NHCONH\tsb{2}\tsp{.}H\tsb{2}SiF\tsb{6}} & \multicolumn{1}{l|}{CH\tsb{3}СОNHCONH\tsb{2}} & \multicolumn{1}{l|}{2CH\tsb{3}СОNHCONH\tsb{2}\tsp{.}H\tsb{2}SiF\tsb{6}} & \multicolumn{1}{l|}{C\tsb{6}Н\tsb{5}CONH\tsb{2}} & \begin{tabular}[c]{@{}l@{}}C\tsb{6}Н\tsb{5}CONH\tsb{2}\tsp{.}\\  H\tsb{2}SiF\tsb{6}\end{tabular} \\ \hline
ν (OН) & \multicolumn{1}{l|}{} & \multicolumn{1}{l|}{3560} & \multicolumn{1}{l|}{} & \multicolumn{1}{l|}{3435} & \multicolumn{1}{l|}{} & 3410 \\ \hline
ν\tsb{a} (NH) & \multicolumn{1}{l|}{3430} & \multicolumn{1}{l|}{3410} & \multicolumn{1}{l|}{3350} & \multicolumn{1}{l|}{3365} & \multicolumn{1}{l|}{3380} & 3320 \\ \hline
ν\tsb{s}(NH) & \multicolumn{1}{l|}{3310} & \multicolumn{1}{l|}{3335} & \multicolumn{1}{l|}{3250} & \multicolumn{1}{l|}{3285} & \multicolumn{1}{l|}{3320} & 3250 \\ \hline
δ (COH) & \multicolumn{1}{l|}{} & \multicolumn{1}{l|}{1700} & \multicolumn{1}{l|}{} & \multicolumn{1}{l|}{} & \multicolumn{1}{l|}{} & 1700 \\ \hline
δ (NH2) & \multicolumn{1}{l|}{1680} & \multicolumn{1}{l|}{1680} & \multicolumn{1}{l|}{1640} & \multicolumn{1}{l|}{1635} & \multicolumn{1}{l|}{1618} & 1610 \\ \hline
ν (CO) & \multicolumn{1}{l|}{1605} & \multicolumn{1}{l|}{1615} & \multicolumn{1}{l|}{1680} & \multicolumn{1}{l|}{1660} & \multicolumn{1}{l|}{1660} & 1690 \\ \hline
ν кольца & \multicolumn{1}{l|}{\begin{tabular}[c]{@{}l@{}}1590\\  1560\end{tabular}} & \multicolumn{1}{l|}{\begin{tabular}[c]{@{}l@{}}1592\\  1555\end{tabular}} & \multicolumn{1}{l|}{} & \multicolumn{1}{l|}{} & \multicolumn{1}{l|}{\begin{tabular}[c]{@{}l@{}}1610\\  1570\end{tabular}} & \begin{tabular}[c]{@{}l@{}}1600\\  1565\end{tabular} \\ \hline
ν\tsb{a}(CN) & \multicolumn{1}{l|}{1450} & \multicolumn{1}{l|}{1503} & \multicolumn{1}{l|}{1420} & \multicolumn{1}{l|}{1460} & \multicolumn{1}{l|}{1405} & 1465 \\ \hline
ν\tsb{s}(CN) & \multicolumn{1}{l|}{1035} & \multicolumn{1}{l|}{1042} & \multicolumn{1}{l|}{\begin{tabular}[c]{@{}l@{}}945\\  815\end{tabular}} & \multicolumn{1}{l|}{820} & \multicolumn{1}{l|}{} &  \\ \hline
δ кольца & \multicolumn{1}{l|}{\begin{tabular}[c]{@{}l@{}}865\\  755\end{tabular}} & \multicolumn{1}{l|}{\begin{tabular}[c]{@{}l@{}}840\\  740\end{tabular}} & \multicolumn{1}{l|}{} & \multicolumn{1}{l|}{} & \multicolumn{1}{l|}{} &  \\ \hline
δ (CO) & \multicolumn{1}{l|}{700} & \multicolumn{1}{l|}{695} & \multicolumn{1}{l|}{703} & \multicolumn{1}{l|}{720} & \multicolumn{1}{l|}{} &  \\ \hline
\end{tabular}
}
\end{table}

\begin{multicols}{2}
В области частот валентных колебаний связей N-H в спектре фенилкарбамида
наблюдается несколько полос поглощения (3225, 3310, 3430
см\textsuperscript{-1}). Высокочастотная компонента отнесена к валентным
антисимметричным колебаниям связей N-H, а частоты с максимумами при
(3225-3310 см\textsuperscript{-1}) обусловлены валентными симметричными
колебаниями этих связей. В спектре соединения дигидрогексафторосиликата
дифенилкарбамида
2C\textsubscript{6}H\textsubscript{5}NHCONH\textsubscript{2}{\bfseries \textsuperscript{.}}H\textsubscript{2}SiF\textsubscript{6}
высокочастотная компонента (3430см\textsuperscript{-1}) смещена в
низкочастотную область (3410см\textsuperscript{-1}).

Полоса поглощения при 3310см\textsuperscript{-1}, обусловленная
симметричными валентными колебаниями связей N-H, также раздваивается в
спектре соединения на высокочастотную при 3335см\textsuperscript{-1} и
низкочастотную 3240 см\textsuperscript{-1}.

К валентным колебаниям связи С=О в спектре фенилкарбамида отнесены
полосы поглощения при 1605 см\textsuperscript{-1}, а в спектре
ацетилкарбамида к валентным колебаниям связи C=O амидного фрагмента
отнесены полосы поглощения при 1680 см\textsuperscript{-1}.
Высокочастотная компонента обусловлена колебаниями ν(СО) ацетильного
фрагмента молекулы, низкочастотная -ν(СО) амидного фрагмента.

Полоса ν(СО) амидного фрагмента в спектре дигидрогексафторосиликата
диацетилкарбамида
2CH\textsubscript{3}СОNHCONH\textsubscript{2}{\bfseries \textsuperscript{.}}H\textsubscript{2}SiF\textsubscript{6}
смещена в низкочастотную область, что вызвано ослаблением этой связи за
счет О -- протонирования. В пользу протонирования по атому карбонильного
кислорода свидетельствует также высокочастотное смещение полос
поглощения связи C-N(на 20-25 см\textsuperscript{-1}) в спектрах
соединений фенилкарбамида и ацетилкарбамида с гексафторокремниевой
кислотой. В спектрах полученных амидокомплексов наблюдаются полосы ν(ОН)
при 3560см\textsuperscript{-1}, что также указывает на О --
протонирование.

В спектре дигидрогексафторосиликата диацетилкарбамида полоса поглощения
деформационных колебаний δ(NH\textsubscript{2}) практически не смещается
(1680 см\textsuperscript{-1}). Практически неизменное положение полос
поглощения деформационных колебаний аминогруппы в спектрах соединений
также подтверждает протонирование по карбонильному кислороду.

Полосы валентных колебаний связей N-H в ИКспектре свободного бензамида
наблюдаются две полосы поглощения (таблица 5). Высокочастотная
компонента при 3380 см\textsuperscript{-1} обусловлена валентными
антисимметричными колебаниями, а низкочастотная при 3320
см\textsuperscript{-1} симметричными колебаниями и этих связей. В
спектрах соединения с комплексной кислотой соответствующая полоса
поглощения наблюдается в более низкочастотной области на 70
см\textsuperscript{-1}. Понижение частоты в спектре соединения, как и в
случае производных карбамидных соединений, объясняется образованием
новых водородных связей NH\ldots An , где An\textsuperscript{-} -- анион
кислоты. Интенсивная полоса поглощения при 1660 см\textsuperscript{-1}
обусловлена валентными колебаниями связи С=О, в спектрах
дигидрогексафторосиликата дибензамида
2CH\textsubscript{3}СОNHCONH\textsubscript{2}{\bfseries \textsuperscript{.}}H\textsubscript{2}SiF\textsubscript{6}
ν(СО) несколько смещена в сторону высоких частот
область(1690см\textsuperscript{-1}). Однако в спектре соединения
наблюдается полоса поглощения валентных колебаний связи О-Н
(3410см\textsuperscript{-1}), указывающая на протонирование по
кислороду. Кроме того, полоса поглощения валентных колебаний связи С-N
заметно сдвинута в высокочастотную область по сравнению со спектром
свободного бензамида (60см\textsuperscript{-1}), что объясняется
упрочнением этой связи, происходящим в результате протонирования по
атому кислорода карбонильной группы амида.

Полоса поглощения деформационных колебаний δ(NH\textsubscript{2}) группы
обнаружены в спектре свободного бензамида при 1618
см\textsuperscript{-1}. В спектре соединения
2C\textsubscript{6}H\textsubscript{5}CONH\textsubscript{2}{\bfseries \textsuperscript{.}}H\textsubscript{2}SiF\textsubscript{6}
эта полоса сдвинута на 8 см\textsuperscript{-1} в область низких частот.

В спектрах дигидрогексафторосиликата бензамида полоса валентных
колебаний связи ν(CN) сдвинута на 60 см\textsuperscript{-1} в
высокочастотную область, что свидетельствует об упрочнении связи С-N в
этих соединениях. Полосы валентных симметричных колебаний
ν\textsubscript{s}(NH) в колебательных спектрах соединений бензамида с
гексафторокремниевой кислотой смещены в низкочастотную областьна 70
см\textsuperscript{-1}, что указывает на участие аминогруппы амида в
образовании новых водородных связей.
\end{multicols}

\begin{table}[H]
\caption*{Таблица 6 - Значения характеристических частот (см\textsuperscript{-1}) в ИК спектрах поглощения соединений тиокарбамида, ацетилтиокарбамида с гексафторокремниевой кислотой}
\centering
\resizebox{\textwidth}{!}{%
\begin{tabular}{|l|l|l|l|l|l|}
\hline
Отнесение & CS(NH\tsb{2})\tsb{2} & 2CS(NH\tsb{2})\tsb{2}\tsp{.}H\tsb{2}SiF\tsb{6} & Отнесение & CH\tsb{3}СОNHCSNH\tsb{2} & 2CH\tsb{3}СОNHCSNH\tsb{2}\tsp{.}H\tsb{2}SiF\tsb{6} \\ \hline
ν(NH) & \begin{tabular}[c]{@{}l@{}}3370\\ 3265\\ 3175\end{tabular} & \begin{tabular}[c]{@{}l@{}}3400\\ 3290\\ 3195\end{tabular} & ν(NH) & \begin{tabular}[c]{@{}l@{}}3430\\ 3320\\ 3280\\ 3190\end{tabular} & \begin{tabular}[c]{@{}l@{}}3330\\ 3250\\ 3195\\ 3100\end{tabular} \\ \hline
ν (SН) &  & 2380 & ν (SН) &  & 2555 \\ \hline
 &  &  & ν(CO)ацет. & 1720 & 1710 \\ \hline
δ (NH2) & 1620 & 1623 & δ (NH2) & 1690 & 1690 \\ \hline
ν(CN) & 1410 & \begin{tabular}[c]{@{}l@{}}1480\\  1420\end{tabular} & νa (N-C-N) & 1600 & 1615 \\ \hline
\multirow{2}{*}{ν(CS) + ν(CC)} & \multirow{2}{*}{732} & \multirow{2}{*}{735} & ν(CN) & \begin{tabular}[c]{@{}l@{}}1545\\  1410\end{tabular} & \begin{tabular}[c]{@{}l@{}}1580\\  1420\end{tabular} \\ \cline{4-6}
 &  &  & ν S(N-C-N) & 1050 & 1080 \\ \hline
 &  &  & ν(CS) +r(NH2) & 740 & 720 \\ \hline
\end{tabular}%
}
\end{table}

\begin{multicols}{2}
В таблице 6 приведены характеристические частоты поглощения отдельных
функциональных групп тиоамидов и синтезированных соединений тиоамидов с
гексафторокремниевой кислотой. В спектре тиокарбамида в области
валентных колебаний связей обнаружены три полосы поглощения (3175-3370
см\textsuperscript{-1}). Высокочастотная компонента отнесена к валентным
антисимметричным колебаниям связей N-H, а две другие частоты обусловлены
валентными симметричными колебаниями этих связей. В спектре соединения
дигидрогексафторосиликата дитиокарбамида
2CS(NH\textsubscript{2})\textsubscript{2}{\bfseries \textsuperscript{.}}H\textsubscript{2}SiF\textsubscript{6}
высокочастотная компонента (3370см\textsuperscript{-1}) сместилась в
область высоких частот (3400см\textsuperscript{-1}). В спектре
ацетилтиокарбамида в области валентных колебаний связей N-H наблюдается
значительно большее число полос (3430, 3320, 3280, 3190
см\textsuperscript{-1}), что связано с неравноценностью этих связей
амида из-за наличия у одного из атомов азота ацетильного заместителя. В
спектре соединения дигидрогексафторосиликата диацетилтиокарбамида
2CH\textsubscript{3}СОNHCSNH\textsubscript{2}{\bfseries \textsuperscript{.}}H\textsubscript{2}SiF\textsubscript{6}
происходит низкочастотное смещение полос валентных колебаний связей
N-H(3330, 3250, 3195, 3100 см\textsuperscript{-1}). Это вызвано участием
аминогруппы в образовании новыхводородных связей с кислотным остатком
NH\ldots An\textsuperscript{-} (где An\textsuperscript{-} - анион
кислоты).

Деформационные колебания группы NH\textsubscript{2} в спектрах
соединений тиокарбамида и ацетилтиокарбамида практически сохраняют свое
положение. В спектрах соединений тиоамидов с гексафторокремниевой
кислотой обнаружено дополнительное поглощение в области
2700--2400см\textsuperscript{-1}, которое вызвано валентными колебаниями
S-Н связей. Появление этих связей указывает на протонирование по атому
серы. В пользу S -- протонирования говорит смещение в высокочастотную
область полосы поглощения валентных колебаний связей С-N.

В случае образования связи между протоном кислоты или ионом меди (II) с
атомом кислорода карбонильной группы в спектрах соединений полоса
поглощения связиС=О должна понизиться из-за снижения ее кратности. При
этом частота N-H связей не должна претерпевать особых изменений.

Проведение анализа ИК-спектров соединений солей меди (II) с
протонированным карбамидом (таблица 7) осложнено тем, что карбамид
соединяется с катионом металла уже в протонированном через атом
кислорода состоянии. Частоты валентных антисимметричных колебаний связи
ν\textsubscript{a}(NH) соединений солей меди (II) с протонированным
карбамидом на 10--15 см\textsuperscript{-1} смещаются в низкочастотную
область. Полосы валентных симметричных колебаний N-H связи в меньшей
степени смещены в низкочастотную область (на 5-10
см\textsuperscript{-1}) в результате образования новых водородных связей
между аминогруппой и анионом кислоты. Частоты валентных колебаний
карбонильной связи ν(CO)в спектрах соединений никеля с протонированным
карбамидом смещены в низкочастотную область на 25 - 30
см\textsuperscript{-1}.
\end{multicols}

\begin{table}[H]
\caption*{Таблица 7 - Значения характеристических частот(см\textsuperscript{-1}) в ИК спектрах поглощения соединений протонированного карбамида с солями меди (II)}
\centering
\resizebox{\textwidth}{!}{%
\begin{tabular}{|l|l|l|l|l|l|}
\hline
Отнесение & \textbf{CO(NH\tsb{2})\tsb{2}} & \textbf{CuCl\tsb{2}\tsp{.}4CO(NH\tsb{2})\tsb{2}\tsp{.}HCl} & \textbf{CuCl\tsb{2}\tsp{.}2CO(NH\tsb{2})\tsb{2}\tsp{.}2HCl} & \textbf{Cu(NO\tsb{3})\tsb{2}\tsp{.}4CO(NH\tsb{2})\tsb{2}HNO\tsb{3}} & \textbf{Cu(NO\tsb{3})\tsb{2}\tsp{.}2CO(NH\tsb{2})\tsb{2}HNO\tsb{3}} \\ \hline
ν (OH) &  & 3385 & 3380 & 3380 & 3385 \\ \hline
νa(NH) & \begin{tabular}[c]{@{}l@{}}3420\\  3345\end{tabular} & \begin{tabular}[c]{@{}l@{}}3400\\  3330\end{tabular} & \begin{tabular}[c]{@{}l@{}}3405\\  3335\end{tabular} & \begin{tabular}[c]{@{}l@{}}3405\\  3330\end{tabular} & \begin{tabular}[c]{@{}l@{}}3400\\  3325\end{tabular} \\ \hline
νs(NH) & 3240 & 3235 & 3225 & 3220 & 3230 \\ \hline
ν (CO) & 1655 & 1620 & 1625 & 1625 & 1620 \\ \hline
ν(NH2) & 1610 & 1615 & 1610 & 1610 & 1615 \\ \hline
νa(CN) & 1450 & 1470 & 1475 & 1470 & 1470 \\ \hline
νs(CN) & 1065 & 1085 & 1090 & 1080 & 1080 \\ \hline
δ (OH) & - & 1710 & 1705 & 1710 & 1715 \\ \hline
\end{tabular}%
}
\end{table}

\begin{multicols}{2}
Частоты валентных колебаний связи С-N в спектрах соединений смещены в
высокочастотную область на 20 - 30 см\textsuperscript{-1}, что
свидетельствует об образовании связи через карбонильный кислород.
Появление в ИК-спектрах соединений полос поглощения ν(ОH) в области
3380- 3390 см\textsuperscript{-1} и δ(OH) при 1705 - 1715
см\textsuperscript{-1} также указывает на О -- протонирование. Полосы
поглощения деформационных колебаний δ(NH\textsubscript{2}) при 1610
см\textsuperscript{-1} практически сохраняют свое положение, что также
может быть результатом участия связи С=О в образовании координационного
соединения. Соединения солей меди (II) с протонированным карбамидом
относятся к разнолигандным координационным соединениям, содержащими во
внутренней сфере карбамид и протонированный карбамид наряду с анионами
соответствующих кислот. Протонированный карбамид связан с ионом меди
(II) через атом азота аминной группы.

{\bfseries Выводы.} В работе приведены ряд данных по квантово-химическому
расчету и представлены результаты ИК-спектроскопического анализа
соединений амидов и тиоамидов с соляной и гексафторокремниевой кислотами
и солями меди (II).

Соединения солей меди с протонированным карбамидом содержат во
внутренней сфере одновременно молекулы карбамида и протонированного
карбамида, причем первые связаны с комплексообразователем через атом
кислорода карбонильной группы, а вторые - через атом азота аминной
группы.

Данные расчеты комплексов меди (II)с амидами подтверждает возможность
образования нескольких соединений одной и той же соли металла с
протонированным карбамидом, что указывает на их разнообразие.
Определяющим фактором для образования соединения с определенным
количеством лигандов является природа металла-комплексообразователя.
Структурные исследования координационных соединений методом
квантово-химического расчета дают возможность утверждать, что более
энергетически выгодным для меди (II) является образование комплексов с
координационным числом шесть, геометрия -- искаженный октаэдр. На состав
и структуру координационных соединений значительное влияние оказывает
протонирование карбамида. Координационные соединения меди (II) содержат
во внутренней сфере одновременно молекулы карбамида и протонированного
карбамида, причем первые связаны с комплексообразователем через атом
кислорода карбонильной группы, а вторые -- через атом азота амидной
группы. ИК спектроскопическое исследование показало, что упрочнение
связи С-N и ослабление связи С=О в соединениях амидов и тиоамидов с
соляной и гексафторокремниевой кислотами свидетельствует, что центром
протонирования при кислотно-основном взаимодействии в системе амид-
кислота- вода является атом кислорода карбонильной группы амида.
Низкочастотный сдвиг полос поглощения валентных колебаний связей N-H
объясняется участием азота в водородных связях с кислотным остатком.

Для некоторых соединений амидов и тиоамидов с гексафторокремниевой
кислотой найдены возможные области практического применения: они могут
быть рекомендованы к использованию в качестве инсектицидов против
гусениц серой зерновой совки, антисептиков при консервировании шкур, в
качестве добавок в сырьевую смесь для изготовления стеновых керамических
изделий с целью увеличения их прочности {[}14{]}.
\end{multicols}

\begin{center}
{\bfseries Литература}
\end{center}

\begin{noparindent}
1. Исмаилова Ч.Ш. Клатратно-координационные соединения марганца,
кобальта и цинка с карбамидом (синтез, структура, свойства): автореферат
дис. кандидата химических наук: 02.00.01 и 02.00.04 /Институт химии и
химической технологии Национальной академии наук КР. - Бишкек, 2009. -24
с.

2. Замилацков И. А. Координационные соединения иодидов цинка и кадмия с
амидами: автореферат дис. кандидата химических наук: 02.00.01/
Московская Государственная Академия тонкой химической технологии им М.В.
Ломоносова. -Москва, 2007.-28 с.

3. Джуманазарова З.К., Калмуратова Ш.Т., Бекимбетова Г.Н., Наурызбаева
Т.О. Комплексные соединения нитрата кальция с двумя амидами//
Международный Научный журнал Universum. Серия: Технические науки,
2022.-№ 4(97). - С.38-43.

DOI 10.32743/UniTech.2022.97.4.13545

4. Спектроскопия координационных соединений: Сборник научных трудов XVII
Международной

конференции.- Краснодар, 2020.- 420 с.

http://www.spsl.nsc.ru/FullText/konfe/SpScKS\_2020.pdf

5. Касимов Ш.А., Тураев Х.Х., Джалилов А.Т., Чориева Н.Б., Амонова Н.Д.
ИК спектроскопические исследование и квантово-химические
характеристикиазот и фосфорсодержащего полимерного лиганда//
Международный Научный журнал Universum. Серия: Химия и биология, 2019. -
№ 6 (60)- С.50-53.

https://7universum.com/ru/nature/archive/item/7400

6.Еркасов Р.Ш., Кусепова Л.А., Масакбаева С.Р., Байсалова Г.Ж.,
Болысбекова С.М. Взаимодействие в системе сульфат кобальта -карбамид
-серная кислота -вода при 25 °С// Вестник Евразийского национального
университета имени Л.Н. Гумилева. Серия: Химия. География. Экология,
2017.- № 4 (119). - С.207 - 212.

7.Еркасов Р.Ш., Кусепова Л.А., Байсалова Г.Ж., Масакбаева С.Р.
Взаимодействие в системе нитрат никеля - карбамид - азотная кислота -
вода при 25 °С// Вестник Евразийского национального университета имени
Л.Н. Гумилева. Серия: Химия. География. Экология, 2019.- № 3 (128). -
С.33 - 42.

8.Erkasov R.S., Massakbayeva S.R., Kusepova L.A.,Bolysbekova S.M.
Interaction in the Nickel Perchlorate-Acetamide-Perchloric Acid-Water
System at 25°C// Russian journal of inorganic chemistry, 2017.-Vol.62
(9) - P.1234-1239.

9.Посыпайко В.И., Козырева Н.А., Логачева Ю.П. Химические методы анализа
/ Посыпайко В.И. и др. - Москва: Высшая школа, 1989. - 447 с.

10.Соловьёв М.Е., Соловьёв М.М. Компьютерная химия
/М.Е.Соловьёв.--Москва:СОЛОН-Пресс, 2005.-536 с.

11.Вилков Л.В., Пентин Ю.А. Физические методы исследования в химии.
Структурные методы и оптическая спектроскопия / Вилков Л.В., Пентин Ю.А.
- Москва: Высшая школа, 1987. - 366 с

12.Еркасов Р.Ш., Кусепова Л.А., Байсалова Г.Ж. Квантово-химические
характеристики координационных соединений хлорида меди с протонированным
карбамидом // Перспективы развития науки и образования: тезисы
международной научно-практической конференции. Вестник научных
конференций, 2018. - № 6 (34). - С.62 - 65.

13.Губин А.И., Буранбаев М.Ж., Нурахметов Н.Н., Ташенов А.К., Суюндикова
Ф.О. Кристаллическая и молекулярная структура карбамида с
гексафторокремниевой кислотой состава 2:1 // Кристаллография, 1988. -
Т.33(2) - С.509-510.

14.Сайбулатов С.Ж., Бацко Р.С., Нурахметов Н.Н., Суюндикова Ф.О.,
Ташенов А.К. Сырьевая смесь для изготовления стеновых керамических
изделий. А.C.СССР. №1353757 от 22.07.1987 г.
\end{noparindent}

\begin{center}
{\bfseries References}
\end{center}

\begin{noparindent}
1.Ismailova Ch.Sh. Klatratno-koordinacionnye soedinenija marganca,
kobal\textquotesingle ta i cinka s karbamidom (sintez, struktura,
svojstva): avtoreferat dis. kandidata himicheskih nauk: 02.00.01 i
02.00.04/ Institut himii i himicheskoj tehnologii
Nacional\textquotesingle noj akademii nauk KR. -- Bishkek, 2009. - 24
s.{[}in Russ.{]}

2.Zamilackov I. A. Koordinacionnye soedinenija iodidov cinka i kadmija s
amidami: avtoreferat dis. kandidata himicheskih nauk: 02.00.01/
Moskovskaja Gosudarstvennaja Akademija tonkoj himicheskoj tehnologii im
M.V. Lomonosova. - Moskva, 2007. -28 s.{[}in Russ.{]}

3.Dzhumanazarova Z.K., Kalmuratova Sh.T., Bekimbetova G.N., Nauryzbaeva
T.O. Kompleksnye soedinenija nitrata kal\textquotesingle cija s dvumja
amidami// Mezhdunarodnyj Nauchnyj zhurnal Universum. Serija:
Tehnicheskie nauki, 2022.-№ 4(97). - S.38-43.{[}in Russ.{]}

DOI 10.32743/UniTech.2022.97.4.13545

4.Spektroskopija koordinacionnyh soedinenij: Sbornik nauchnyh trudov
XVII Mezhdunarodnoj konferencii.- Krasnodar, 2020.- 420 s. {[}in
Russ.{]}

http://www.spsl.nsc.ru/FullText/konfe/SpScKS\_2020.pdf

5.Kasimov Sh.A., Turaev H.H., Dzhalilov A.T., Chorieva N.B., Amonova
N.D. IK spektroskopicheskie issledovanie i kvantovo-himicheskie
harakteristikiazot i fosforsoderzhashhego polimernogo liganda//
Mezhdunarodnyj Nauchnyj zhurnal Universum. Serija: Himija i biologija
2019. - № 6 (60) S.50-53. {[}in Russ.{]}

https://7universum.com/ru/nature/archive/item/7400

6.Erkasov R.Sh., Kusepova L.A., Masakbaeva S.R., Bajsalova G.Zh.,
Bolysbekova S.M. Vzaimodejstvie v sisteme sul\textquotesingle fat
kobal\textquotesingle ta -- karbamid -- sernaja kislota -- voda pri 25
°S// Vestnik Evrazijskogo nacional\textquotesingle nogo universiteta
imeni L.N. Gumileva. Serija: Himija. Geografija. Jekologija, 2017.- № 4
(119). - S.207 - 212. {[}in Russ.{]}

7.Erkasov R.Sh., Kusepova L.A., Bajsalova G.Zh., Masakbaeva S.R.
Vzaimodejstvie v sisteme nitrat nikelja -- karbamid -- azotnaja kislota
-- voda pri 25 °S// Vestnik Evrazijskogo nacional\textquotesingle nogo
universiteta imeni L.N. Gumileva. Serija: Himija. Geografija. Jekologija
2019.- № 3 (128). - S.33 - 42. {[}in Russ.{]}

8.Erkasov R.S., Massakbayeva S.R., Kusepova L.A.,Bolysbekova S.M.
Interaction in the Nickel Perchlorate-Acetamide-Perchloric Acid-Water
System at 25°C// Russian journal of inorganic chemistry, 2017.--Vol.62
(9)-. P.1234-1239.

9.Posypajko V.I., Kozyreva N.A., Logacheva Ju.P. Himicheskie metody
analiza / Posypajko V.I. i dr. - Moskva: Vysshaja shkola, 1989. - 447 s.
{[}in Russ.{]}

10.Solov\textquotesingle jov M.E., Solov\textquotesingle jov M.M.
Komp\textquotesingle juternaja himija
/M.E.Solov\textquotesingle jov.--Moskva:SOLON-Press, 2005.-536 s. {[}in
Russ.{]}

11.Vilkov L.V., Pentin Ju.A. Fizicheskie metody issledovanija v himii.
Strukturnye metody i opticheskaja

spektroskopija / Vilkov L.V., Pentin
Ju.A. - Moskva: Vysshaja shkola, 1987. - 366 s. {[}in Russ.{]}

12.Erkasov R.Sh., Kusepova L.A., Bajsalova G.Zh. Kvantovo-himicheskie
harakteristiki koordinacionnyh

soedinenij hlorida medi s protonirovannym
karbamidom // Perspektivy razvitija nauki i obrazovanija: tezisy
mezhdunarodnoj nauchno-prakticheskoj konferencii.-Vestnik nauchnyh
konferencij, 2018. - № 6 (34). - S.62 - 65. {[}in Russ.{]}

13.Gubin A.I., Buranbaev M.Zh., Nurahmetov N.N., Tashenov A.K.,
Sujundikova F.O. Kristallicheskaja i

molekuljarnaja struktura karbamida
s geksaftorokremnievoj kislotoj sostava 2:1 // Kristallografija, 1988. -
T.33.- Vyp.2. - S.509-510. {[}in Russ.{]}

14.Sajbulatov S.Zh., Backo R.S., Nurahmetov N.N., Sujundikova F.O.,
Tashenov A.K. Syr\textquotesingle evaja smes\textquotesingle{} dlja
izgotovlenija stenovyh keramicheskih izdelij. A.C.SSSR. №1353757 ot
22.07.1987 g. {[}in Russ.{]}
\end{noparindent}

\emph{Сведение об авторах}

\begin{noparindent}
Кусепова Л.А. - кандидат химических наук, доцент кафедры химия,
Евразийский национальный университет им. Л.Н. Гумилева, Астана,
Казахстан, e-mail: kusepova71@mail.ru;

Суюндикова Ф.О. - кандидат химических наук, доценткафедры химия,
Евразийский национальный университет им. Л.Н. Гумилева, Астана,
Казахстан, e-mail: sfaiziya@mail.ru
\end{noparindent}

{\bfseries Information about the authors}

\begin{noparindent}
Kusepova L.А. - Candidate of Chemical Sciences, Associate Professor of
Department Chemistry, L.N. Gumilyov Eurasian National University,
Astana, Kazakhstan, e-mail: kusepova71@mail.ru;

Suyndikova F.O{\bfseries .} - Candidate of Chemical Sciences, Associate
Professor of Department Chemistry, L.N. Gumilyov Eurasian National
University, Astana, Kazakhstan, e-mail: sfaiziya@mail.ru
\end{noparindent}
