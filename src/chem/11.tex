\newpage
{\bfseries IRSTI 61.31.51}
\hfill {\bfseries \href{https://doi.org/10.58805/kazutb.v.2.23-352}{https://doi.org/10.58805/kazutb.v.2.23-352}}

\sectionwithauthors{V.I Romanov, V.V. Merkulov, S.K. Kabiyeva, E.S. Bestembek, R.K. Zhaslan, G.M. Zhumanazarova}{INVESTIGATION OF THE CHEMICAL AND MINERALOGICAL COMPOSITION OF
METALLURGICAL SLAGS OF JSC ``QARMET'' TEMIRTAU}

\begin{center}
{\bfseries V.I Romanov, V.V. Merkulov, S.K. Kabiyeva\envelope, E.S. Bestembek, R.K. Zhaslan, G.M. Zhumanazarova}

NLC «Karaganda Industrial University», Temirtau, Kazakhstan

\envelope Corresponding author: kabieva.s@mail.ru
\end{center}

«Qarmet» JSC is one of the largest metallurgical enterprises in
Kazakhstan, producing steel and other metals. Study of chemical and
mineralogical composition of metallurgical slags from this enterprise
can be crucial for its process optimization, environmental compliance
improvement and efficient waste management.

Study of chemical and mineralogical composition of metallurgical slags
from «Qarmet» JSC emphasizes importance of researching metal production
waste for its efficient recycling. Further research in this area may
facilitate development of new waste recycling technologies and
improvement of metallurgical production sustainability.

During the study of the chemical composition of metallurgical slags of
Qarmet JSC, it was found that they contain a significant amount of metal
oxides such as iron, manganese, silicon and others. These elements can
be potentially useful for reuse in other manufacturing processes or for
the production of building materials.

In addition, mineralogical analysis has shown that metallurgical slags
contain various mineral phases such as silicates, oxides and other
compounds. This indicates the complex structure of the slags and the
possibility of using them as additives to cement or other building
materials.

{\bfseries Keywords:} metallurgical blast furnace slag, properties,
chemical composition, mineralogical composition, slag activity, building
materials, wastes, oxides.

\begin{center}
{\large\bfseries ТЕМІРТАУ ҚАЛАСЫНЫҢ «QARMET» АҚ МЕТАЛЛУРГИЯЛЫҚ ҚОЖДАРЫНЫҢ
ХИМИЯЛЫҚ ЖӘНЕ МИНЕРАЛОГИЯЛЫҚ ҚҰРАМЫН ЗЕРТТЕУ}

{\bfseries В.В. Романов, В.В. Меркулов, С.К. Кабиева\envelope, Е.С. Бестембек, Р.Қ. Жаслан, Г.М. Жуманазарова}

«Қарағанды индустриялық универсиеті» КеАҚ, Теміртау, Қазақстан,

e-mail: kabieva.s@mail.ru
\end{center}

Қазақстандағы ең ірі металлургиялық кәсіпорындардың бірі Болат және
басқа металдар өндірумен айналысатын «Qarmet» АҚ болып табылады. Осы
кәсіпорынның металлургиялық шлактарының химиялық және минералогиялық
құрамын зерттеу өндірістік процестерді оңтайландыру, экологиялық
қауіпсіздікті жақсарту және қалдықтарды тиімді пайдалану үшін маңызды
болуы мүмкін.

«Qarmet» АҚ металлургиялық қождардың химиялық және минералогиялық
құрамын зерттеу металдар өндірісінің қалдықтарын оларды тиімді басқару
және қайта өңдеу мақсатында зерделеудің маңыздылығын атап көрсетеді. Осы
саладағы қосымша зерттеулер қалдықтарды қайта өңдеудің жаңа
технологияларын дамытуға және металлургия өндірісінің тұрақтылығын
арттыруға ықпал етуі мүмкін.

"Qarmet" АҚ металлургиялық қождардың химиялық құрамын зерттеу барысында
олардың құрамында темір, марганец, кремний және басқалары сияқты металл
оксидтерінің едәуір мөлшері бар екендігі анықталды. Бұл элементтер басқа
өндірістік процестерде немесе құрылыс материалдарын өндіруде қайта
пайдалану үшін пайдалы болуы мүмкін.

Сонымен қатар, минералогиялық талдау металлургиялық шлактарда
силикаттар, оксидтер және басқа қосылыстар сияқты әртүрлі минералды
фазалар бар екенін көрсетті. Бұл токсиндердің күрделі құрылымын және
оларды цемент немесе басқа құрылыс материалдарына қоспалар ретінде
пайдалану мүмкіндігін көрсетеді.

{\bfseries Түйін сөздер:} металлургиялық домна пешінің қожы, қасиеттері,
химиялық құрамы, минералогиялық құрамы, қож белсенділігі, құрылыс
материалдары, қалдықтар, оксидтер.

\begin{center}
{\large\bfseries ИССЛЕДОВАНИЕ ХИМИЧЕСКОГО И МИНЕРАЛОГИЧЕСКОГО СОСТАВА
МЕТАЛЛУРГИЧЕСКИХ ШЛАКОВ АО «QARMET» Г.ТЕМИРТАУ}

{\bfseries В.В. Романов, В.В. Меркулов, С.К. Кабиева\envelope, Е.С. Бестембек, Р.Қ. Жаслан, Г.М. Жуманазарова}

НАО «Карагандинский индустриальный университет», Темиртау, Казахстан,

e-mail: kabieva.s@mail.ru
\end{center}

Одним из крупнейших металлургических предприятий в Казахстане является
АО «Qarmet», которое занимается производством стали и других металлов.
Исследование химического и минералогического состава металлургических
шлаков этого предприятия может быть ключевым для оптимизации
производственных процессов, улучшения экологической безопасности и
эффективного использования отходов.

Исследование химического и минералогического состава металлургических
шлаков АО «Qarmet» подчеркивает важность изучения отходов производства
металлов с целью их эффективного управления и переработки. Дальнейшие
исследования в этой области могут способствовать разработке новых
технологий переработки отходов и повышению устойчивости
металлургического производства.

В ходе исследования химического состава металлургических шлаков АО
«Qarmet» было обнаружено, что они содержат значительное количество
оксидов металлов, таких как железо, марганец, кремний и другие. Эти
элементы могут быть потенциально полезными для повторного использования
в других производственных процессах или для производства строительных
материалов.

Кроме того, минералогический анализ показал, что металлургические шлаки
содержат различные минеральные фазы, такие как силикаты, оксиды и другие
соединения. Это свидетельствует о сложной структуре шлаков и возможности
использования их в качестве добавок к цементу или другим строительным
материалам.

{\bfseries Ключевые слова:} металлургический доменный шлак, свойства,
химический состав, минералогический состав, активность шлака,
строительные материалы, отходы, оксидтер.

\begin{multicols}{2}
{\bfseries Introduction.} Metallurgical slag is one of the main wastes of
iron and steel production. They are formed as a result of the melting of
agglomerates, fluxes and other additives during the processing process
to obtain the main product - cast iron {[}1-2{]}. Chemical and
mineralogical composition of metallurgical slags may vary significantly
depending on initial materials\textquotesingle{} composition, production
technology and other factors study {[}3-4{]}.

It is widely known that one of specific features of metallurgical slags
is their activity - ability to display hydraulic properties when
interacting with water, similarly to cement. This study shows results of
researching the process of obtaining non-clinker binder based on
granulated blast furnace slag from Qarmet JSC.

Global concrete producers widely use metallurgical slag as cement
replacement. The substitution of cement by slag provides two clear
advantages; the first one is use of a waste that otherwise must be
managed in a landfill, and the second one, even more relevant, is
reduction in cement consumption, so the reduction of CO\textsubscript{2}
emissions during its production. The authors of study {[}5{]} used
metallurgical slags from a plant in Spain.

Besides, multiple studies have analyzed general tendencies of slag
chemical composition which show that composition of different slags may
vary depending on either place of production or year of steel production
{[}6{]}.

Considering the abovementioned, there is high importance of finding
mineralogical and chemical composition of metallurgical slag produced by
Karaganda metallurgical plant Qarmet JSC. Temirtau, Kazakhstan. This
issue is addressed in this study.

{\bfseries Materials and methods.} To conduct the study, granulated blast
furnace slag of JSC Qarmet was used. Slag samples were taken from
various points in the storage area, and the samples were averaged.

Slag activity is determined by its chemical composition including up to
30 elements, primarily CaO, MgO, SiO\textsubscript{2},
Al\textsubscript{2}O\textsubscript{3}, FeO, MnO, and their mineralogical
composition {[}7-10{]}. The most used in binder materials production are
the slags with sufficient hydraulic activity characterized by basicity
module M\textsubscript{b} and activity module M\textsubscript{a},
containing large amount of glass of helenite-melelite, wollastonite and
aluminosilicate composition {[}11-14{]}.

To determine the chemical composition of the slag, the content of oxides
CaO, MgO, МnО, Al\textsubscript{2}O\textsubscript{3} was analyzed
according to SS 5382-2019. {[}15{]}

The most acceptable chemical composition for this article is the slag
presented in table 1.
\end{multicols}

\begin{table}[H]
\caption*{Table 1 - Chemical composition of active slags}
\centering
\begin{tabular}{|l|l|l|l|l|}
\hline
СаО & S & МnО & Аl2О3 & МgO \\ \hline
More than 40\% & No more than 4-5\% & Less than 2\% & No less than 9 \% & No less than 4-10\% \\ \hline
\end{tabular}
\end{table}

\begin{table}[H]
\caption*{Table 2 - Chemical and mineralogical composition of metallurgical slags}
\centering
\resizebox{\textwidth}{!}{%
\begin{tabular}{|llllllllllp{0.13\textwidth}|}
\hline
\multicolumn{1}{|l|}{\multirow{2}{*}{Year of slag production}} & \multicolumn{7}{l|}{Content, \% by mass} & \multicolumn{1}{p{0.13\textwidth}|}{\multirow{2}{=}{Basicity module, Мb}} & \multicolumn{1}{p{0.13\textwidth}|}{\multirow{2}{=}{Activity module, Ма}} & \multirow{2}{=}{Mineralogical composition} \\ \cline{2-8}
\multicolumn{1}{|l|}{} & \multicolumn{1}{l|}{SiO2} & \multicolumn{1}{l|}{AL2O3} & \multicolumn{1}{l|}{CaO} & \multicolumn{1}{l|}{MgO} & \multicolumn{1}{l|}{MnO} & \multicolumn{1}{l|}{FeO} & \multicolumn{1}{l|}{S} & \multicolumn{1}{l|}{} & \multicolumn{1}{l|}{} &  \\ \hline
\multicolumn{11}{|l|}{Converter slag} \\ \hline
\multicolumn{1}{|l|}{2020} & \multicolumn{1}{l|}{9,103} & \multicolumn{1}{l|}{1,612} & \multicolumn{1}{l|}{42,542} & \multicolumn{1}{l|}{8,173} & \multicolumn{1}{l|}{3,424} & \multicolumn{1}{l|}{19,103} & \multicolumn{1}{l|}{0,122} & \multicolumn{1}{l|}{4,733} & \multicolumn{1}{l|}{0,177} & \multirow{6}{*}{} \\ \cline{1-10}
\multicolumn{1}{|l|}{2021} & \multicolumn{1}{l|}{9,61} & \multicolumn{1}{l|}{1,6} & \multicolumn{1}{l|}{42,94} & \multicolumn{1}{l|}{7,71} & \multicolumn{1}{l|}{2,96} & \multicolumn{1}{l|}{19,36} & \multicolumn{1}{l|}{0,13} & \multicolumn{1}{l|}{4,518} & \multicolumn{1}{l|}{0,166} &  \\ \cline{1-10}
\multicolumn{1}{|l|}{2022} & \multicolumn{1}{l|}{9,65} & \multicolumn{1}{l|}{1,34} & \multicolumn{1}{l|}{41,34} & \multicolumn{1}{l|}{7,18} & \multicolumn{1}{l|}{2,54} & \multicolumn{1}{l|}{26,38} & \multicolumn{1}{l|}{0,12} & \multicolumn{1}{l|}{4,415} & \multicolumn{1}{l|}{0,139} &  \\ \cline{1-10}
\multicolumn{1}{|l|}{Average value} & \multicolumn{1}{l|}{10,52} & \multicolumn{1}{l|}{1,39} & \multicolumn{1}{l|}{42,56} & \multicolumn{1}{l|}{7,32} & \multicolumn{1}{l|}{3,98} & \multicolumn{1}{l|}{19,33} & \multicolumn{1}{l|}{0,12} & \multicolumn{1}{l|}{4,21} & \multicolumn{1}{l|}{0,13} &  \\ \cline{1-10}
\multicolumn{1}{|l|}{Root mean square deviation} & \multicolumn{1}{l|}{1,05} & \multicolumn{1}{l|}{0,15} & \multicolumn{1}{l|}{1,04} & \multicolumn{1}{l|}{0,98} & \multicolumn{1}{l|}{1,32} & \multicolumn{1}{l|}{3,24} & \multicolumn{1}{l|}{0,02} & \multicolumn{1}{l|}{0,35} & \multicolumn{1}{l|}{0,03} &  \\ \cline{1-10}
\multicolumn{1}{|l|}{Variation coefficient, \%} & \multicolumn{1}{l|}{10} & \multicolumn{1}{l|}{10,55} & \multicolumn{1}{l|}{2,45} & \multicolumn{1}{l|}{13,32} & \multicolumn{1}{l|}{33,27} & \multicolumn{1}{l|}{16,79} & \multicolumn{1}{l|}{19,14} & \multicolumn{1}{l|}{8,21} & \multicolumn{1}{l|}{19,4} &  \\ \hline
\multicolumn{11}{|l|}{Blast furnace slag} \\ \hline
\multicolumn{1}{|l|}{2020} & \multicolumn{1}{l|}{36,99} & \multicolumn{1}{l|}{13,05} & \multicolumn{1}{l|}{39,52} & \multicolumn{1}{l|}{9,46} & \multicolumn{1}{l|}{0,56} & \multicolumn{1}{l|}{0,44} & \multicolumn{1}{l|}{1} & \multicolumn{1}{l|}{0,979} & \multicolumn{1}{l|}{0,353} & \multirow{6}{=}{Helenite-melilite glass, helenite, wollastonite} \\ \cline{1-10}
\multicolumn{1}{|l|}{2021} & \multicolumn{1}{l|}{36,63} & \multicolumn{1}{l|}{13,81} & \multicolumn{1}{l|}{39,72} & \multicolumn{1}{l|}{9,42} & \multicolumn{1}{l|}{0,39} & \multicolumn{1}{l|}{0,42} & \multicolumn{1}{l|}{0,98} & \multicolumn{1}{l|}{0,974} & \multicolumn{1}{l|}{0,377} &  \\ \cline{1-10}
\multicolumn{1}{|l|}{2022} & \multicolumn{1}{l|}{35,31} & \multicolumn{1}{l|}{14,86} & \multicolumn{1}{l|}{38,67} & \multicolumn{1}{l|}{10} & \multicolumn{1}{l|}{0,5} & \multicolumn{1}{l|}{0,51} & \multicolumn{1}{l|}{0,99} & \multicolumn{1}{l|}{0,97} & \multicolumn{1}{l|}{0,421} &  \\ \cline{1-10}
\multicolumn{1}{|l|}{Average value} & \multicolumn{1}{l|}{36,31} & \multicolumn{1}{l|}{13,91} & \multicolumn{1}{l|}{39,3} & \multicolumn{1}{l|}{9,63} & \multicolumn{1}{l|}{0,48} & \multicolumn{1}{l|}{0,46} & \multicolumn{1}{l|}{0,99} & \multicolumn{1}{l|}{0,97} & \multicolumn{1}{l|}{0,38} &  \\ \cline{1-10}
\multicolumn{1}{|l|}{Root mean square deviation} & \multicolumn{1}{l|}{0,72} & \multicolumn{1}{l|}{0,74} & \multicolumn{1}{l|}{0,46} & \multicolumn{1}{l|}{0,26} & \multicolumn{1}{l|}{0,07} & \multicolumn{1}{l|}{0,04} & \multicolumn{1}{l|}{0,01} & \multicolumn{1}{l|}{0,004} & \multicolumn{1}{l|}{0,03} &  \\ \cline{1-10}
\multicolumn{1}{|l|}{Variation coefficient, \%} & \multicolumn{1}{l|}{1,99} & \multicolumn{1}{l|}{5,34} & \multicolumn{1}{l|}{1,16} & \multicolumn{1}{l|}{2,75} & \multicolumn{1}{l|}{14,56} & \multicolumn{1}{l|}{8,45} & \multicolumn{1}{l|}{0,82} & \multicolumn{1}{l|}{0,37} & \multicolumn{1}{l|}{7,34} &  \\ \hline
\end{tabular}
}
\end{table}

\begin{multicols}{2}
Basic slags (basicity module more than 1) display hydraulic activity at
high alumina content and manganese oxide no more than 5\%, when there is
a shortage of raw materials. Acidic slags (basicity module less than 1)
show sufficient activity at basicity module no less than 0,65 and
activity module no less than 0,33 at manganese oxide no more than 4\%.

In order to reveal and facilitate slags\textquotesingle{} hydraulic
properties, they should be mixed with alkaline-containing solidification
catalysts which saturate water solution during slag hydration with
Са\textsuperscript{2+}, ОН\textsuperscript{−}, and
SO\textsubscript{4}\textsuperscript{2−} ions, thus creating conditions
for alkaline and sulfate activation of slag glass. This process also
yields low-basicity calcium hydrosilicates calcium which are the main
product of granulated slag hydration and hydrolysis in presence of
alkaline catalyst. Low-basicity calcium hydrosilicates after complete
consolidation have hardness close to crystalhydrate newgrowths obtained
through cement hydration and hydrolysis, and are better than the latter
in deformation properties as hardness of bonds formed through
consolidation is lower than hardness of chrystallization contacts
through coalescence. The activity of the slag was determined according
to the SS 25094-2015 {[}16{]} method.

Ore from the Lisakovsky deposit was used as raw material. The methods
for testing the stability of blast furnace slag are based on standard
methods according to SS for lime, silicate, sulfide decomposition (SS
3476-2019, SS 5382-2019) {[}15, 17{]}

Another important parameter of slags influencing their use for making
construction materials is their disintegration property. There is
limestone, silicate and sulphidic disintegration. Slag structure is
considered to be resistant to limestone disintegration if its calcium
oxide content equals or less than critical value calculated using the
formula:

CaO ≤ 0,92SiO\textsubscript{2} + Al\textsubscript{2}O\textsubscript{3} +
0,2 MgO\textsubscript{2}\hfill (1)

Silicate disintegration appears due to the fact that during
crystallization slag oxides form dicalcium silicate
2CaOSiO\textsubscript{2}. This depends, firstly, on amount of lime, and
secondly, on initial slag temperature when it is cooled down quickly. In
case of absence of obvious connection between these two factors and
disintegration, for practical purposes it is considered that slags with
lime content over 45\% are prone to disintegration. Slags are resistant
at lime content under 45\%. However, positive impact of alumina presence
on resistance should be considered. At alumina content of approximately
18\% slag is resistant to disintegration even if CaO content is over
50\%. Magnesia presence also increases resistance. At MgO content
increase from 5 to 15\% resistance rises. Increase of structure
resistance in presence of alumina and magnesia is explained by chemical
reactions causing formation of helenite --
2CaOAl\textsubscript{2}O\textsubscript{3}SiO\textsubscript{2} and
okermanite -- 2CaOМgO\textsubscript{2}SiO\textsubscript{2}, which
contain large amounts of calcium oxide. This, in its turn, creates
conditions for decreasing amount of dicalcium silicate. It should be
noted that these reactions facilitate slags resistance increase against
limestone disintegration as well.

{\bfseries Results and discussion.} Sulphidic disintegration can be seen in
slags containing significant amount of iron or manganese sulphides. At
sulphides\textquotesingle{} interaction with water the substance volume
increase by up to 38\% occurs, which causes slag cracking and
destruction. At iron or manganese content over 2\% (expressed as FеО or
МnО) slag is considered unstable.

In order to estimate the opportunity of using metallurgical slags from
Qarmet JSC for non-clinker binders' production, their chemical and
mineralogical composition have been studied. Table 2 shows results of
analyzing slags produced in different years. The goal of the research
was to determine chemical-mineralogical stability of these
materials\textquotesingle{} properties and estimate their hydraulic
activity and structure\textquotesingle s resistance against
disintegration through computational methods.

Provided results of converter slag chemical-mineralogical definition
characterize it as basic ferriferrous slag with Мb=4,21;
FеО\textgreater5\% with minor activity module Ма=0,13, high content of
manganese oxide 3,98\%, unstable from year to year -- variation
coefficient 33,27 \%, and low content of
Al\textsubscript{2}O\textsubscript{3} 1,39\%, which classifies it as a
slag without prominent hydraulic activity, which is also proven by
mineralogical composition lacking active minerals. In terms of
mineralogical composition, converter slag mostly consists of periclase
and menganosite. It was researched that slags containing significant
amount of helenite-melilite, wollastonite and aluminosilicate glass are
the most suitable for binders' production.

Chemical composition of converter slag shows its tendency to various
types of disintegration. Estimative ability of converter slag to lime
disintegration is expressed in following:

СаО = 42,56

0,92SiO\textsubscript{2} + Al\textsubscript{2}O\textsubscript{3} + 0,2
MgO\textsubscript{2} = 12,53\hfill (2)

This way, content of calcium oxide significantly surpasses amount of
other main components combined.

Contents of iron oxide (19,33\%) and manganese (3,98\%) in converter
slag significantly surpass 2\% limit providing slag resistance against
sulphidic disintegration, and low content of alumina (1,39\%) at
relatively high share of СаО (42\%) shows its tendency to silicate
disintegration as well.

Table 3 shows chemical elements contained in blast furnace slag
providing its resistance against various disintegrations.
\end{multicols}

\begin{table}[H]
\caption*{Table 3 - Chemical elements in blast furnace slag providing its resistance against disintegrations for SS 3476-2019}
\centering
\begin{tabular}{|p{0.13\textwidth}|l|p{0.25\textwidth}|}
\hline
Type of disintegration & Main conditions of disintegration & Chemical elements in providing its resistance against disintegrations \\ \hline
Lime & CaO ≤ 0,92SiO\tsb{2} + Al\tsb{2}O\tsb{3} + 0,2 MgO\tsb{2} & 39,3 \% \textless 49,24 \% \\ \hline
Silicate & \begin{tabular}[c]{@{}l@{}}CaO\textgreater 45 \%; resistance rises at  \\  Al\tsb{2}O\tsb{3} up to 18 \% and MgO = 5 – 15 \%\end{tabular} & \begin{tabular}[c]{@{}l@{}}CaO = 39,3 \%\\  Al\tsb{2}O\tsb{3} = 13,9 \%\\  MgO = 9,63 \%\end{tabular} \\ \hline
Sulphid & \begin{tabular}[c]{@{}l@{}}FeO\textgreater 2 \%\\  MnO\textgreater 2 \%\end{tabular} & \begin{tabular}[c]{@{}l@{}}FeO = 0,46 \%\\  MnO = 0,48 \%\end{tabular} \\ \hline
\end{tabular}%
\end{table}

\begin{multicols}{2}
Considering the possibility of involving the composition of the charge
in order to increase the volume of production of a commercial product,
blast furnace production slag having a similar chemical composition,
with the exception of the increased content of ferrous oxide, a study
was carried out on the chemical and mineralogical components of
steelmaking blast furnace slag, which is reflected in table 3.

Blast furnace slag in terms of its chemical composition can be
classified as acidic magnesial slags with Мb=0,97\%; with МgО=9,63\%,
activity module Ма=0,38 and Al\textsubscript{2}O\textsubscript{3} and
MgO within limits suitable for slags with latent potential hydraulic
activity, which is proven by its mineralogical composition including
minerals prone to hydrolysis and hydration forming hydraulically active
compounds: helenite-melilite glass, helenite, wollastonite.

Chemical composition of blast furnace slag characterizes it as a
material which is not prone to various disintegration.

{\bfseries Conclusion.} During the study of the chemical composition of
blast furnace and steelmaking, in particular converter slag of Qarmet
JSC, it was discovered that they contain a significant amount of metal
oxides, such as iron, manganese, silicon and others. These elements
could be potentially useful for reuse in other industrial processes or
for the production of building materials for fertilizer, neutralization
of acidic soils and iron extraction.

In addition, mineralogical analysis has shown that metallurgical slags
contain various phase structures, such as silicates, oxides and other
compounds. This indicates the complex structure of slag with different
proportions of chemical compounds and the possibility of using them as
additives to cement or other building materials.

\emph{{\bfseries Financing.} The article was prepared based on the results
of scientific research within the framework of the state order of the
Ministry of Science and Higher Education for the implementation of the
grant fundamental scientific and technical project IRN: AP19678263 on
the topic «Rational use of man-made waste from metallurgical
industries».}

Авторы выражают благодарность за выделенное грантовое финансирование.
\end{multicols}

\begin{center}
{\bfseries References}
\end{center}

\begin{noparindent}
1.Piatak N. M. Environmental characteristics and utilization potential
of metallurgical slag // In book: Environmental Geochemistry: Site
Characterization, Data Analysis and Case Histories, 2018. -P. 487-519.

https://doi.org/10.1016/B978-0-444-63763-5.00020-3

2.Piatak N. M., Ettler V. (ed.). Metallurgical Slags: Environmental
Geochemistry and Resource Potential -- Royal Society of Chemistry,
2021.- 306 p. DOI:10.1039/9781839164576

3.Gaskell D. R. Chapter fourteen-The determination of phase diagrams for
slag systems //Methods for Phase Diagram Determination.-Elsevier Science
Ltd, 2007. - P. 442-458.

https://doi.org/10.1016/B978-008044629-5/50014-8

4.Piatak N. M., Parsons M. B., Seal II R. R. Characteristics and
environmental aspects of slag: A review //Applied
Geochemistry.-2015.-Т.57. - P.236-266. DOI
10.1016/j.apgeochem.2014.04.009

5.Parron-Rubio M.E., Perez-Garcia F., Gonzalez-Herrera A., Oliveira
M.J., Rubio-Cintas M.D. Slag substitution as a cementing material in
concrete: mechanical, physical and environmental properties // Materials
(Basel). -2019. - Vol.12(18).- P. 2845. DOI 10.3390/ma12182845.

6.Ren Z, Li D. Application of steel slag as an aggregate in concrete
production: a review // Materials (Basel).- 2023.-Vol.16(17). - P.5841.
DOI 10.3390/ma16175841.

7.Gaskell D. R. The determination of phase diagrams for slag systems //
Methods for phase diagram determination. -Elsevier Science Ltd.-2007.
-P. 442-458. DOI 10.1016/B978-008044629-5/50014-8.

8.Mahdi Safhi A., Rivard P., Yahia A. Durability and transport
properties of SCC incorporating dredged sediments //Construction and
Building Materials.- 2021.-Vol.288.- P.123116. DOI
10.1016/j.conbuildmat.2021.123116.

9.Jang K., Xiaadong M., Zhu J., Xu H. Phase Equilibria in the System
``FeO''-CaO-SiO\textsubscript{2}-Al\textsubscript{2}O\textsubscript{3}-MgO
with CaO/SiO\textsubscript{2} 1.3 // ISIJ
International.-2016.-Vol.56(6).-P.967-976.

DOI 10.2355/isijinternational.ISIJINT-2016-099

10.Zhang Y. The effect of slag chemistry on the reactivity of synthetic
and commercial slags //Construction and Building Materials. -2022. -Vol.
335:127493

DOI 10.1016/j.conbuildmat.2022.127493.

11.Tsakiridis P., Papadimitriou G., Tsivilis S., Koroneos C., Tsakiridis
P. Utilization of steel slag for Portland cement clinker production //
Journal of Hazardous Materials. - 2008.- Vol.152.- P.

805--811. DOI 10.1016/j.jhazmat.2007.07.093.

12.Çelik E., Nalbantoglu Z. Effects of ground granulated blastfurnace
slag (GGBS) on the swelling properties of lime-stabilized
sulfate-bearing soils // Engineering Geology.-2013.-Vol.163.-P.20-25.

DOI 10.1016/j.enggeo.2013.05.016.

13.Shi C., Hu S. Cementitious properties of ladle slag fines under
autoclave curing conditions // Cement and Concrete Research. - 2003. -
Vol. 33. - P. 1851-1856.

DOI 10.1016/S0008-8846(03)00211-4.

14.Pellegrino C., Cavagnis P., Faleschini F., Brunelli K. Properties of
concretes with Black/Oxidizing Electric Arc Furnace slag aggregate //
Cement and Concrete Composites.- 2013.- Vol.37. -P. 232-240.

DOI:10.1016/j.cemconcomp.2012.09.001

15.GOST 5382-2019. Tsementy i materialy tsementnogo proizvodstva. Metody
khimicheskogo analiza.

-M: Standartinform, 2019. -65 s. {[}in Russian{]}

16.GOST 25094 -- 2015. Dobavki aktivnye mineral\textquotesingle nye dlya
tsementov. Metod opredeleniya aktivnosti. -M: Standartinform, 2019. -5
s. {[}in Russian{]}

17.GOST 3476-2019. Shlaki domennye i elektrotermofosfornye
granulirovannye dlya proizvodstva tsementov. - M: Standartinform, 2019.
-7 s. {[}in Russian{]}
\end{noparindent}

\emph{{\bfseries Information about authors}}

\begin{noparindent}
Romanov V. - Candidate of Technical Sciences, associate professor of NLC
«Karaganda Industrial University», Temirtau, Kazakhstan, e-mail:
v.romanov@tttu.edu.kz;

Merkulov V. - Candidate of Chemical Sciences, associate professor of NLC
«Karaganda Industrial University», Temirtau, Kazakhstan, e-mail:
v.merkulov@tttu.edu.kz;

Kabiyeva S. - Candidate of Chemical Sciences, associate professor of NLC
«Karaganda Industrial University», Temirtau, Kazakhstan,e-mail:
kabieva.s@mail.ru;

Bestembek E. - Candidate of Technical Sciences, associate professor of
NLC «Karaganda Industrial University», Temirtau, Kazakhstan,e-mail:
ye.bestembek@tttu.edu.kz;

Zhaslan R. - Doctor of PhD of NLC «Karaganda Industrial
University»,Temirtau, Kazakhstan,e-mail:

r.zhaslan@tttu.edu.kz;

Zhumanazarova G. - master of Technical Sciences, doctoral student of NLC
«Karaganda Industrial University», Temirtau, Kazakhstan,e-mail:
g.zhumanazarova@tttu.edu.kz.
\end{noparindent}

\emph{{\bfseries Сведения об авторах}}

\begin{noparindent}
Романов В. -кандидат технических наук, доцент НАО «Карагандинский
индустриальный университет», Темиртау, Казахстан,e-mail:
v.romanov@tttu.edu.kz;

Меркулов В. - кандидат химических наук, доцент НАО «Карагандинский
индустриальный университет», Темиртау, Казахстан,e-mail:
v.merkulov@tttu.edu.kz;

Кабиева С. - кандидат химических наук, доцент НАО «Карагандинский
индустриальный университет», 101400.Темиртау, Казахстан,e-mail:
kabieva.s@mail.ru;

Бестембек Е. - кандидат технических наук, доцент НАО «Карагандинский
индустриальный университет», 101400.Темиртау, Казахстан,e-mail:
ye.bestembek@tttu.edu.kz;

Жаслан Р.- Доктор РhD НАО «Карагандинский индустриальный
университет»,Темиртау, Казахстан,e-mail: r.zhaslan@tttu.edu.kz;

Жуманазарова Г. - mагистр технических наук, докторант НАО
«Карагандинский индустриальный университет»,Темиртау, Казахстан,e-mail:
g.zhumanazarova@tttu.edu.kz
\end{noparindent}
