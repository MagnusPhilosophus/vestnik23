IRSTI 50.47.31

{\bfseries RESEARCH AND CONSTRUCTION OF A MATHEMATICAL MODEL USING DISCRETE
PROGRAMMING METHODS FOR THE MIXING AND MELTING OF COPPER CONCENTRATES}

{\bfseries U. Imanbekova\textsuperscript{1}, A.
Kalizhanova\textsuperscript{2,1}, A.Kozbakova\textsuperscript{2,3}, A.
Imanbekova\textsuperscript{4}, A. Utegenova\textsuperscript{1,2}}

\textsuperscript{1}Almaty University of Power Engineering and
Telecommunications named after G.Daukeyev, Almaty, Kazakhstan,

\textsuperscript{2}Institute of Information and Computational
Technologies CS MSHE RK, Almaty, Kazakhstan,

\textsuperscript{3}Almaty Technological University, Almaty, Kazakhstan,

\textsuperscript{4}Taraz Regional University named after M.Kh. Dulaty,
Taraz, Kazakhstan,

е-mail: uli.08@mail.ru

This paper describes the construction of an optimal schedule for a
metallurgical workshop using discrete programming, and in particular,
the method of branches and boundaries. In this regard, a heuristic
approach to solving this problem is proposed, which includes: building
an initial graph that satisfies the constraints of the optimization
problem, as well as sequential search optimization of the initial graph
according to a given criterion. To test the effectiveness of using the
task of forming a melting schedule as part of the operational control
subsystem in the converter department, experimental studies were
conducted. Considering that the creation of the automated process
control system of the copper plant provides for the introduction of
systems using modern software and hardware controls, cargo flow control,
and an automated conversion process control system, the efficiency of
using optimal melting schedules will significantly increase.

{\bfseries Keywords.} Copper raw materials, blending, smelting.

\begin{quote}
{\bfseries МЫС КОНЦЕНТРАТТАРЫН ШИХТАУ ЖӘНЕ БАЛҚЫТУ ҮШІН ДИСКРЕТТІ
БАҒДАРЛАМАЛАУ ӘДІСТЕРІМЕН МАТЕМАТИКАЛЫҚ МОДЕЛЬДІ ЗЕРТТЕУ ЖӘНЕ ҚҰРУ}

{\bfseries У.Иманбекова\textsuperscript{1},
А.Калижанова\textsuperscript{2,1}, А.Козбакова\textsuperscript{2,3},
А.Иманбекова\textsuperscript{4}, А.Утегенова\textsuperscript{1,2}}
\end{quote}

\textsuperscript{1}Ғ.Дәукеев атындағы Алматы энергетика және байланыс
университеті, Алматы, Қазақстан,

\textsuperscript{2}Ақпараттық және есептеуіш технологиялар институты ҚР
ҒЖБМ ҒК, Алматы, Қазақстан,

\textsuperscript{3}Алматы технологиялық университеті, Алматы, Қазақстан,

\textsuperscript{4}М.Х. Дулати атындағы Тараз өңірлік университеті,
Тараз, Қазақстан,

е-mail: uli.08@mail.ru

Бұл жұмыста дискретті бағдарламалау арқылы, атап айтқанда, бұтақтар мен
шекаралар әдісімен металлургия цехының оңтайлы графигін құру
сипатталған. Осыған байланысты осы мәселені шешуге эвристикалық тәсіл
ұсынылады, оған мыналар кіреді: оңтайландыру мәселесінің шектеулерін
қанағаттандыратын бастапқы графикті құру, сондай-ақ, берілген критерий
бойынша бастапқы графикті дәйекті іздеу жүйесін оңтайландыру.
Конверторлық бөлімшеде жедел басқару кіші жүйесінің құрамымен балқыту
кестесін қалыптастыру міндетін пайдаланудың тиімділігін тексеру үшін
эксперименттік зерттеулер жүргізілді. Мыс зауытының баж ТП құру заманауи
бағдарламалық-техникалық басқару құралдарын, жүк ағынын бақылауды,
айырбастау процесін басқарудың автоматтандырылған жүйесін қолдана отырып
жүйелерді енгізуді көздейтінін ескере отырып, балқымалардың оңтайлы
кестелерін пайдалану тиімділігі айтарлықтай артады.

{\bfseries Түйін сөздер:} мыс шикізаты, араластыру, балқыту.

\begin{quote}
{\bfseries ИССЛЕДОВАНИЕ И ПОСТРОЕНИЕ МАТЕМАТИЧЕСКОЙ МОДЕЛИ МЕТОДАМИ
ДИСКРЕТНОГО ПРОГРАММИРОВАНИЯ ДЛЯ ШИХТОВКИ И ПЛАВЛЕНИЯ МЕДНЫХ
КОНЦЕНТРАТОВ}

{\bfseries У.Иманбекова\textsuperscript{1},
А.Калижанова\textsuperscript{2,1}, А.Козбакова\textsuperscript{2,3},
А.Иманбекова\textsuperscript{4}, А.Утегенова\textsuperscript{1,2}}

\textsuperscript{1}Алматинский университет энергетики и связи им. Г.
Даукеева, Алматы, Казахстан,

\textsuperscript{2}Институт информационных и вычислительных технологий
КН МНВО РК, Алматы, Казахстан,

\textsuperscript{3}Алматинский технологический университет, Алматы,
Казахстан,

\textsuperscript{4}Таразский региональный университет им.М. Х. Дулати,
Тараз, Казахстан,
\end{quote}

е-mail: uli.08@mail.ru

В данной работе описывается построение оптимального графика
металлургического цеха с помощью дискретного программирования, и в
частности, методом ветвей и границ. В сзязи с этим предлагается
эвристический подход к решению данной задачи включающий в себя:
построение исходного графика, удовлетворяющего ограничениям
оптимизационной задачи, а также последовательную поисковую оптимизацию
исходного графика по заданному критерию. Для проверки эффективности
использования задачи формирования графика плавки с составе подсистемы
оперативного управления в конверторном отделении были проведены
экспериментальные исследования. Учитывая, что созданием АСУ ТП медь
завода предусматривается внедрение систем с применением современных
программно-технических средств управления, контроля грузопотоков,
автоматизированной системы управления процессом конвертирования,
эффективность использования оптимальных графиков плавок существенно
повысится.

{\bfseries Ключевые слова.} Медное сырье, купажирование, плавка.

{\bfseries Introduction.} The solution of the formulated problem of
constructing an optimal graph using discrete programming methods, and in
particular, the method of branches and boundaries, involves a large
amount of calculations, difficult when implementing the system. At the
first stage of the procedure, an initial graph is constructed that
satisfies all the constraints of the task. i.e., it is valid {[}1{]}.

Restoration of the process state in each unit of the site (at the time
of the start of the algorithm) based on the initial information and
using a mathematical model of the process: formation of the initial
section of the graph by entering the first heats of each unit into the
Gantt table, calculation based on a mathematical model of the
characteristics of an exemplary melting with a given average blast
consumption and an average number of buckets of loaded matte, assignment
of the characteristics of all swimming trunks (except the first ones) to
the values of the characteristics of an exemplary melting, filling in
the Gantt table with swimming trunks, taking into account the
restrictions and if the condition is met, the first stage of the
procedure is completed {[}2{]}.

Checking the possibility of changing the average flow rate of the blast
in the required direction and calculation of the required value by which
the average blast consumption must be changed and adjustment of the
average blast consumption by this value. Checking the possibility of
changing the matte loading of the smelts {[}3{]}. If possible, the
transition to is carried out, otherwise at the end of the first stage of
the procedure. The choice of melting, which can change the matte loading
in the required direction and adjust the number of loaded matte buckets
{[}4{]}.

{\bfseries Materials and methods.} At the second stage of the procedure, a
direct search for the optimal schedule is carried out by step-by-step
improvement of the original schedule. The search step is to select a
pair of swimming trunks for which it is possible to redistribute the
material (matte) or temporary (duration of the first period and downtime
after melting) resource, redistribute the selected resource, evaluate
the newly obtained schedule option and, if unsatisfactory, return to the
old option. The choice of two melting points in the redevelopment of the
schedule, for which it is possible to redistribute the resource {[}5{]}:

1. Adjustment of the characteristics of the selected melts in accordance
with the redistribution performed.

2. Checking for violation of the condition. If this condition of
redistribution is violated, the transition to paragraph 7 is rejected
and made, otherwise to paragraph

3. The schedule (Gantt table) is adjusted in accordance with the
production redistribution of the resource.

4. Evaluation of the change in the criterion ∆F. If ∆F \textless{} 0,
then the step is considered successful and the transition is made to
point 1, otherwise to point 6.

5. Return to the previous version of the schedule (reverse adjustment of
the Gantt table).

6. Return resource redistribution for previously selected swimming
trunks.

7. Checking for a decrease in the value of the criterion after
conducting cycles of redistribution of all types of resources. If the
decrease in the criterion value is not marked, the optimization
procedure is completed, and the cycle is repeated from point 1.

We present an algorithm for forming the work schedule of the converter
department together with a semiotic model, which form a system for
forming the optimal work schedule of the converter section {[}6{]}.

The purpose of the research was: to determine the feasibility of optimal
schedules of converter melts generated by the system; identification of
the reasons causing the deviation of the actual values from the
specified melting schedules (start and end times of operations, the
amount

of processed and received materials, etc.), development of
recommendations to increase the degree of use of optimal melting
schedules {[}7{]}.

The development of an algorithm for operational control of the converter
department involves testing the control algorithm together with the
control object. However, conducting experiments with an algorithm that
has not yet been debugged at the facility is not possible due to the
strenuous operation of the separation units in industrial conditions.
The output of the latter\textquotesingle s operating modes beyond the
scope of the regulated one entailed significant production losses. In
addition, there are difficulties caused by the need to simultaneously
measure a large number of parameters involved in evaluating the
operability and effectiveness of the control algorithm. In real
production conditions, this can only be done with a long delay, low
accuracy and insufficient data collection frequency {[}8{]}.

{\bfseries Results and discussion.} In these conditions, it is advisable to
debug the control algorithm based on experiments with a simulation model
of the converter compartment {[}9{]}. The process of forming and using a
simulation model includes the following steps: clarification of
information flows into the modeling system and schemes of their
transformation; assessment of the probabilistic characteristics of the
material flows of the converter unit in order to form a model of
compensating effects; development of the structure of a software system
that simulates the functioning of a control object based on a
mathematical model of the process, development of a software system
simulating the functioning of an object (with active disturbances) in
conjunction with a control system; conducting simulation experiments
with the software system; evaluation by a specialist of the results of
the simulation experiment; adjustment of the parameters (structure) of
the control algorithm {[}10{]}:

Block 1 organizes a cycle for all units in operation.

Block 2 generates the initial information for simulating the process at
the time interval \emph{τ} in the \emph{i-th} unit of the site. For this
purpose, values characterizing the state of the process at the end of
the \emph{t1} time interval (composition, quantity and temperature of
the mass and slag in the bath), as well as the values of control actions
on the time interval \emph{τ} (flow rate of blast, ore, amount of poured
matte and drained slag, the moment of commencement) are rewritten into
the working array from the general memory area of the system or the end
of blowing). These values are considered as mathematical expectations of
the corresponding values.

To obtain noisy values, each of the listed parameters is exposed to
interference according to the following Equation 1.

\begin{quote}
ʘ\(\ _{j}\)=\(M_{ʘj}\)+ξ\(\ _{ʘj}\) (1)
\end{quote}

where ξ\(\ _{ʘj}\) - is a random variable distributed according to a
normal law with zero mathematical expectation and a variance equal to
the variance of the real "played" parameter obtained by statistical
processing of experimental data.

The value of ξ\(\ _{ʘj}\) - is generated by block 8, which implements a
pseudorandom number sensor with a given distribution law.The
compositions of materials loaded into the converter are also subjected
to noise.

Block 3 contains a mathematical model of the conversion process
described in section 2, and the initial state of the process is
transformed into the final one, taking into account the specified
control actions.

Block 4 generates the output technological parameters of the process
(temperature, amount and composition of the exhaust gas). It is assumed
that the air sucked into the exhaust gas is directly proportional to the
amount of converter gas to the following Equation 2.

\begin{quote}
\(M\left\{ G_{p}\lbrack\tau\rbrack \right\} = K_{p}G_{g}\lbrack\tau\rbrack\)
(2)
\end{quote}

The value of the \(K_{П}\)coefficient was found during the
identification of the mathematical model. In this case, the amount and
composition of the gas mixture is determined by to the following
Equations 3-4:

\begin{quote}
\(G_{m}\lbrack\tau\rbrack\)=\(G_{g}\lbrack\tau\rbrack + G_{p}\lbrack\tau\rbrack\)+ξ\(G_{p}\lbrack\tau\rbrack\)
(3)

\(\alpha_{{so}_{2}}^{m}\lbrack\tau\rbrack = \alpha_{{so}_{2}}^{2}\lbrack\tau\rbrack G_{g}\lbrack\tau\rbrack 100\%/G_{m}\lbrack\tau\rbrack\)
(4)
\end{quote}

In addition, the block updates the values characterizing the state of
the process located in the general memory area of the system, thereby
preparing for the simulation of the next time interval. In conclusion,
we return to block 1 to simulate the operation of the next unit.

Block 5 calculates the parameters of the converter gas flow from the
converter compartment by to the following Equations 5-6:

\begin{quote}
\(G_{m}^{\sum}\lbrack\tau\rbrack =\)
\(\sum_{i = 1}^{n}G_{m}^{i}\lbrack\tau\rbrack\) (5)

\(\alpha_{{so}_{2}}^{\sum}\lbrack\tau\rbrack =\)
\(\sum_{i = 1}^{n}\alpha_{{so}_{2}}^{i}\lbrack\tau\rbrack G_{m}^{i}\lbrack\tau\rbrack\)/\(G_{m}^{\sum}\lbrack\tau\rbrack\)
(6)
\end{quote}

After that, the calculation of the current statistical characteristics
of the total gas flow, the average downtime of the units, the average
deviation of time parameters, melting parameters from those set by the
graph is performed by to the following Equations 7-8:

\begin{quote}
\(\overline{X}\lbrack n\rbrack = \overline{X}\lbrack n - 1\rbrack +\)
\(\frac{1}{n}\) \(\overline{(X}\){[}n{]}-\(\ \overline{X}\){[}n-1{]})
(7)

\(\sigma_{X}^{2}\){[}n{]}=\(\ \sigma_{X}^{2}\){[}n-1{]}+\(\frac{1}{n - 1}\)
\(\left\{ \ X\lbrack n\rbrack - \overline{X}\ \lbrack n - 1\rbrack)^{2} - \sigma_{X}^{2}\lbrack n - 1\rbrack \right\}\ \)
(8)
\end{quote}

The results generated by block 5 are recorded in the general memory area
of the system.

Block 6 is used to simulate random disturbances on the process,
reflecting fluctuations in the flow rate of blast and ore into the
converter, the amount and composition of the loaded matte and other
materials, the duration and moments of the beginning and end of the
smelting purges. For these purposes, a standard procedure for generating
pseudorandom sequences distributed according to a normal law is used.
The variances of the corresponding parameters to be noisy are
transmitted to block 2 and 4 when accessing block 6.

The next stage consists in the development of a software system,
providing simulation of the operation of the control object together
with the control system, the developed system functions as follows.

Block 1 prepares the initial values of the system time counter τ and the
control time interval l. Block 2 organizes a cycle for all units of the
department that are in operation. Block 3 generates the initial state of
the process in the i-th unit necessary to start the system in operation,
the following parameters are generated: the start time of the current
melting, the average consumption since the beginning of the current
melting, the number of matte buckets loaded into the current melting by
the time the system starts, the compositions of materials loaded into
the current melting. This sets not only the initial conditions for the
i-th unit, but also the integral control actions applied before the
start of the system.

Block 4 generates a micro description of the current technological
situation at the site in the language of a systematic model, thereby
preparing information for the operation of the situational management
unit. Information for the formation of a micro description is selected
from the general memory area of the system sequentially for each unit of
the site.

Block 5 generates a solution for managing the converter department based
on the semiotic model of the latter. Depending on the chosen solution, a
transition can be made to block 6, 8 or 9 (AB). The selected solution(s)
are recorded in the general memory area of the system.

Block 6 generates the optimal work schedule of the department for a
given time interval, taking into account the management decisions
developed in block 5. The generated schedule is fixed in the general
memory area of the system. Block 7 organizes the cycle according to the
melts scheduled in block 6 for a specified time interval. Block 8
generates the optimal schedule for the next melting, in addition, if
block 5 makes a decision to adjust the schedule of some melting, it
generates a new schedule for it, taking into account the changed
situation in the department.

Block 9 (AB) contains a simulation model of the converter compartment.

Block 10 increases the values of the accounts \emph{τ} and l by one,
preparing the output for the next simulation cycle. Block 11 analyzes
the value of the system time. If the simulation of the next shift has
ended, a transition is made to block 12, otherwise to block 14. Block 12
prints the results of the simulation of functioning during the last
shift. Block 13 resets the counter values to zero, thereby switching to
simulating the next shift. Block 14 analyzes the value of the counter l.
If the l value is equal to the control cycle time set when the system is
started, the transition is made to block 15, otherwise to block 9. Block
15 resets the l counter values to zero.

The developed simulation system allows checking the operability and
effectiveness of the proposed system of operational management of the
converter department in various modes of its operation. The data
obtained as a result of the next cycle of the experiment is evaluated by
a specialist who makes a conclusion about the quality of the control
system and decides whether to adjust the control algorithm aimed at
eliminating the identified shortcomings, or to end the experiment if
satisfactory results are obtained, both the structure of the control
algorithm and the values of its objects, the field can be adjusted then
a series of experiments is repeated.

Simulation modeling of the control system showed the operability of the
system and the possibility of a significant increase in the efficiency
of technological processes of the metallurgical workshop during its
implementation presented in Table 1.

{\bfseries Table 1 - System performance results}

\begin{longtable}[]{@{}
  >{\raggedright\arraybackslash}p{(\columnwidth - 6\tabcolsep) * \real{0.1087}}
  >{\raggedright\arraybackslash}p{(\columnwidth - 6\tabcolsep) * \real{0.3184}}
  >{\raggedright\arraybackslash}p{(\columnwidth - 6\tabcolsep) * \real{0.2866}}
  >{\raggedright\arraybackslash}p{(\columnwidth - 6\tabcolsep) * \real{0.2863}}@{}}
\toprule\noalign{}
\begin{minipage}[b]{\linewidth}\raggedright
№
\end{minipage} & \begin{minipage}[b]{\linewidth}\raggedright
The name of the indicator
\end{minipage} & \begin{minipage}[b]{\linewidth}\raggedright
Significance in current practice
\end{minipage} & \begin{minipage}[b]{\linewidth}\raggedright
Importance in situational management
\end{minipage} \\
\midrule\noalign{}
\endhead
\bottomrule\noalign{}
\endlastfoot
1 & Dispersion of SO content in converter gases & 2,89 & 0,36 \\
2 & Downtime variance & 212 & 36 \\
\end{longtable}

The functioning of the operational management system of the converter
department is aimed at the formation of work schedules of the
department, which regulate the conditions of the process and are issued
as written instructions to the technological staff. It is shown above
that the algorithm for forming the work schedule of the converter
department ensures that the schedule is optimal from the point of view
of coordinating the operation of individual units and is most
appropriate to the current situation in the metallurgical workshop. The
size of the planned period in the formation of the
department\textquotesingle s work schedule is determined by the
established management practice, is limited by the dimension of the
optimization task solved at the Central computer and is assumed to be
equal to one day or less (depending on the moment of the beginning of
planning). The results obtained in this case could be issued for
execution once a day. However, the non-stationarity of the control
object, as a rule, leads to the need to revise the schedule during the
day. The main reasons for non-stationarity are:

- failures of the converter compartment units, resulting in the need to
force the operating modes to perform the production program with a
smaller number of operating units;

- deviations of the qualitative and quantitative characteristics of the
raw materials of copper matte) from the regulated norms due to
violations in the operation of the electric furnace and charge
departments, leading to the need to force or artificially reduce the
pace of operation of the converter department;

- disturbances in the operation of the electric furnace compartment,
leading to excessive downtime of the converter compartment units,
leading to excessive downtime of the converter compartment units in
anticipation of the next portions of matte.

- the effect of the "human factor" - staff errors in fulfilling planned
schedules, the inability to reach the planned blast consumption due to a
decrease in the intensity of the work of the shapers, etc.;

- the drift of the characteristics of aggregates due to their aging, the
listed factors, on the one hand, lead to significant deviations from the
specified schedule, which do not allow using it in the future, on the
other hand, to the inadequacy of the mathematical models underlying the
algorithm for selecting modes and, consequently, to the non-optimality
or even inadmissibility of the management decisions being formed. Under
these conditions, the effectiveness of the control system can be ensured
by introducing a rational frequency of submission of control actions, as
well as on the basis of the application of the principle of adaptation,
involving the introduction of feedback on the structure and parameters
of the control system.

Under these conditions, the appropriate task is to select algorithms for
adapting the mathematical model of the control object and to develop a
general algorithm for functioning that regulates the sequence and
frequency of operation of individual algorithmic blocks of the control
system.

The mathematical model of the control object is presented, as well as
the structure of the predicate system. The adaptation of the control
system should be reduced to adjusting the parameters of the analytical
model and the structure of the predicate system based on current
information about the functioning of the facility and the control
system. The first task is solved on the basis of the use of
probabilistic interactive adaptation algorithms, the type and
characteristics of which should take into account the statistical
features of the current information and is selected during a special
study. The algorithm of the current correction of the structure of the
predicate system is based, ultimately, on the correction of the values
of the weighting coefficients appearing as arguments in the
corresponding predicates, which, in turn, connect the concept of
different levels of description.

The structure of the general algorithm for the operation of the control
system. The main blocks of this algorithm are:

Block 1 converts the information generated by the subsystem of
centralized control of the automated control system into a form
convenient for further use (in other words, the block restores a
micromodel of the current situation in terms of language)
\(\left\{ a_{i}^{j} \right\}\), \(\left\{ \tau v \right\}\).

Block 2 performs, based on the predicate system
\(\left\{ P_{S}^{K} \right\}\)\emph{,} the selection of a management
solution \emph{U} that is optimal for this situation. If the developed
solution is immediately ready for implementation, then it is issued to
the site for execution, otherwise other blocks of the system are
launched in order to specify the sequential one.

Block 3 is started if the management decision relates to the formation
(adjustment) of the work schedule of the converter department. At the
same time, the initial state of the converter department is fixed and,
taking into account the previously developed management decision of the
department. At the same time, the initial state of the converter
department is fixed and, taking into account the previously developed
management decision, \(U_{\Psi}\)the formation of a work schedule for
the converter department for a day (until the end of the day from the
moment the unit is put into operation);

Block 4 implements a system of equations, which is an analytical
mathematical model of the conversion process.

The set of blocks 1-4 is a system of operational control of the
converter department by deviation. The frequency of operation of the
system units is 3-4 hours, which corresponds to the dynamics of the most
"rapid" disturbances (failures of aggregates, the action of the "human
factor", etc.), the control object in this case is considered as
quasi-stationary. Accounting for the non-stationarity of the control
object, which allows to increase the efficiency of the control system,
is provided by the introduction of blocks 5-7. Block 5, comparing the
characteristics of converter melts obtained on the basis of an
analytical model with the fixed block 1, adjusts the parameters of the
model by implementing a probabilistic interactive algorithm by to the
following Equation 9:

\begin{quote}
\(A\lbrack n\rbrack = A\lbrack n - 1\rbrack + \gamma\lbrack n\rbrack\nabla_{A}\lbrack Z^{0}\lbrack n\rbrack - Z(Q\lbrack n\rbrack,A\lbrack n - 1\rbrack)^{2}\)
(9)
\end{quote}

where A is the vector of model parameters; Z\textsuperscript{0}-output
of the control object; Z- output of the model; \emph{Q} is the output of
disturbances; \emph{U} is the vector of control actions;

Block 6, based on the process model (block 4), digitally simulates a
certain production situation in the converter department, sequentially
simulates the adoption of each of the available management decisions by
to the following Equation 10:

\begin{quote}
\(U_{\Psi}\)(Ψ=1,2\ldots) (10)
\end{quote}

based on the criterion \emph{Q(x,u)} the effectiveness of each
management decision is evaluated, after which the optimal management, in
the sense of the specified criterion, is selected; Block 7 captures the
situation \emph{Х{[}n{]},} generated by block 6, selects the optimal
management solution based on the predicate system
\(\left\{ P_{S}^{K} \right\} \rightarrow {\overline{U}}_{on}\lbrack n\rbrack\),
compares both U\textsubscript{on}{[}n{]} and
\({\overline{U}}_{on}\lbrack n\rbrack\) in case of their discrepancy,
corrects the predicate system in order to eliminate the resulting
mismatch.

The implementation of blocks 5-7 is associated with significant
computational difficulties. The need to launch blocks 5-7 is due to
changes in technology (for example, the transition to new types of raw
materials) or changes in the organization of work and should be carried
out with a periodicity of about a month.

{\bfseries Conclusion.} In the process of long-term operation of the system
for the operational formation of an optimal work schedule for the
converter department, the need to improve individual units of the system
has been identified. In particular, the technological staff of the
metallurgical workshop expressed the wish that it was necessary to take
into account the raw materials and energy limitations of the electric
furnace electronic department (FED).

To implement these wishes, the following changes were made to the
algorithm for forming the work schedule of the converter department: a
fragment of an algorithm has been developed that corrects a given plan
for the production of rough copper, taking into account the energy and
raw material constraints of the FED, as well as the schedule of the
latter\textquotesingle s PPD; the procedure for creating a schedule at
time intervals corresponding to the FED\textquotesingle s PPR has been
changed. At these intervals, it is planned to reduce the intensity of
the converters, the melts are spaced over time so that they can be
provided with matte from one furnace; an algorithm has been developed
for the formation of an application plan for the issuance of a matte FED
with the development of the latter for shift tasks, which specify the
time the output of each stein bucket, as well as the total number of
buckets per shift and per day. The required number of pellets to be
processed and the cost of electricity are also indicated.

The listed changes and additions are made in the form of separate
fragment programs and independent programs and will be included in the
system software being developed.

\emph{{\bfseries Acknowledgment.}} \emph{Research were carried out within
the framework of the Grant Financing Project No. AP19679153 ``Research
and development of method and technologies for creating composite
structures with built-in photonic sensors PSBC (Photonic Smart Bragg
Composites)'' Institute of Information and Computational Technologies of
the Science Committee of the Ministry of Science and Higher Education of
the Republic of Kazakhstan.}

\emph{{\bfseries Sources of funding for research presented in a scientific
article or scientific article itself.}}

This work is supported by grant from the Ministry of Science and Higher
Education of the Republic of Kazakhstan within the framework of the
Grant Financing Project No. AP19679153 ``Research and development of a
method and technology for creating composite structures with built-in
photonic sensors PSBC (Photonic Smart Bragg Composites)'', Institute of
Information and Computational Technologies CS MSHE RK.

{\bfseries References}

1.Malfliet A., Lotfian S., Scheunis L., Petkov V., Pandelaers L., Jones
P.T., Blanpain B. Degradation mechanisms and use of refractory linings
in copper production processes // Journal of the European Ceramic
Society. A critical review.- 2014.- Vol.34, Iss.3. - P. 849-876.
https://doi.org/10.1016/j.jeurceramsoc.2013.10.005

2.Cheng P., Herreros P., Lalpuria M., Grossmann I. Optimal scheduling of
copper concentrate operations under uncertainty.//Computers \& Chemical
Engineering.-2020.- Vol.140, 106919.
doi.org/10.1016/j.compchemeng.2020.106919

3.Guimarães F.Y., Santos I.D., Dutra J.B. Direct recovery of~copper~from
printed circuit boards (PCBs) powder~concentrate~by a simultaneous
electroleaching--electrodeposition~process.//
Hydrometallurgy.-2014.-Vol. 149.- P. 63-70.
https://doi.org/10.1016/j.hydromet.2014.06.005

4.Zhai Q.,Runqing Lui. Simultaneous recovery of arsenic and copper from
copper smelting slag by flotation: Redistribution behavior and toxicity
investigation. Journal of Cleaner Production. -2023.-Vol. 425, 138811
https://doi.org/10.1016/j.jclepro.2023.138811

5.Imanbekova U., Hotra O., Koshimbayev S. K., Popiel P., Tanas J.
Optimal control of blending and melting of copper concentrates.//
Proceedings of SPIE-The International Society for Optical Engineering
.-2015, 966246. DOI:10.1117/12.2205446

6.Rubanenko O. O., Komar V. O., Petrushenko O. Y., Smolarz A., Smailova
S., Imanbekova U., Determinition of similatiry criteria in optimization
tasks by means of neuro-fuzzy modelling. //Journal Przegląd
Elektrotechniczny.-2017.-Vol. 1(3).- P.95-98.
DOI:10.15199/48.2017.03.226.

7. Hotra O.Z., Koshimbayev S.K., Imanbekova U.N. Modelling in Matlab
using fuzzy logic for improving the economic factors of melting of
copper concentrate charge.// Actual problems of
economics.-2014.-Vol.11.-P. 380-387.

8. Da-wei~Wang,~Yan-jie~Liang. Comprehensive recovery of zinc, iron and
copper from copper slag by co-roasting with SO2--O2./ Journal of
Materials Research and Technology.-2022.- Vol.19.-P.2546-2555.
https://doi.org/10.1016/j.jmrt.2022.05.177

9. Istadi I., Bindar Y. Improved cooler design of~electric~arc furnace
refractory in mining industry using thermal analysis modeling and
simulation.//Applied Thermal Engineering.- 2014.- Vol.73, Iss.1.-
P.1129-1140. https://doi.org/10.1016/j.applthermaleng.2014.08.070

10.Imanbekova U., Hotra O., Koshimbayev S., Optimal control of copper
concentrate blending and melting based on intelligent systems.// Journal
Przegląd Elektrotechniczny.2016.-Vol.8.- P.125-128.
doi:10.15199/48.2016.08.34

\emph{{\bfseries Information about authors}}

Ulzhan Imanbekova - PhD, Associate Professor, Almaty University of Power
Engineering and Telecommunications named after G.Daukeyev, Almaty,
Kazakhstan, e-mail: uli.08@mail.ru;

Aliya Kalizhanova- Professor, Institute of Information and Computational
Technologies CS MSHE RK, Almaty University of Power Engineering and
Telecommunications named after G.Daukeyev, Almaty, Kazakhstan,e-mail:
kalizhanova.aliya@gmail.com;

Ainur Kozbakova- PhD, Institute of Information and Computational
Technologies CS MSHE RK, Almaty Technological University, Almaty,
Kazakhstan, e-mail: ainur79@mail.ru;

Aliya Imanbekova- Senior Lecturer, M. H. Dulati Taraz Regional
University, Taraz, Kazakhstan, e-mail: aleka.12@mail.ru;

Anar Utegenova -PhD, Institute of Information and Computational
Technologies CS MSHE RK, Almaty University of Power Engineering and
Telecommunications named after G.Daukeyev, Almaty, Kazakhstan, e-mail:
an.utegenova@aues.kz

\emph{{\bfseries Сведения об авторах}}

Иманбекова У. -PhD, ассоциированный профессор, Алматинский университет
энергетики и связи им. Г. Даукеева, Алматы, Казахстан, e-mail:
uli.08@mail.ru;

Калижанова А.- к.ф.-м.н., профессор, Институт информационных и
вычислительных технологий КН МНВО РК. Алматинский университет энергетики
и связи им. Г. Даукеева, Алматы, Казахстан, e-mail:
kalizhanova.aliya@gmail.com;

Козбакова А. - PhD, Институт информационных и вычислительных технологий
КН МНВО РК, Алматинский технологический университет, Алматы, Казахстан,
e-mail: ainur79@mail.ru;

Иманбекова А.-старший преподаватель, Таразский региональный университет
им.М. Х. Дулати, Тараз, Казахстан, e-mail: aleka.12@mail.ru;

Утегенова А. -- PhD, Институт информационных и вычислительных технологий
КН МНВО РК. Алматинский университет энергетики и связи им. Г. Даукеева,
Алматы, Казахстан, e-mail: an.utegenova@aues.kz
